%%%%%%%%%%%%%%%%%%%%%%%%%%%%%%%%%%%%%%%%%%%%%%%%%%%%
\subsection{Retirer la priorité}
%%%%%%%%%%%%%%%%%%%%%%%%%%%%%%%%%%%%%%%%%%%%%%%%%%%%%

\begin{longtblr}{colspec = {X[3,l] X[7,l] X[5,l]}, rowsep = 1pt}

\textbf{retrait prio.=} & & \\
\hline
\textbf{lieu et \mydots} & \textbf{où?} & \\

 & -- à l'IB, au RO, ou à l'IPEA
 & \pctrule{90bis}.3.c) \\

\SetHline{2-3}{}
\textbf{actes et \mydots} & \textbf{quoi?} & \\

 & -- cf. modalités de retrait
 & \\

\SetHline{2-3}{}
\textbf{\mydots delai} & \textbf{quand?} & \\

principe
 & -- 30 mois / date de priorité 
 & \pctrule{90bis}.3.a) \\

\ok{retarder la publication}
 & -- retrait de la revendication de priorité la plus ancienne
 avant l'achèvement de la préparation technique de la publication
 internationale
 & \pctrule{90bis}.3.d) \\

\nok{pas retard publication}
 & -- ... après l'achèvement de la préparation technique de la publication
 internationale
 & \pctrule{90bis}.3.e) \\

\hline
\textbf{effets} & \textbf{lesquels?} & \\
 & -- le document de priorité retiré ne sera pas accessible aux tiers
 & \pctrule{17}.2.c.ii) \\

\end{longtblr}