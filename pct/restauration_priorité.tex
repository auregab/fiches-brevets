%%%%%%%%%%%%%%%%%%%%%%%%%%%%%%%%%%%%%%%%%%%%%%%%%%%%
\subsection{Restaurer la priorité}
%%%%%%%%%%%%%%%%%%%%%%%%%%%%%%%%%%%%%%%%%%%%%%%%%%%%%

\begin{longtblr}{colspec = {X[3,l] X[7,l] X[5,l]}, rowsep = 1pt}

\textbf{restaurer prio.=} & & \\

\hline
\textbf{delai et \mydots} & \textbf{quand?} & \\

dépôt de la DI
& -- 2 mois / date d'expiration du délai de priorité de 12 mois
& \pctrule{26bis}.3.a) \\

tous les actes
& -- 2 mois / date d'expiration du délai de priorité
& \pctrule{26bis}.3.e) \\

& -- avant la fin des préparatifs techniques de publication
lorsque publication anticipée
& GD 9.014\\

\SetHline{2-3}{}
\textbf{\mydots actes} & \textbf{quoi?} & \\

& -- requête en restauration du droit de priorité auprès du RO
& \pctrule{26bis}.3.b.i) \\

& -- requête $\in$ motifs pour lesquels la DI a été déposée hors délai
& \pctrule{26bis}.3.b.ii) \\

& -- payer la taxe de restauration du droit de priorité
& \pctrule{26bis}.3.d) \\

& -- présenter toute déclaration ou autres preuves exigées par
le RO
& \pctrule{26bis}.3.b.iii) + \pctrule{26bis}.3.f) \\

premier critère
 & -- toute la diligence requise a été exercée et/ou
 & \pctrule{26bis}.3.a.i) \\

deuxième critère
 & -- l'inobservation du délai n'a pas été intentionnelle
 & \pctrule{26bis}.3.a.ii) \\

\hline
\textbf{effets} & \textbf{lesquels?} & \\

\ok{restauration}
 & -- restauration produit ses effets dans tous les Etats désignés
 qui ont le même critère ou un critère plus favorable
 & \pctrule{49ter}.1.a) ou \pctrule{49ter}.1.b) \\ 

\nok{pas restauration}
 & -- réexamen de la décision du RO est entrepris par OD/OE
 & \\

\end{longtblr}