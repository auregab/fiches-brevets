\section{Anticiper les exigences nationales}

\begin{longtblr}{colspec = {X[3,l] X[7,l] X[5,l]}, rowsep = 1pt}

\textbf{exigences nat. =} & & \\
\hline
\textbf{général et \mydots} & \textbf{quoi?} & \\

\ok{permise par le PCT}
 & -- exigences de forme et de contenu plus favorables au déposant  
 & \pctart{27}(4) \\

 & -- preuves à l’appui des allégations ou déclarations  
 & \pctart{27}(2) \\

 & -- mesures relatives à la défense nationale  
 & \pctart{27}(8) \\

 & -- mesures de protection des intérêts économiques de l’État  
 & \pctart{27}(8) \\

%  & -- définition de l’état de la technique
%  & \pctart{27}, \pctrule{51bis}, \pctrule{76} \\
%  & -- traduction certifiée en cas de doute raisonnable
%  & \\
\nok{exclues du PCT}
 & -- exigence nationale différente
    de celle prévue par le PCT
 & \pctart{27} \\

 & -- exigence nationale supplémentaire
    non prévue par le PCT
 & \\

\SetHline{2-3}{}
\SetCell[r=2]{valign=h}\textbf{brevetabilité et \mydots} \newline \ok{permises par le PCT}
 &
 & \\

 & -- traduction confirmée ou certifiée de la demande internationale  
 & \pctrule{51bis}.1(d) \\

 & -- traduction du document de priorité lorsque la validité de la priorité est pertinente  
 & \pctrule{51bis}.1(e)(ii) \\

 & -- justification de divulgations non opposables ou d’exceptions au défaut de nouveauté  
 & \pctrule{51bis}.1(a)(v) \\

\SetHline{2-3}{}
\SetCell[r=2]{valign=h}\textbf{traduction et \mydots} \newline \ok{permises par le PCT en cas d'erreur} & & \\

 & -- correction ou rectification de la traduction sans dépasser la portée du texte d’origine
 & \pctart{46} \\

\SetHline{2-3}{}
\SetCell[r=2]{valign=h} \textbf{restauration priorité et \mydots} \newline \ok{reconnaissance} & & \\
%  & -- incorporation par renvoi des éléments ou parties manquantes, sous réserve d’absence d’incompatibilité nationale
%  & \pctrule{20.8(b)}+\pctrule{82ter} \\

 & -- l’office désigné applique le critère du caractère non intentionnel
 & \pctrule{49ter}.1(a) \\

\nok{pas reconnaissance} & -- l’office désigné applique un critère plus strict que celui de l’office récepteur
 & \pctrule{49ter}.1(b) \\

\nok{obligation pour déposant}
 & (i) nouvelle requête en restauration du droit de priorité lorsque l’office désigné et l’office récepteur ont notifié une incompatibilité nationale
 & \pctrule{49ter}.2(a) \\

 & (ii) nouvelle requête en restauration du droit de priorité lorsque seul l’office désigné a notifié une incompatibilité nationale
 & \pctrule{49ter}.2(a) \\

\SetHline{2-3}{}
\SetCell[r=2]{valign=h} \textbf{\mydots rectification d'erreur} \newline \nok{DO peut ignorer} & & \\

 & -- le traitement ou l’examen avait déjà commencé (ouverture anticipée de la phase nationale)
 & \pctrule{91}.3(f) \\

 & -- l’office désigné ne l’aurait pas autorisée, après possibilité donnée au déposant de présenter des observations
 & \pctrule{91}.3(f) \\

% \SetHline{2-3}{}
% & \textbf{garanties?} & \\
%  & -- un délai additionnel pour respecter
%     les exigences admises
%  & \pctart{27}, \pctrule{17.1(c)}, \pctrule{17.1(d)},
%    \pctrule{49}, \pctrule{51bis} \\
%  & -- un pouvoir de présenter des arguments
%     en cas de non-respect des exigences admises
%  & \\
% \SetHline{2-3}{}
% \textbf{conditions} & \textbf{à vérifier} & \\
% vérifier
%  & -- « Réserves et incompatibilités »
%     sur la page PCT du site Internet de l’OMPI
%  & \\
\end{longtblr}