%%%%%%%%%%%%%%%%%%%%%%%%%%%%%%%%%%%%%%%%%%%%%%%%%%%%
\section{Requête en révision}
%%%%%%%%%%%%%%%%%%%%%%%%%%%%%%%%%%%%%%%%%%%%%%%%%%%%%

\begin{longtblr}{colspec = {X[3,l] X[7,l] X[5,l]}, rowsep = 1pt}

\textbf{requête en révision =} & & \\

\hline
\textbf{juridique} & \textbf{fondement?} & \\

\ok{recevable}
 & -- question de droit soumise par les chambres de recours
 & \textbf{\epcart{112}}(1)a)+\epcart{22}(1)a) \\

 & -- question de droit soumise par le Président de l’OEB
 & \textbf{\epcart{112}}(1)a)+\epcart{22}(1)b) \\

 & -- requête en révision
 & \textbf{\epcart{112}}bis(1)+\epcart{22}(1)c) \\

\hline
\textbf{préjudice} & \textbf{fondement ?} & \\

\ok{admissible}
 & -- conflit d’intérêt
 & \epcart{112bis}(2)a) \\

 & -- un membre n’était pas membre d’une chambre de recours
 & \epcart{112bis}(2)b) \\

 & -- violation du droit d’être entendu
 & \epcart{112bis}(2)c) \\

 & -- vice fondamental de procédure
 & \epcart{112bis}(2)d) \\

 & -- corruption d’un membre
 & \epcart{112bis}(2)e) \\

 & -- absence de mention d’un argument déterminant et évident
    pour le lecteur moyen
 & \epcrule{7}/22 \\

 & -- absence de notification d’une décision d’une chambre de recours
 & \epcrule{7}/09 \\


\nok{non admissible}
 & -- le « droit d’être entendu » n’est pas l’occasion
    de rediscuter le fond des décisions de la chambre de recours
 & \epcrule{1}/08, motif 2.1 \\

 & -- absence de mention d’un argument dans la décision
    $\Rightarrow$ présomption de considération
 & \epcrule{10}/20 \\

 & -- réponse d’une chambre de recours laissée équivoque
 & \epcrule{6}/22 \\

\end{longtblr}
