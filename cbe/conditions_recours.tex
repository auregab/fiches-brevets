%%%%%%%%%%%%%%%%%%%%%%%%%%%%%%%%%%%%%%%%%%%%%%%%%%%%
\section{Former recours}
%%%%%%%%%%%%%%%%%%%%%%%%%%%%%%%%%%%%%%%%%%%%%%%%%%%%%

\begin{longtblr}{colspec = {X[3,l] X[7,l] X[5,l]}, rowsep = 1pt}

\textbf{recours =} & & \\

\hline
\textbf{personne et \mydots} & \textbf{qui?} & \\

\ok{peut former recours} & -- toute partie lesée par une décision de l’OEB & \epcart{107} \\
 & -- deux parties, dont une nouvelle partie après cession d’une partie du brevet EP
 & \epocase{t015819}{T 158/19} \\

\hline
\textbf{\delai{délai} et \mydots } & \textbf{quand?} & \\

\delai{acte de recours} & -- \delai{2 mois à compter de la signification de la décision}
 & \epcart{108} \\

\delai{mémoire de recours} & -- \delai{4 mois à compter de la signification de la décision}
 & \\

\hline
\textbf{fondement et \mydots} & \textbf{argumentation?} & \\

\nok{pas fondée}
 & -- raisonnement non cohérent
 & \epocase{t092205}{T 922/05} \\

 & -- raisonnement insuffisamment détaillé
 & \epocase{t022083}{T 220/83} \\

 & -- absence de report de la décision de recours
 & \epocase{t059215}{T 592/15} \\

\ok{fondée}
 & -- objections surmontées
 & \epocase{t013987}{T 139/87} \\

 & -- refus injustifié
 & \epocase{t006407}{T 64/07} \\

\hline
\textbf{preuves et \mydots} & \textbf{moyens?} & \\

\ok{recevables}
 & -- principe : reprise conjointe des moyens invoqués en 1\textsuperscript{re} instance
 & A12(3) RPCR \\

\nok{pas recevables}
 & -- moyens invoqués en 1\textsuperscript{re} instance mais pas au stade du recours
 & \epocase{t064420}{T 644/20}, exergue 1 \\

 & -- requête qui aurait dû être présentée en 1\textsuperscript{re} instance
 & \epocase{t014409}{T 144/09} \\

\hline
\textbf{\mydots motifs} & \textbf{nouveaux?} & \\

\ok{recevables}
 & -- principe : nouveaux motifs uniquement avec l’accord du titulaire
 & \epocase{g001091}{G 10/91} \\

 & -- nouveau motif soulevé par l’intervenant
 & \epocase{g000194}{G 1/94} \\

 & -- motif de nouveauté sur un ETTP après activité inventive sur l’ETTP
 & \epocase{g000795}{G 7/95}, sommaire \\

 & -- motif d’activité inventive sur un ETTP après nouveauté sur l’ETTP
 & \epocase{t013101}{T 131/01}, sommaire \\

 & -- motif soulevé par un autre opposant
 & \\

\hline
\textbf{effet juridique} & \textbf{non reformatio in pejus?} & \\

\ok{s’applique}
 & -- suppression d’une caractéristique
 & \epocase{g000199}{G 1/99}, sommaire \\

 & -- introduction d’une extension dans la limite de \epcart{123}(3)
 & \epocase{g000199}{G 1/99} \\

\nok{ne s’applique pas}
 & -- introduction d’une limitation supplémentaire conforme à \epcart{123}(3)
 & \\

 & -- recours sur examen
 & \epocase{g001093}{G 10/93}, sommaire \\

\end{longtblr}
