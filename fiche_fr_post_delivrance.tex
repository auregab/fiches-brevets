% Options for packages loaded elsewhere
\PassOptionsToPackage{unicode}{hyperref}
\PassOptionsToPackage{hyphens}{url}
%
\documentclass[
]{article}
\usepackage{amsmath,amssymb}
\usepackage{iftex}
\ifPDFTeX
  \usepackage[T1]{fontenc}
  \usepackage[utf8]{inputenc}
  \usepackage{textcomp} % provide euro and other symbols
\else % if luatex or xetex
  \usepackage{unicode-math} % this also loads fontspec
  \defaultfontfeatures{Scale=MatchLowercase}
  \defaultfontfeatures[\rmfamily]{Ligatures=TeX,Scale=1}
\fi
\usepackage{lmodern}
\ifPDFTeX\else
  % xetex/luatex font selection
\fi
% Use upquote if available, for straight quotes in verbatim environments
\IfFileExists{upquote.sty}{\usepackage{upquote}}{}
\IfFileExists{microtype.sty}{% use microtype if available
  \usepackage[]{microtype}
  \UseMicrotypeSet[protrusion]{basicmath} % disable protrusion for tt fonts
}{}
\makeatletter
\@ifundefined{KOMAClassName}{% if non-KOMA class
  \IfFileExists{parskip.sty}{%
    \usepackage{parskip}
  }{% else
    \setlength{\parindent}{0pt}
    \setlength{\parskip}{6pt plus 2pt minus 1pt}}
}{% if KOMA class
  \KOMAoptions{parskip=half}}
\makeatother
\usepackage{xcolor}
\usepackage{longtable,booktabs,array}
\usepackage{multirow}
\usepackage{calc} % for calculating minipage widths
% Correct order of tables after \paragraph or \subparagraph
\usepackage{etoolbox}
\makeatletter
\patchcmd\longtable{\par}{\if@noskipsec\mbox{}\fi\par}{}{}
\makeatother
% Allow footnotes in longtable head/foot
\IfFileExists{footnotehyper.sty}{\usepackage{footnotehyper}}{\usepackage{footnote}}
\makesavenoteenv{longtable}
\ifLuaTeX
  \usepackage{luacolor}
  \usepackage[soul]{lua-ul}
\else
  \usepackage{soul}
\fi
\setlength{\emergencystretch}{3em} % prevent overfull lines
\providecommand{\tightlist}{%
  \setlength{\itemsep}{0pt}\setlength{\parskip}{0pt}}
\setcounter{secnumdepth}{-\maxdimen} % remove section numbering
\ifLuaTeX
  \usepackage{selnolig}  % disable illegal ligatures
\fi
\IfFileExists{bookmark.sty}{\usepackage{bookmark}}{\usepackage{hyperref}}
\IfFileExists{xurl.sty}{\usepackage{xurl}}{} % add URL line breaks if available
\urlstyle{same}
\hypersetup{
  hidelinks,
  pdfcreator={LaTeX via pandoc}}

\author{}
\date{}

\begin{document}

\begin{enumerate}
\def\labelenumi{\arabic{enumi}.}
\item
  \textbf{\ul{LIMITATION/RENONCIATION}}
\end{enumerate}

\begin{longtable}[]{@{}
  >{\raggedright\arraybackslash}p{(\columnwidth - 4\tabcolsep) * \real{0.2033}}
  >{\raggedright\arraybackslash}p{(\columnwidth - 4\tabcolsep) * \real{0.6364}}
  >{\raggedright\arraybackslash}p{(\columnwidth - 4\tabcolsep) * \real{0.1604}}@{}}
\toprule\noalign{}
\begin{minipage}[b]{\linewidth}\raggedright
\end{minipage} & \begin{minipage}[b]{\linewidth}\raggedright
\textbf{Mesure~?}
\end{minipage} & \begin{minipage}[b]{\linewidth}\raggedright
\textbf{Source}
\end{minipage} \\
\midrule\noalign{}
\endhead
\bottomrule\noalign{}
\endlastfoot
De la limitation & - une ou plusieurs revendications & L613-24 al.1
CPI \\
& \textbf{Qui~?} & \\
Peut limiter & \begin{minipage}[t]{\linewidth}\raggedright
- titulaire\\
- mandataire\\
- consentement des titulaires de droits\strut
\end{minipage} & \\
& \textbf{Acte~?} & L613-24 al.3 \\
\multirow{2}{*}{Valable} & - requête de limitation & \\
& - texte modifié sur la modification la plus récente des procédures
& \\
& - requête lorsque les effets du brevet ont cessé & \\
Pas valable & - texte modifié sur une modification en limitation et pas
en tribunal & Arrêt du 16 mai 2025 de la cour d'appel de Paris, RG
23/11407 \\
& \textbf{Recours~?} & \\
Valable & - recours en annulation pour le titulaire contre une décision
de rejet d'une limitation & \\
Pas valable & - recours en annulation par un tier contre une décision de
rejet d'une limitation & \\
& \textbf{Effets~?} & \\
& - limitation inscrite au RNB & \\
& - rétroaction de la limitation à la date du dépôt de la demande & \\
\end{longtable}

\begin{enumerate}
\def\labelenumi{\arabic{enumi}.}
\setcounter{enumi}{1}
\item
  \textbf{\ul{OPPOSITION (L613-23)}}
\end{enumerate}

\begin{longtable}[]{@{}
  >{\raggedright\arraybackslash}p{(\columnwidth - 4\tabcolsep) * \real{0.2188}}
  >{\raggedright\arraybackslash}p{(\columnwidth - 4\tabcolsep) * \real{0.6379}}
  >{\raggedright\arraybackslash}p{(\columnwidth - 4\tabcolsep) * \real{0.1433}}@{}}
\toprule\noalign{}
\begin{minipage}[b]{\linewidth}\raggedright
\end{minipage} & \begin{minipage}[b]{\linewidth}\raggedright
\textbf{Qui~?}
\end{minipage} & \begin{minipage}[b]{\linewidth}\raggedright
\end{minipage} \\
\midrule\noalign{}
\endhead
\bottomrule\noalign{}
\endlastfoot
Peut former oppo & - tout personne & \\
& - par jonction de procédure & R613-44-3 \\
pas former oppo. & - titulaire du brevet & \\
& \textbf{Procédure~?} & \\
& - accusatoire & L613-23-2 \\
& - contradictoire & \\
& \textbf{Motifs~?} & L613-23-1 \\
& - exclusion à la brevetabilité & \\
& - nouveauté, activité inventive, application industrielle & \\
& - insuffisance de description & \\
& - extension de l'objet & \\
& \textbf{Délai~?} & \\
& - 9 mois / à la publication de la mention de délivrance & \\
& - après / l'entrée en vigueur de la loi PACTE 01/04/2020 & \\
& - pas de recours en restauration pour le délai d'oppo. & L612-16 al.
4 \\
& \textbf{Actes~?} & \\
recevable & - identité de l'opposant & R613-44-1 \\
& - désignation du mandataire / le pouvoir & \\
& - les références du brevet opposé & \\
& - un mémoire comprenant portée, motifs, faits et preuves & \\
& - payer la redevance & \\
& \textbf{Phase d'instruction~?} & R613-44-6 \\
& - notification du mémoire & \\
& - avis d'instruction & \\
& - réponses des parties à l'avis d'instruction & \\
& - dernier échange & \\
& - procédure orale & \\
& \textbf{Modification~?} & L613-23-3 \\
Admissible & - répond à un motif d'opposition & \\
& - conforme à L611-10 etc. & \\
\multirow{2}{*}{Pas admissible} & - étend l'objet du brevet & \\
& - étend la protection conférée par le brevet & \\
& \textbf{Eléments tardif~?} & R613-44-7 \\
\multirow{3}{*}{Non admissible} & - nouveaux motifs d'opposition après
délai d'opposition & \\
& - nouvelle mesure de l'opposition après délai d'opposition & \\
& - élément pas débattu (principe du contradictoire) & \\
Admissible & - conséquence directe de nouvelles observations et/ou
nouvelles modifications & \\
& \textbf{Effets~?} & \\
& - brevet révoqué, maintenu modifié, maintenu & L613-23-4 \\
& - notification de la décision & L411-5 \\
& - effet absolu et rétroactif & L613-23-6 \\
& - effet d'un jugement à titre exécutoire & L613-23-2 \\
& \textbf{Frais~?} & \multirow{4}{*}{L613-23-5} \\
& - 600€ phase écrite \\
& - 100€ phase orale \\
& - 500€ frais de représentation \\
& \textbf{Suspension~?} & L613-44-10 \\
admissible & - action en revendication de propriété du brevet & \\
& - à l'initiative de l'INPI ie. liquidation judiciaire & \\
& - à la demande d'une des parties & L613-44-11 \\
& \textbf{Clôture~?} & \multirow{5}{*}{R613-44-12} \\
Admissible & - tous les opposants ont retiré leur opposition \\
& - brevet déclaré nul par décision de justice passée en force de chose
jugée \\
& - renonciation par l'opposition \\
& - effets du brevet ont cessé \\
& \textbf{/ Nullité~?} & \\
& - \ul{principe~}: action judiciaire \textgreater{} opposition
\textgreater{} limitation & \\
possible & - suspension lorsque action en nullité après formation
opposition & \multirow{2}{*}{R613-44-10} \\
& - surseoir à statuer à la discrétion du juge lorsque formation
opposition après action en nullité \\
& \textbf{/ limitation~?} & \\
clôture & - \ul{principe}~: clôture d'une procédure de limitation à la
date de formation de l'opposition & L613-24 al. 5, R613-45-3 \\
Pas clôture & - une limitation dans une action en nullité à la date de
formation de l'opposition & \\
& - une limitation dans une action en nullité après la date de formation
de l'opposition & L613-24 al. 4 \\
\end{longtable}

\end{document}
