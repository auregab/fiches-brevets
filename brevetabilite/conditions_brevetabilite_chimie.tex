\section{Les conditions de brevetabilité en chimie}

\begin{longtblr}{colspec = {X[3,l] X[8,l] X[5,l]}}

\hline &
\textbf{invention?} &
A52\\

\nok{pas invention} &
-- une découverte &
TGI Paris Evinerude c. Aair Lichens\\

&
\textbf{exception?} &
A53, R28-29\\

\nok{pas brevetable} &
-- méthode thérapeutique: (i) curatif ou (ii) prophylactique
suffit à guérir une partie des symptômes &
T24/91\\

\ok{brevetable} &
-- première application thérapeutique &
A54(5) CBE 1973 \\

&
-- deuxième application thérapeutique &
A54(5) CBE 2000 \\

\nok{pas brevetable} &
-- deuxième application thérapeutique &
A54(5) CBE 1973 \\

\hline \textbf{première utilisation?} &
&
\\

revendication cbe 1973 &
-- "utilisation d'une substance X pour la fabrication
d'un médicament destiné au traitement de la maladie Y." &
G6/83, CA Paris 29/10/2004 \\

revendication cbe 2000 &
-- \delai{à/c 29/01/2011}: "substance ou composition X
pour utilisation dans le traitemetn de la maladie Y" &
G2/08 \\

\hline \textbf{deuxième utilisation?} &
&
\\

\ok{brevetable} &
-- traitement d'une nouvelle maladie &
\\

&
-- nouveau groupe de sujets &
T19/86 \\

&
-- nouveau mode/voie d'administration &
T51/93 \\

&
-- nouvelle posologie laus attention défaut activité inventive
CA Paris 09/07/2025 &
G2/08, pourvoi n°20-15.480 + TJ Paris 28/03/2024 \\

&
-- combinaison avec un autre principe actif &
TJ Paris 20/10/2023 \\

\nok{pas brevetable} &
-- expliquer un effet qui existait avant &
T254/93 \\

&
-- donner un mécanisme d'action $\Rightarrow$
utilisation concrète &
\\

&
\textbf{cosmétique?} &
\\

&
-- utilisation exclusivement non thérapeutiques:
nettoyage de la peau, déodorant, blanchiment des dents &
\\

première forme revendication &
-- utilisation non-thérapeutiques et thérapeutiques séparables:
amincissement, effet sur le système pileux &
T36/83 \\

&
-- utilisation exclusivement thérapeutiques:
traitement de l'acné, de brûlures ou de blessures &
\\

deuxième forme revendication &
-- utilisation non-thérapeutique et thérapeutiques
inttrinséquement liés:
protection solaire, élimination de la plaque dentaire &
CA Paris EMS c. Dentsply, T1635/09\\

\hline \textbf{nouveauté} &
\textbf{divulgation non opposable?} &
\\

&
-- EP: délai 6 mois / la date de dépôt &
G3/98, G2/99 \\

&
-- FR: délai 6 mois / la date de priorité &
\\

&
\textbf{accessibilité au public?} &
\\

&
-- personnes non tenues au secret &
(?) \\

&
-- personnes tenues au secret &
(?) \\

&
\textbf{divulgation d'un produit?} &
\\

analysable &
-- avec les méthodes d'analyse disponibles à la date de priorité &
T2068/15 \\

&
-- "en aveugle" &
CA 28/02/1977 \\

&
-- sans avoir besoin de raisons pour analyser le produit &
G1/92 \\

&
-- toute propriété / composition analysable du produit 
est considérée comme publique &
G1/23 \\

reproductible (critère presque obsolète) &
-- obtenu et possédé physiquement par l'homme du métier,
ce qui est le cas si le produit est sur le marché &
G1/92, G1/23 \\

&
\textbf{divulgation orale?} &
\\

\ok{preuve valable} &
-- caractère éphémère: preuve au-delà de tout doute raisonnable &
T1057/09, T1212/97 \\

\nok{preuve non valable} &
-- pas par le seul conférencier &
\\

&
-- ouvrage rapportant le contenu d'un séminaire antérieur &
CA Paris 28/05/1999 \\

&
\textbf{Document par document?} &
\\

divulgation &
-- résultat obligatoire d'un procédé suffisamment décrit & 
T12/81\\

pas divulgation &
-- isoler le principe actif responsable de l'activité thérapeutique
connue d'une plante &
T1250/18 \\

&
-- divulgation insuffisante pour que la PdM produise l'invention &
\\

&
\textbf{Combinaison de documents?} &
\\

\ok{combinable} &
-- intépréter un art antérieur opposable &
CA Paris 27/11/84 \\

&
-- référence explicite à un autre document &
T153/85 \\

\nok{pas combinable} &
-- combiner des modes de réalisation distincts d'un même document &
T305/87 \\

&
-- combiner un perfectionnement spécifique et une technique antérieure
générale sans référence particulière &
T(?)\\

&
\textbf{invention de séletion?} &
lire les directives\\

\ok{nouveau} &
-- plage de valeur sélectionnée étroite et éloignée &
T261/15\\

&
-- sélection non arbitraire? &
T198/84, T279/89 \\

&
-- critères critiqués : pas norme de référence &
T1688/20, T667/23\\

&
\textbf{sélection dans plusieurs listes?} &
\\

\nok{pas nouveau} &
-- une sélection dans une seule liste &
T181/82 \\

\ok{nouveau} &
-- sélection à partir de deux listes ou plus d'une certaine
longeur &
T2350/16 \\

&
\textbf{formule de Markush?} &
\\

\ok{nouveau} &
-- (?) &
\\

&
\textbf{disclaimer?} &
\\

\ok{autorisé} &
-- clair et concis &
G1/03, T4/80\\

&
-- la portée des revendications serait limitée indument
par des caractéristiques positives &
T1050/93 \\

disclaimer non divulgué &
-- pas A123(2) mais (i) rétablir la nouveauté par rapport à
un état de la technique A54(3) (ii) rétablir la nouveauté par
rapport à une divulgation fortuite A54(2) (iii) exclure une 
excetion à la brevetabilité &
G1/03, G2/03, G1/16 \\

&
-- ne doit pas retrancer plus que nécessaire pour rétablir la
nouveauté &
\\

&
-- pas devenir pertinent pour l'appréciation de l'activité
inventive ou la suffisance de l'exposé: améliore la position
de l'invention par rapport à d'autres documents &
\\

&
-- rester clair et concis &
\\

&
$\Rightarrow$ n'affecte pas le droit de priorité &
\\

disclaimer divulgué &
-- pas A123(2) la PdM saurait déduire explicitement ou implicitement,
l'objet restant dans la revendication &
G2/10\\

&
\textbf{degré de pureté? }&
\\

anciennement &
-- un document divulgant un composé chimique rend ce composé
accessible au public dans tous les degrés de purté souhaités &
T990/96, T360/07, TGI Paris 20/02/2009, CA Paris (?)\\

maintenant &
-- si le document de l'art antérieur ne divulgue pas un degré
de pureté, quel serait le degré de pureté obtenu par la PdM? &
T1085/13, T43/18\\

\hline \textbf{activité inventive} &
&
L611-14, A56\\

&
\textbf{personne du métier?} &
\\

&
-- doit obligatoirement définie &
Cass. 20/11/2012 \\

&
\textbf{approche problème-solution?} &
\\

&
-- l'approche problème-solution ne s'impose pas aux
juridications françaises &
CA Paris 14/11/2017 \\

ETTP &
-- utilisation semblable et qui appelle le moins de modifications
structurelles et fonctionnelles &
T606/89 \\

&
-- un point de départ réaliste qui vise le même but ou le même
effet technique &
T2759/17 \\

PTO &
-- examiner si ce problème a bien été résolu par la solution revendiquée
sinon reformuler un problème technique moins ambitieux
(souvent une alternative) &
T479/06 \\

&
-- moyens de preuve soumis après dépôt ne peuvent être écartés
(i) effet englobé dans l'enseignement technique
(ii) effet faisant partie de la même invention initialement divulguée &
G2/21\\

&
\textbf{indices d'activité inventive?} &
\\

\ok{indices favorables} &
-- préjugé technique vaincu &
TGI Lyon 13/11/1997\\

&
-- effet de synergie &
\\

&
-- résultat inattendu &
\\

\nok{indices défavorables} &
-- choix arbitraire &
\\

&
-- chance raisonnable de succès &
\\

&
-- évident d'essayer &
\\

&
-- situation à sens unique, effet bonus &
TGI Paris 12/02/2008, CA Paris 17/10/1980\\

&
\textbf{brevetabilité des énantiomères?} &
\\

&
-- dépendant des techniques de séparations au moment du dépôt &
TGI Paris 30/09/2010 \\

\textbf{suffisance de description} &
&
L612-5, A83\\

date pertinente &
-- date de dépôt &
CA Paris 19/10/2005\\

caractère spéculatif &
-- pas d'expériences venant démontrer les effets allégués pour 
l'énantiomère (-)&
TGI Paris 06/10/2009\\

&
-- résultat a été recherché et existe pour une deuxième
utilisation médicale & 
CA Paris 30/01/2015 \\

modes de réalisation &
-- (i) si l'effet pas obtenu est décrit dans une revendication,
l'exposé n'est pas suffisant
(ii) si l'effet n'est pas décrit dans une revendication, mais fait
partie du problème à résoudre, il se pose un problème d'activité inventive &
G1/03 \\

\textbf{extension} &
&
\\

A123(2) &
-- la demande de brevet européen ou le brevet européen ne peut
être modifié de manière que son objet s'étende au-delà du contenu
de la demande telle qu'elle a été déposée &
\\

A76(1) &
-- toute demande divisionnaire ... ne peut être déposée que pour
des éléments qui ne s'étedent pas au-delà du contenu de la demande
antérieure telle qu'elle a été déposée &
\\

A123(3) &
-- le brevet européen ne peut être modifié de façon à étendre la
protection qu'il confère &
\\

\ok{supportée} &
-- combiner deux domaines partiels &
T2/81, T1170/02, T612/09\\

&
-- extraires de caractéristiques qui ne sont pas étroitement
liées &
TGI Paris 28/05/2015, T201/83\\

\nok{non supportée} &
-- généralisation intermédiaire &
\\

\end{longtblr}