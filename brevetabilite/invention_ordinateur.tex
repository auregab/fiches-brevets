\section{Conditions de brevetabilité de programme d'ordinateur}

\begin{longtblr}{colspec = {X[3,l] X[7,l] X[5,l]}}

\hline
\textbf{définitions} \newline logiciel
 & -- ensemble des programes, procédés et règles
 et éventuellement de la documentation, relatifs au 
 fonctionnement d'un ensemble de traitement de données
 & Arrêté du 22 décembre 1981 \\

programme d'ordinateur
 & -- un programme d'ordinateur est un ensemble d'instructions
 dont le but est d'exécuter par un dispositif de traitement
 de l'information, tel qu'un ordinateur, ses fonctionnalités
 & Livre vert européen, juin 1988 \\

droit d'auteur
 & -- code source, interface graphique
 & A1(1) Directives 2009/24/CE \\

droit brevets
 & -- idées et principes
 & A1(2) Directives 2009/24/CE \\

\SetHline{2-3}{}

\textbf{typologies} & & \\

\ok{condition brevetabilité classique}
 & -- matériel (hardware)
 & \\

\nok{conditions de brevetabilité spécifique}
 & -- logiciel (software) invention "invisible":
 détection contrefaçon difficile, par exemple
 algorithme, IA, calcul, compression de données
 & \\

 & -- accessibilité pour la compréhension:
 difficile mais chances de complétude sont
 normales pour la rédaction
 & \\

\nok{conditions de brevetabilité spécifique}
 & -- logiciel (software) invention "visible"
 détection contrefaçon facile, par exemple
 interface graphique
 & \\
 
 & -- accessibilité pour la compréhension:
 facile mais chances de complétude sont faibles
 pour la rédaction
 & \\

exemple
 & -- déplacer un objet d'un point A à un point B
 sur un écran au moyen d'une souris: 
 première interaction 2 clics, 1 déplacements;
 deuxième interaction 1 clic, 1 déplacement, ou encore,
 existence d'une zone exclusive d'interaction avec
 la souris
 & \\

\SetHline{2-3}{}
\textbf{anticiper}
obsolescence
 & -- la rédaction doit anticiper le risque d'obsolescence
 dans le domaine informatique
 & \\

risque terminologique
 & -- bande passante: plage de fréquence ou débit d'information?
 & \\

\end{longtblr}