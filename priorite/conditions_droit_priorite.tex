%%%%%%%%%%%%%%%%%%%%%%%%%%%%%%%%%%%%%%%%%%%%%%%%%%%%%
\section{Conditions d'une priorité}
\label{sec:verifier-priorite}
%%%%%%%%%%%%%%%%%%%%%%%%%%%%%%%%%%%%%%%%%%%%%%%%%%%%%

\begin{longtblr}{colspec = {X[3,l] X[8,l] X[3,l]}}

\textbf{Prio. valable =} &
& 
\epcart{87}(1) \\ 
\hline

\textbf{Celui qui \underline{et}\mydots} & \textbf{Preuve ?} & \\

\ok{Valable à l'OEB} &
-- présomption réfragable du droit de priorité &
\epocase{g220001ex1}{G1/22}, \epocase{g220002ep1}{G2/22} \\

\ok{Valable à l'INPI} &
-- « copie de la demande antérieure » &
\\
& -- « autorisation de revendiquer la priorité donnée par écrit par le propriétaire de la demande antérieure » &
\href{https://www.legifrance.gouv.fr}{Cass. arrêt n° 537 FS-B} \\

\hline

\textbf{Rég. dép. \underline{et}\mydots} & \textbf{État partie à l’OMC ?} & \\

\nok{Non valable à l'OEB} &
-- \delai{avant 13/12/2007} &
\epocase{g020002ep1}{G2/02}, \epocase{g030002ep1}{G3/02} \\

\ok{Valable à l'OEB} &
-- \delai{après 13/12/2007 (inclus)} &
CBE 2000 \\

& \textbf{État accédant à CUP ou OMC ?} & \\

\ok{Valable à l'OEB} &
-- \delai{accord en vigueur à la date de la priorité} &
\\
& -- \delai{législation applicable à la date de dépôt} &
\\

\hline

\textbf{M. inv. \underline{et}\mydots} & \textbf{Fondement ?} & \\

\ok{Valable à l'OEB} &
-- « suffisamment décrit » dans la demande antérieure &
\epocase{g150001ex1}{G1/15} \\
& (i) allouer des dates de priorité dans une même revendication & \\
& (ii) DDSA des pièces de la demande antérieure &
\epocase{g980002ep1}{G2/98} \\

\nok{Non valable à l'OEB} &
-- combiner des documents non liés (mosaïque) &
\epogl{f_vi_1_5}{Dir. F-VI, 1.5} \\
& -- état de la technique cité dans la description &
\epogl{f_vi_2_2}{Dir. F-VI, 2.2} \\
& -- disclaimer décrit dans la demande antérieure & \\

\hline

\textbf{\delai{12 mois} \underline{et}\mydots} & \textbf{Calcul ?} & \\

\ok{Valable à l'OEB} &
(i) appliquer la règle 131(4) &
\epcrule{131} \\

& (ii) appliquer règles 134(1)–(4) &
\epcrule{134} \\

& (iii) appliquer règle 133 + règle 134(5) &
\epcrule{133} \\

\end{longtblr}