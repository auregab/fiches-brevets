\section{Méthodologie du cas pratique}

\begin{longtblr}{colspec = {X[3,l] X[7,l] X[5,l]}, rowsep = 1pt}

\hline
\textbf{identifier les faits pertinents} & -- lire tout le sujet
& \\

& -- surligner les faits pertinents
& \\

& -- La société X, spécialisée dans l’électroménager, est \textbf{propriétaire d’un brevet FR} dont l’inventeur est l’un de ses salariés. Le brevet porte sur \textbf{la lame spécifique d’un nouveau robot} qu’elle commercialise dans toute l’Europe.
& \\

& -- Elle souhaite assigner en \textbf{contrefaçon} la société Y qui vend, en France, des robots \textbf{intégrant une lame, fabriqués en Chine}, qui, selon elle, \textbf{portent atteinte à son brevet}.
& \\

& -- Question : pensez-vous que la société X puisse faire interdire à la société Y la vente des robots litigieux, sur le territoire français ?
& \\

\SetHline{2-3}{}

\SetCell[r=3]{valign=h}\textbf{qualifier les faits pertinents} & -- qui: la société X est prioritaire d'un brevet FR
& \\

& -- titre: le brevet porte sur un produit
& \\

& -- actes de cf.: Y offre, met dans le commerce en FR, importe des lames
destinées à des robots, objet du brevet
& \\

\SetHline{2-3}{}

\SetCell[r=3]{}\textbf{construction du raisonnement juridique} \newline majeure 
 & -- énonciation de la règle de droit applicable:
"selon les dispositions de ...", "en vertu de...", "il résulte de l'article...".
 & \\
 
sélection de textes & -- L615-2, L613-3(a), L615-1, L615-3, L615-7
& \\

exposer règle et interprétation & -- les conditions de l'interdiction provisoire: 
la vraisemblance de la validité du brevet
& \\

mineure & -- applique la règle de droit aux faits de l'espèce:
"or, ...", "en l'espèce..."
 & \\

conclusion & -- énonce la solution du problème: "donc", "il en résulte"
 & \\


\end{longtblr}