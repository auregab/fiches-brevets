% Options for packages loaded elsewhere
\PassOptionsToPackage{unicode}{hyperref}
\PassOptionsToPackage{hyphens}{url}
%
\documentclass[
]{article}
\usepackage{amsmath,amssymb}
\usepackage{iftex}
\ifPDFTeX
  \usepackage[T1]{fontenc}
  \usepackage[utf8]{inputenc}
  \usepackage{textcomp} % provide euro and other symbols
\else % if luatex or xetex
  \usepackage{unicode-math} % this also loads fontspec
  \defaultfontfeatures{Scale=MatchLowercase}
  \defaultfontfeatures[\rmfamily]{Ligatures=TeX,Scale=1}
\fi
\usepackage{lmodern}
\ifPDFTeX\else
  % xetex/luatex font selection
\fi
% Use upquote if available, for straight quotes in verbatim environments
\IfFileExists{upquote.sty}{\usepackage{upquote}}{}
\IfFileExists{microtype.sty}{% use microtype if available
  \usepackage[]{microtype}
  \UseMicrotypeSet[protrusion]{basicmath} % disable protrusion for tt fonts
}{}
\makeatletter
\@ifundefined{KOMAClassName}{% if non-KOMA class
  \IfFileExists{parskip.sty}{%
    \usepackage{parskip}
  }{% else
    \setlength{\parindent}{0pt}
    \setlength{\parskip}{6pt plus 2pt minus 1pt}}
}{% if KOMA class
  \KOMAoptions{parskip=half}}
\makeatother
\usepackage{xcolor}
\usepackage{longtable,booktabs,array}
\usepackage{calc} % for calculating minipage widths
% Correct order of tables after \paragraph or \subparagraph
\usepackage{etoolbox}
\makeatletter
\patchcmd\longtable{\par}{\if@noskipsec\mbox{}\fi\par}{}{}
\makeatother
% Allow footnotes in longtable head/foot
\IfFileExists{footnotehyper.sty}{\usepackage{footnotehyper}}{\usepackage{footnote}}
\makesavenoteenv{longtable}
\ifLuaTeX
  \usepackage{luacolor}
  \usepackage[soul]{lua-ul}
\else
  \usepackage{soul}
\fi
\setlength{\emergencystretch}{3em} % prevent overfull lines
\providecommand{\tightlist}{%
  \setlength{\itemsep}{0pt}\setlength{\parskip}{0pt}}
\setcounter{secnumdepth}{-\maxdimen} % remove section numbering
\ifLuaTeX
  \usepackage{selnolig}  % disable illegal ligatures
\fi
\IfFileExists{bookmark.sty}{\usepackage{bookmark}}{\usepackage{hyperref}}
\IfFileExists{xurl.sty}{\usepackage{xurl}}{} % add URL line breaks if available
\urlstyle{same}
\hypersetup{
  hidelinks,
  pdfcreator={LaTeX via pandoc}}

\author{}
\date{}

\begin{document}

\textbf{DROIT DE LA RESPONSABILITE EXTRACONTRACTUELLE}

\begin{longtable}[]{@{}
  >{\raggedright\arraybackslash}p{(\columnwidth - 0\tabcolsep) * \real{1.0000}}@{}}
\toprule\noalign{}
\begin{minipage}[b]{\linewidth}\raggedright
\textbf{\hl{Une obligation}} définit un rapport juridique qui unit une
première personne appelée débiteur, à une deuxième personne appelée
créancier. Le lien de l'obligation présente~:\\
(i) \hl{un caractère personnel}~; et,\\
(ii) \hl{un caractère contraignant.}\strut
\end{minipage} \\
\midrule\noalign{}
\endhead
\bottomrule\noalign{}
\endlastfoot
\end{longtable}

\textbf{CONSENTEMENT}

\emph{En résumé.} Consentement = offre, acceptation et rencontre.

\begin{longtable}[]{@{}
  >{\raggedright\arraybackslash}p{(\columnwidth - 0\tabcolsep) * \real{1.0000}}@{}}
\toprule\noalign{}
\begin{minipage}[b]{\linewidth}\raggedright
\textbf{\hl{Article 1114 du Code Civil}
(offre)~}:~«~L\textquotesingle offre, faite à personne déterminée ou
indéterminée, comprend \hl{les éléments essentiels du contrat envisagé
{[}(i) caractère précis{]}} et exprime \hl{la volonté de son auteur
d\textquotesingle être lié en cas d\textquotesingle acceptation {[}(ii)
caractère ferme{]}}. A défaut, il y a seulement invitation à entrer en
négociation.~»
\end{minipage} \\
\midrule\noalign{}
\endhead
\bottomrule\noalign{}
\endlastfoot
\end{longtable}

\begin{longtable}[]{@{}
  >{\raggedright\arraybackslash}p{(\columnwidth - 0\tabcolsep) * \real{1.0000}}@{}}
\toprule\noalign{}
\begin{minipage}[b]{\linewidth}\raggedright
\textbf{\hl{Article 1118 du Code Civil}
(acceptation)~}:~«~L\textquotesingle acceptation est la manifestation de
volonté de son auteur d\textquotesingle être lié dans les termes de
l\textquotesingle offre.\\
Tant que l\textquotesingle acceptation n\textquotesingle est pas
parvenue à l\textquotesingle offrant, elle peut être librement
rétractée, pourvu que \hl{la rétractation parvienne à
l\textquotesingle offrant avant l\textquotesingle acceptation {[}(i)
caractère délai (+ fixé ou raisonnable){]}.}\\
L\textquotesingle acceptation non conforme à l\textquotesingle offre est
dépourvue d\textquotesingle effet, sauf à constituer \hl{une offre
nouvelle {[}(ii) contre-offre =/= caractère pure et simple de
l'offre{]}}.~»\strut
\end{minipage} \\
\midrule\noalign{}
\endhead
\bottomrule\noalign{}
\endlastfoot
\end{longtable}

\begin{longtable}[]{@{}
  >{\raggedright\arraybackslash}p{(\columnwidth - 0\tabcolsep) * \real{1.0000}}@{}}
\toprule\noalign{}
\begin{minipage}[b]{\linewidth}\raggedright
\textbf{\hl{Article 1130 du Code Civil} (vices du
consentement)~:}~«~\hl{L\textquotesingle erreur, le dol et la violence}
vicient le consentement lorsqu\textquotesingle ils sont de telle nature
que, sans eux, l\textquotesingle une des parties
n\textquotesingle aurait pas contracté ou aurait contracté à des
conditions substantiellement différentes.\\
Leur caractère déterminant s\textquotesingle apprécie eu égard aux
personnes et aux circonstances dans lesquelles le consentement a été
donné.~»\strut
\end{minipage} \\
\midrule\noalign{}
\endhead
\bottomrule\noalign{}
\endlastfoot
\end{longtable}

\begin{longtable}[]{@{}
  >{\raggedright\arraybackslash}p{(\columnwidth - 0\tabcolsep) * \real{1.0000}}@{}}
\toprule\noalign{}
\begin{minipage}[b]{\linewidth}\raggedright
\textbf{\hl{Article 1132 du Code Civil}
(l'erreur)~:}~«~L\textquotesingle erreur de droit ou de fait, à moins
qu\textquotesingle elle ne soit \hl{inexcusable {[}(i)~: celui qui est
victime de l'erreur avait pas les moyens raisonnables de la
dissiper{]}}, est une cause de nullité du contrat
lorsqu\textquotesingle elle porte sur les \hl{qualités essentielles
(ii)} de la prestation due ou sur celles du cocontractant.~»
\end{minipage} \\
\midrule\noalign{}
\endhead
\bottomrule\noalign{}
\endlastfoot
\end{longtable}

\begin{longtable}[]{@{}
  >{\raggedright\arraybackslash}p{(\columnwidth - 0\tabcolsep) * \real{1.0000}}@{}}
\toprule\noalign{}
\begin{minipage}[b]{\linewidth}\raggedright
\textbf{\hl{Article 1137 du Code Civil} (dol)~}:~«~Le dol est le fait
pour un contractant d\textquotesingle obtenir le consentement de
l\textquotesingle autre par des \hl{manœuvres {[}1\textsuperscript{er}
élément matériel{]}} //escroquerie, base~: \hl{Article 313-1 du Code
Pénal}{]} ou des \hl{mensonges {[}2\textsuperscript{e} élément
matériel{]}}.

Constitue également un dol la \hl{dissimulation intentionnelle
{[}3\textsuperscript{e} élément matériel~: réticence dolosive
//obligation d'information précontractuelle{]}} par l\textquotesingle un
des contractants d\textquotesingle une information dont il sait le
caractère déterminant pour l\textquotesingle autre partie.

Néanmoins, ne constitue pas un dol le fait pour une partie de ne pas
révéler à son cocontractant son estimation de la valeur de la
prestation.~»\\
\hl{distinguer~: dol principal et dol incident.}

\textbf{\hl{Article 1140 du Code Civil} (violence)~:~}«~Il y a violence
lorsqu\textquotesingle une partie s\textquotesingle engage sous la
pression d\textquotesingle une \hl{contrainte} qui lui inspire la
crainte d\textquotesingle exposer sa personne, sa fortune ou celles de
ses proches à un mal considérable.~»\\
- contrainte physique~: menaces, séquestration~;\\
- contrainte morale~: chantage, pression, endoctrinement~;\\
- contrainte économique~: autorité économique, abus état de
dépendance.\strut
\end{minipage} \\
\midrule\noalign{}
\endhead
\bottomrule\noalign{}
\endlastfoot
\end{longtable}

\textbf{SANCTIONS DE L'INEXECUTION DU CONTRAT}

\begin{longtable}[]{@{}
  >{\raggedright\arraybackslash}p{(\columnwidth - 0\tabcolsep) * \real{1.0000}}@{}}
\toprule\noalign{}
\begin{minipage}[b]{\linewidth}\raggedright
\textbf{\hl{Article 1217 du Code Civil} (sanctions à
l'inexécution)~}:~«~La partie envers laquelle
l\textquotesingle engagement n\textquotesingle a pas été exécuté, ou
l\textquotesingle a été imparfaitement, peut :

- \hl{{[}(i) l'exception d'inexécution{]}} refuser
d\textquotesingle exécuter ou suspendre l\textquotesingle exécution de
sa propre obligation ;

- \hl{{[}(ii) l'exécution forcée{]}} poursuivre
l\textquotesingle exécution forcée en nature de
l\textquotesingle obligation ;

- \hl{{[}(iii) réduction du prix{]}} obtenir une réduction du prix ;

- \hl{{[}(iv) résolution du contrat{]}} provoquer la résolution du
contrat ;

- \hl{{[}(v) responsabilité contractuelle{]}} demander réparation des
conséquences de l\textquotesingle inexécution.

Les sanctions qui ne sont pas incompatibles peuvent être cumulées ; des
dommages et intérêts peuvent toujours s\textquotesingle y ajouter
\hl{{[}les sanctions pas incompatibles peuvent être cumulées{]}.}~»
\end{minipage} \\
\midrule\noalign{}
\endhead
\bottomrule\noalign{}
\endlastfoot
\end{longtable}

\emph{En résumé.} \hl{Responsabilité contractuelle~:}\\
- \hl{(i) première condition~: inexécution totale} ou partielle
imputable au débiteur, en cas d'obligation de résultat, la charge de la
preuve est au débiteur, en cas d'obligation de moyens, la charge de la
preuve est au créancier~;\\
- \hl{(ii) deuxième condition~: le dommage} doit être \hl{prévisible}
(au moment de la conclusion du contrat) et \hl{certain}~;\\
\hl{- (iii) troisième condition~: lien de causalité} entre le fait
générateur de l'inexécution et le dommage.

\textbf{DROIT DE LA RESPONSABILITE EXTRACONTRACTUELLE}

\begin{longtable}[]{@{}
  >{\raggedright\arraybackslash}p{(\columnwidth - 0\tabcolsep) * \real{1.0000}}@{}}
\toprule\noalign{}
\begin{minipage}[b]{\linewidth}\raggedright
\textbf{\hl{Article 1240 du Code civil} (responsabilité
extracontractuelle)~}:~«~Tout fait quelconque de
l\textquotesingle homme, qui \hl{cause {[}(i) première caractéristique~:
théorie de la causalité adéquate, théorie de l'équivalence des
conditions, possibilité d'exonérations{]}} à autrui \hl{un dommage
{[}(ii) deuxième caractéristique~: certain, légitime, direct{]}}, oblige
celui par \hl{la faute {[}(iii) troisième caractéristique~: un
comportement et illicéité{]}} duquel il est arrivé à le réparer.~»
\end{minipage} \\
\midrule\noalign{}
\endhead
\bottomrule\noalign{}
\endlastfoot
\end{longtable}

\emph{En résumé.} Réparation~:\\
- évaluation du préjudice par la victime~;\\
- réparation du préjudice uniquement, ne permet pas de punir son
auteur~;\\
- exclusion des actions de groupe sauf droit de la consommation par une
association de représentation de consommateurs et action collective pour
produits de santé.

\end{document}
