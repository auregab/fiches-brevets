% Options for packages loaded elsewhere
\PassOptionsToPackage{unicode}{hyperref}
\PassOptionsToPackage{hyphens}{url}
%
\documentclass[
]{article}
\usepackage{amsmath,amssymb}
\usepackage{iftex}
\ifPDFTeX
  \usepackage[T1]{fontenc}
  \usepackage[utf8]{inputenc}
  \usepackage{textcomp} % provide euro and other symbols
\else % if luatex or xetex
  \usepackage{unicode-math} % this also loads fontspec
  \defaultfontfeatures{Scale=MatchLowercase}
  \defaultfontfeatures[\rmfamily]{Ligatures=TeX,Scale=1}
\fi
\usepackage{lmodern}
\ifPDFTeX\else
  % xetex/luatex font selection
\fi
% Use upquote if available, for straight quotes in verbatim environments
\IfFileExists{upquote.sty}{\usepackage{upquote}}{}
\IfFileExists{microtype.sty}{% use microtype if available
  \usepackage[]{microtype}
  \UseMicrotypeSet[protrusion]{basicmath} % disable protrusion for tt fonts
}{}
\makeatletter
\@ifundefined{KOMAClassName}{% if non-KOMA class
  \IfFileExists{parskip.sty}{%
    \usepackage{parskip}
  }{% else
    \setlength{\parindent}{0pt}
    \setlength{\parskip}{6pt plus 2pt minus 1pt}}
}{% if KOMA class
  \KOMAoptions{parskip=half}}
\makeatother
\usepackage{xcolor}
\usepackage{graphicx}
\makeatletter
\def\maxwidth{\ifdim\Gin@nat@width>\linewidth\linewidth\else\Gin@nat@width\fi}
\def\maxheight{\ifdim\Gin@nat@height>\textheight\textheight\else\Gin@nat@height\fi}
\makeatother
% Scale images if necessary, so that they will not overflow the page
% margins by default, and it is still possible to overwrite the defaults
% using explicit options in \includegraphics[width, height, ...]{}
\setkeys{Gin}{width=\maxwidth,height=\maxheight,keepaspectratio}
% Set default figure placement to htbp
\makeatletter
\def\fps@figure{htbp}
\makeatother
\ifLuaTeX
  \usepackage{luacolor}
  \usepackage[soul]{lua-ul}
\else
  \usepackage{soul}
\fi
\setlength{\emergencystretch}{3em} % prevent overfull lines
\providecommand{\tightlist}{%
  \setlength{\itemsep}{0pt}\setlength{\parskip}{0pt}}
\setcounter{secnumdepth}{-\maxdimen} % remove section numbering
\ifLuaTeX
  \usepackage{selnolig}  % disable illegal ligatures
\fi
\IfFileExists{bookmark.sty}{\usepackage{bookmark}}{\usepackage{hyperref}}
\IfFileExists{xurl.sty}{\usepackage{xurl}}{} % add URL line breaks if available
\urlstyle{same}
\hypersetup{
  hidelinks,
  pdfcreator={LaTeX via pandoc}}

\author{}
\date{}

\begin{document}

\hl{Droit (111, 112; 121, 122; 131, 132)~}: ensemble \hl{de règles
(210)} régissant les relations sociales, issues des \hl{sources du droit
(310, 320)}.

Distinguer~:\\
\hl{Droit objectif (111)}~: ensemble règles applicables à tous~;\\
\hl{Droit subjectif (112)}~: prérogatives dévolues à une personne
particulière~;\\
\hl{Droit romano-germanique (121)}~: fondé sur l'écrit~;\\
\hl{Common law (122)}~: fondé sur la jurisprudence~;\\
\hl{Droit positif (131)}~: se réfère exclusivement aux textes~;\\
\hl{Droit naturel (132)}~: des principes supérieurs au droit
guident/corriger le droit écrit.

Caractéristiser une \hl{règle de droit (210)~}: une règle est
\hl{obligatoire (211) et générale (212),} une règle de droit est
\hl{coercitive par l'autorité publique (213)} et présente une
\hl{finalité mesurée (214).}

\includegraphics[width=2.25347in,height=1.56667in]{media/image1.png}Distinguer~:\\
\hl{Sources formelles (310)~}: constitution (311), traités
internationaux (312), lois (313), ordonnances (314), règlements (315),
décrets (316), arrêtés (317)~;\\
\hl{Sources informelles (320)~}: jurisprudence, coutumes (ex. usages
commerciaux), doctrine, soft law (ex. recommandations)

\textbf{CONSTITUTION}

\hl{Bloc de constitutionnalité (311)}~ comprend~:\\
(i) la constitution de la Ve République de 1958~;\\
(ii) décision 1971 «~liberté d'association~» inclut les préambules des
constitutions de 1958 et 1946~;\\
(iii) décision 1973 inclut la DDHC de 1789~;\\
(iv) la charte de l'environnement de 2004.

\hl{\textbf{Article 61 de la Constitution} contrôle de
constitutionnalité \emph{a priori}}\emph{~}: vérification par le Conseil
Constitutionnel saisi par Président de la République, le Premier
ministre, le Président de l'une des deux assemblées, 60 députés ou 60
sénateurs\\
Intervient avant la promulgation de la loi\\
Les dispositions anticonstitutionnel sont censurés, seules les parties
qui sont conformes à la constitution sont promulguées.

\hl{\textbf{Article 61-1 de la Constitution} contrôle de
constitutionnalité \emph{a posteriori~}:} \hl{Question Prioritaire de
Constitutionalité} sous conditions\\
(i) la disposition contestée est applicable au litige en cours\\
(ii) la disposition contestée ne peut avoir été déclarée conforme à la
Constitution au préalable\\
(iii) la QPC présente un caractère sérieux\\
premier filtre : le juge d'instance puis deuxième filtre : le conseil
d'Etat ou la Cour de cassation puis troisième filtre : transmis au
Conseil constitutionnel.

\textbf{\hl{Loi Veil du 17 janvier 1975}} sur l'interruption volontaire
de grossesse~: \hl{le Conseil constitutionnel} s'est déclaré incompétent
pour vérifier la conformité des lois aux traités.

\includegraphics[width=1.26319in,height=1.43958in]{media/image2.png}\includegraphics[width=3.31181in,height=1.67986in]{media/image3.png}\textbf{\hl{Arrêt
de la chambre mixte \emph{Jacques Vabre} du 24 mai 1975}}~\hl{: la Cour
de cassation} vérifie les lois sont conformes aux traités internationaux
pour les litiges d'ordre judiciaire = affaires entre parties privées.

\textbf{\hl{Arrêt \emph{Nicolo} du 20 octobre 1989}}~: \hl{le conseil
d'Etat} vérifie si les lois sont conformes aux traités internationaux
pour les litiges d'ordre administratif.

\textbf{TRAITES}

Distinguer~:\\
\hl{traités-lois}~: obligations réciproques\\
\hl{traités-contrats}~: harmoniser un droit commun entre des Etats
signataires

\textbf{\hl{Article 53 et 53 de la Constitution}~}: \hl{modifier la
constitution} avant de ratifier le traité lorsque le traité est non
conforme à la Constitution.

\textbf{\hl{Article 55 de la Constitution}~:} les traités sont
supérieurs aux lois.

\includegraphics[width=4.46038in,height=1.84906in]{media/image4.png}

Distinguer~:\\
application directe du traité~: directives européennes par ex., pas
valable en litige~;\\
application indirecte du traité~: règlement européen par ex., valable en
litige.

\textbf{\hl{Article 234 du Traité instituant la Communauté européenne}}
\hl{Question Préjudicielle} lorsqu'un juge national demande une
précision sur l'application/interprétation d'un règlement européen à
condition que\\
(i) la réponse est nécessaire pour que le juge rende son jugement ET\\
(ii) aucune recours juridictionnel n'est disponible dans le droit
national OU\\
(iii) le juge national s'est déclaré incompétent.

\textbf{LOIS}

\textbf{\hl{Article 34 de la Constitution}~:~} les domaines seulement
légiférées par \hl{le Parlement}~: les libertés publiques, l'état et la
capacité des personnes, la procédure pénale, la détermination des
crimes, des délits ainsi que leurs sanctions, les impôts, les régimes
électoraux.

\textbf{\hl{Article 37 de la Constitution~}}: les domaine de compétence
du règlement légiféré par \hl{le Gouvernement}~: «~les matières autres
que celles qui sont du domaines de la loi~» = compétence résiduelle ou
de droit commun.

\textbf{\hl{Article 38 de la Constitution}}~: exceptionnellement, le
Gouvernement peut prendre par \hl{ordonnance} des mesures relevant du
domaine de législation du Parlement à condition que le Parlement vote
une délégation de pouvoirs pour (i) un temps limité et (ii) sur un sujet
limité.

Les ordonnances doivent être \hl{ratifiées} par le Parlement~:\\
- avant leur ratification, obtient qualité d'actes réglementaires\\
- après leur ratification, obtient qualité de loi.

\hl{\textbf{Article 1} \textbf{du Code Civil}} dispose que «~les lois
sont exécutoires dans tout le territoire français, en vertu de la
\hl{promulgation} qui en est faite par le Président de la République
{[}\ldots{]}, les lois entrent en vigueur\\
(i) à la date qu'ils fixent, OU\\
à défaut, (ii) le lendemain de leur publication» OU\\
(iii) «~si la loi nécessite l'adoption de mesures d'application,
l'entrée en vigueur est reportée à la date d'entrée en vigueur de ces
mesures~» OU (iv) à la date que le décret d'application précise.~»

\textbf{Présomption irréfragable} «~Nul n'est censé ignorer la loi~»

\hl{\textbf{Article 2} \textbf{du Code Civil}} «~la loi ne dispose que\\
(i) pour l'avenir \hl{{[}principe d'application immédiate{]}}~;\\
(ii) elle n'a point d'effet rétroactif \hl{{[}principe de non
rétroactivité{]}}\textbf{~}».

Distinguer~:\\
\hl{Fait juridique}~: produisent des effets de droit indifféremment de
la volonté des personnes, \textbf{le jugement s'applique à partir de la
loi applicable au moment où le fait juridique est produit}~;\\
\hl{Actes juridiques~}:conclus dans le but de produire des effets de
droit

\includegraphics[width=4.61667in,height=2.13559in]{media/image5.png}

\emph{Exceptions au principe de rétroactivité~}:\\
(i) la loi est expressément rétroactive~;\\
(ii) lois interprétatives~;\\
(iii) lois pénales plus douces~:\\
(iv) compétence et procédure.

\emph{Exceptions au principe d'application immédiate~}:\\
(i) dispositions transitoires\\
(ii) survie des contrats en cours (sauf pénal)~: \textbf{le jugement
d'un contrat s'applique à partir de la loi applicable au moment où le
contrat est signé.}

\textbf{\hl{Article 6 du Code Civil}}~: \textbf{règles impératives}
«~une règle d'ordre public auquel la volonté individuelle ne peut
déroger.~» =/= \textbf{règles supplétives}

\textbf{\hl{Abrogation}\\
}(i) expresse par une nouvelle loi de même niveau ou de niveau supérieur
à la l'ancienne loi\\
(ii) tacite lorsque les règles de la nouvelle loi sont incompatibles
avec les règles de l'ancienne loi, PAS par désuétude.

\textbf{COUTUMES}

La coutume \hl{secundum legem}, la coutume qui suit la loi~:\\
(i) la loi renvoie expressément à la coutume OU\\
(ii) la loi renvoie implicitement à la coutume.

La coutume \hl{contra legem~}: La coutume n'a pas le pouvoir de déroger
à la loi.

La coutume \hl{praeter legem~}: La coutume s'applique à défaut de loi.

\textbf{JURISPRUDENCE}

i) l'ensemble des décisions judiciaires prononcés par les juridictions\\
ii) tendance dans la manière d'appliquer les règles de droit par les
tribunaux.

\textbf{\hl{Article 4 du Code Civil}}~: «~Le juge qui refusera de juger,
sous prétexte du silence, de l\textquotesingle obscurité ou de
l\textquotesingle insuffisance de la loi, pourra être poursuivi comme
coupable de déni de justice.~»

\textbf{\hl{Article 5 du Code Civil}}~:~«~Il est défendu aux juges de
prononcer par voie de disposition générale et réglementaire sur les
causes qui leur sont soumises.~»

\textbf{\hl{Article 12 du code de procédure civil}} «~Le juge tranche le
litige conformément aux règles de droit qui lui sont applicables~».

\textbf{DROIT PRIVE}

\emph{summa divisio}, distinguer~:\\
\hl{droit privé}~: régit les rapports entre les personnes entre elles.\\
\hl{droit public~}: régit les rapports entre les personnes et des
personnes publiques (Etats, régions, départements, communes).

\includegraphics[width=4.61667in,height=2.59891in]{media/image6.png}

\textbf{DROIT PUBLIC}

\includegraphics[width=4.61667in,height=2.66154in]{media/image7.png}

\textbf{DROIT SUBJECTIF}

Distinguer~:\\
(i) \hl{la personnalité juridique}~: affectée aux sujets de droit~;\\
(ii) \hl{la capacité juridique}~: affectée à un majeur capable ou sous
tutelle ou sous curatelle.

Distinguer~:\\
(i) \hl{personne physique}~: existe de la naissance, \emph{pars viscerum
matris} (exp. Thérie de l'infans conceptus) jusqu'à la mort prononcé par
médecin (base~: \textbf{\hl{Article 718 du Code Civil}}) (exp. La
disparition, base~: \textbf{\hl{Article 88 du Code Civil}} ou l'absence,
base~: \textbf{\hl{Article 112 du Code Civil}})~;\\
(ii) \hl{les personnes morales}~: fiction juridique.

Distinguer~:\\
(i) \hl{les droits réels}~: les droits sur les choses (spé. droit réel
accessoire)\\
(ii) \hl{les droits personnels}~: le lien de droit entre deux personnes
(spé. droits extra patrimoniaux).

Distinguer, en droit réel~:\\
(i) choses corporelles / choses incorporelles~;\\
(ii) choses meubles (et meubles par destination) / choses immeuble (et
immeubles par destination)~;\\
(iii) choses appropriées, choses communes (ex~: \emph{res communis},
base~: \textbf{\hl{Article 714 du Code Civil}}), choses sans maître
(ex~: \emph{res nullius}, \emph{res derelicatae})~;\\
(iv) choses fongibles et choses non-fongibles~;\\
(v) choses consomptibles et choses non-consomptibles.

\textbf{\hl{Article 544 du Code Civil}} «~la propriété est le droit de
jouir des choses de la manière la plus absolue, pourvu qu'on n'en fasse
pas un usage prohibé par les lois ou les règlements.~»

Distinguer~: (i) \emph{usus~;} (ii) \emph{fructus~;} (iii) \emph{absus.}

Distinguer~: (i) copropriété (ii) indivision (iii) usufruit (iv)
servitude.

Distinguer, en droit personnel~:\\
(i) obligation de faire ou de ne pas faire~;\\
(ii) obligation de payer et obligation en nature~;\\
(iii) obligation de moyen et obligation de résultat.

\textbf{ORGANISATION DE LA JUSTICE}

\includegraphics[width=4.61667in,height=2.50273in]{media/image8.png}

On saisit toujours le tribunal du domicile du \hl{défendeur.}

Distinguer, en \hl{organisations supranationales~:}\\
(i) la Cour Européenne des Droits de l'Homme (CEDH)~;\\
(ii) la Cour de Justice de l'Union Européenne (CJUE) comprenant la Cour
de Justice et le Tribunal.

\hl{Autorités administratives}~: CNIL, Autorité de la Concurrence,
Agence européenne des produits chimiques, European Medecines Agency,
INPI

\end{document}
