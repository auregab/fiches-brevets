% Options for packages loaded elsewhere
\PassOptionsToPackage{unicode}{hyperref}
\PassOptionsToPackage{hyphens}{url}
%
\documentclass[
]{article}
\usepackage{amsmath,amssymb}
\usepackage{iftex}
\ifPDFTeX
  \usepackage[T1]{fontenc}
  \usepackage[utf8]{inputenc}
  \usepackage{textcomp} % provide euro and other symbols
\else % if luatex or xetex
  \usepackage{unicode-math} % this also loads fontspec
  \defaultfontfeatures{Scale=MatchLowercase}
  \defaultfontfeatures[\rmfamily]{Ligatures=TeX,Scale=1}
\fi
\usepackage{lmodern}
\ifPDFTeX\else
  % xetex/luatex font selection
\fi
% Use upquote if available, for straight quotes in verbatim environments
\IfFileExists{upquote.sty}{\usepackage{upquote}}{}
\IfFileExists{microtype.sty}{% use microtype if available
  \usepackage[]{microtype}
  \UseMicrotypeSet[protrusion]{basicmath} % disable protrusion for tt fonts
}{}
\makeatletter
\@ifundefined{KOMAClassName}{% if non-KOMA class
  \IfFileExists{parskip.sty}{%
    \usepackage{parskip}
  }{% else
    \setlength{\parindent}{0pt}
    \setlength{\parskip}{6pt plus 2pt minus 1pt}}
}{% if KOMA class
  \KOMAoptions{parskip=half}}
\makeatother
\usepackage{xcolor}
\usepackage{longtable,booktabs,array}
\usepackage{calc} % for calculating minipage widths
% Correct order of tables after \paragraph or \subparagraph
\usepackage{etoolbox}
\makeatletter
\patchcmd\longtable{\par}{\if@noskipsec\mbox{}\fi\par}{}{}
\makeatother
% Allow footnotes in longtable head/foot
\IfFileExists{footnotehyper.sty}{\usepackage{footnotehyper}}{\usepackage{footnote}}
\makesavenoteenv{longtable}
\usepackage{graphicx}
\makeatletter
\def\maxwidth{\ifdim\Gin@nat@width>\linewidth\linewidth\else\Gin@nat@width\fi}
\def\maxheight{\ifdim\Gin@nat@height>\textheight\textheight\else\Gin@nat@height\fi}
\makeatother
% Scale images if necessary, so that they will not overflow the page
% margins by default, and it is still possible to overwrite the defaults
% using explicit options in \includegraphics[width, height, ...]{}
\setkeys{Gin}{width=\maxwidth,height=\maxheight,keepaspectratio}
% Set default figure placement to htbp
\makeatletter
\def\fps@figure{htbp}
\makeatother
\ifLuaTeX
  \usepackage{luacolor}
  \usepackage[soul]{lua-ul}
\else
  \usepackage{soul}
\fi
\setlength{\emergencystretch}{3em} % prevent overfull lines
\providecommand{\tightlist}{%
  \setlength{\itemsep}{0pt}\setlength{\parskip}{0pt}}
\setcounter{secnumdepth}{-\maxdimen} % remove section numbering
\ifLuaTeX
  \usepackage{selnolig}  % disable illegal ligatures
\fi
\IfFileExists{bookmark.sty}{\usepackage{bookmark}}{\usepackage{hyperref}}
\IfFileExists{xurl.sty}{\usepackage{xurl}}{} % add URL line breaks if available
\urlstyle{same}
\hypersetup{
  hidelinks,
  pdfcreator={LaTeX via pandoc}}

\author{}
\date{}

\begin{document}

Conventions internationales sur les brevets

\begin{itemize}
\item
  1883 Convention de Paris
\item
  1967 convention instituant l'OMPI
\item
  1970 PCT
\item
  1971 arranegement de Strasbourg
\item
  1977 Traité de Budapest~: microorganismes
\item
  1994~: accord sur adpic (administré par l'OMC)
\item
  2000 PLT sur le droit des brevets
\end{itemize}

Européen~:

\begin{itemize}
\item
  1973 convention sur le brevet européen
\item
  1978 traité entre lichtenstein et suisse
\item
  2023 Brevet unitaire
\end{itemize}

Régionaux~:

\begin{itemize}
\item
  \_\_\_
\end{itemize}

Convention de Paris pour la protection de la propriété industrielle 1883

\begin{itemize}
\item
  Langue faisait foi~: français
\item
  181 «~pays parties~»
\item
  A1.1 une seule entité juridique
\item
  A1.2 protection de la propriété industrielle
\item
  A12 établissement de services nationaux de PI
\item
  A19 arrangements particuliers~: les membres entre eux peuvent convenir
  d'arrangement particuliers~; par exemple un délai de priorité de 12
  mois mais peut être supérieur à 12 mois
\item
  A4.D droit de priorité~; A5bis par exemple délai de grâce de 6 mois
  minimum pour le paiement des taxes de maintien~; A11 protection
  temporaire dans le cadre d'expositions internationales officielles
\item
  A25 donner effet à la Convention en droit national, intégrer la CP
  dans le droit national
\item
  A4ter l'inventeur doit être nommé
\item
  Question du ressort de chaque pays~: critères de brevetabilité, examen
  sur le fond ou pas
\item
  A4.A.3 dépôt national régulier = date de dépôt = premier demande
\item
  A4.B 4.C.4 dépôt ultérieur
\end{itemize}

Convention instituant l'OMPI~ 1967~:

\begin{itemize}
\item
  Langues faisant foi~: français, anglais, espagnol et russe
\item
  194 Etats membres
\item
  Administre 25 traités internationaux, dont 5 traités dans le domaine
  des brevets CUP, PCT, PLT, Strasbourg, Budapest
\end{itemize}

PCT 1970

\begin{itemize}
\item
  Langues faisant foi~: français et anglais
\item
  158 états contractants
\end{itemize}

Arrangement de Strasbourg~:

\begin{itemize}
\item
  Langue faisant foi~: français et anglais
\item
  67 «~pays~» membres
\item
  Entrée en vigueur 1 an après (pas 3 mois)
\item
  Arrangement particulier au sens de la convention de Paris~
\item
  \ul{Classification internationale des brevets}
\end{itemize}

\ul{Traité de Budapest sur la reconnaissance internationale du dépôt des
micro-organismes aux} fins de la procédure en matière de brevet 1977

\begin{itemize}
\item
  Divulgation d'un micro-organisme auprès d'une autorité reconnue
\end{itemize}

\textbf{Traité sur le droit des brevets PLT 2000}

\begin{itemize}
\item
  Langues faisant foi~: français, anglais, arabe, chinois, espagnol,
  russe
\item
  43 membres sont «~parties contractantes~»
\item
  Traité sur la forme (suite de la convention de paris) dans le but
  d'harmoniser et rationaliser les procédures de forme
\end{itemize}

\textbf{Accord sur les aspects des droits de la propriété intellectuelle
qui touchent au commerce (Accord sur les ADPIC) 1994}

\begin{itemize}
\item
  Introduit pour la 1\textsuperscript{ère} fois des règles De PI dans le
  système commercial multilatéral
\item
  166 membres, pas nécessairement des états mais aussi des territoires
  douniers
\item
  Obligation aux membres de l'OMC de se conformer aux droits et
  obligations de la Convention de Paris A1-12 et A19 et Convention de
  Berne
\end{itemize}

OMPI est dépositaire de tous les textes de loi des membres de l'OMC cf.
WIPOLex

\textbf{Accord de Bangui relatif à l'OAPI et au système dépôt et de
délivrance de brevets régionaux OAPI 1977}

\begin{itemize}
\item
  CUP «~arrangement particulier~», PCT «~traité de brevet régional~»
\item
  17 membres «~états~»
\item
  Brevets unitaire // traité suisse-lichenstein
\item
  Etats russophones
\end{itemize}

\textbf{Protocole de Harare relatif aux brevets et dessins et modèles
industriels dans le cadre de l'ARIPO 1982}

\begin{itemize}
\item
  CUP «~arrangement particulier~» A19, PCT «~traité de brevet régional~»
  A45
\item
  21 états membres
\item
  Système de dépôt de demandes régionales de brevets et de délivrance de
  brevets régionaux // CBE
\item
  Etats anglophones
\end{itemize}

\textbf{Convention sur le brevet eurasien CBEA 1994}

\begin{itemize}
\item
  A vocation régionale
\item
  CUP «~arrangement particulier~», PCT «~traité de brevet régional~»
\item
  8 membres «~états~»
\item
  // CBE
\end{itemize}

\textbf{Convention sur le brevet européen 1973}

\begin{itemize}
\item
  CUP «~arrangement particulier~», PCT «~traité de brevet régional~»
\item
  39 membres «~états~»
\item
  Système de dépôt de mandes régionales de brevets et délivrance de
  brevets régionaux
\end{itemize}

\textbf{Traité entre le Liechtenstein et la Suisse 1978}

\begin{itemize}
\item
  A vocation régionale unitaire
\item
  CUP «~arrangement particulier~» A19, PCT «~traité de brevet régional~»
  A45, CBE «~brevet unitaire~» A142
\item
  Désignation automatique du Liechtenstein et la Suisse
\end{itemize}

\textbf{Brevet européen à effet unitaire 2023}

\begin{itemize}
\item
  18 membres «~états~»
\end{itemize}

\textbf{Traité de l'OMPI sur la propriété intellectuelle, les ressources
génétiques et les savoirs traditionnels associés 2024}

\begin{itemize}
\item
  Concerne une question de divulgation dans les demandes de brevets~:
  exigence additionnelle de divulgation supplémentaire si l'invention
  revendiquée est «~sensiblement/directement fondée~» sur des ressources
  génétiques et savoirs traditionnels associés
\end{itemize}

\textbf{Points récapitulatif et comparaisons}

\begin{longtable}[]{@{}
  >{\raggedright\arraybackslash}p{(\columnwidth - 0\tabcolsep) * \real{1.0000}}@{}}
\toprule\noalign{}
\begin{minipage}[b]{\linewidth}\raggedright
\includegraphics[width=6.3in,height=4.68194in]{media/image1.png}
\end{minipage} \\
\midrule\noalign{}
\endhead
\bottomrule\noalign{}
\endlastfoot
\end{longtable}

\end{document}
