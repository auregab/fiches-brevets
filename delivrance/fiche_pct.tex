% Options for packages loaded elsewhere
\PassOptionsToPackage{unicode}{hyperref}
\PassOptionsToPackage{hyphens}{url}
%
\documentclass[
]{article}
\usepackage{amsmath,amssymb}
\usepackage{iftex}
\ifPDFTeX
  \usepackage[T1]{fontenc}
  \usepackage[utf8]{inputenc}
  \usepackage{textcomp} % provide euro and other symbols
\else % if luatex or xetex
  \usepackage{unicode-math} % this also loads fontspec
  \defaultfontfeatures{Scale=MatchLowercase}
  \defaultfontfeatures[\rmfamily]{Ligatures=TeX,Scale=1}
\fi
\usepackage{lmodern}
\ifPDFTeX\else
  % xetex/luatex font selection
\fi
% Use upquote if available, for straight quotes in verbatim environments
\IfFileExists{upquote.sty}{\usepackage{upquote}}{}
\IfFileExists{microtype.sty}{% use microtype if available
  \usepackage[]{microtype}
  \UseMicrotypeSet[protrusion]{basicmath} % disable protrusion for tt fonts
}{}
\makeatletter
\@ifundefined{KOMAClassName}{% if non-KOMA class
  \IfFileExists{parskip.sty}{%
    \usepackage{parskip}
  }{% else
    \setlength{\parindent}{0pt}
    \setlength{\parskip}{6pt plus 2pt minus 1pt}}
}{% if KOMA class
  \KOMAoptions{parskip=half}}
\makeatother
\usepackage{xcolor}
\usepackage{longtable,booktabs,array}
\usepackage{multirow}
\usepackage{calc} % for calculating minipage widths
% Correct order of tables after \paragraph or \subparagraph
\usepackage{etoolbox}
\makeatletter
\patchcmd\longtable{\par}{\if@noskipsec\mbox{}\fi\par}{}{}
\makeatother
% Allow footnotes in longtable head/foot
\IfFileExists{footnotehyper.sty}{\usepackage{footnotehyper}}{\usepackage{footnote}}
\makesavenoteenv{longtable}
\usepackage{graphicx}
\makeatletter
\def\maxwidth{\ifdim\Gin@nat@width>\linewidth\linewidth\else\Gin@nat@width\fi}
\def\maxheight{\ifdim\Gin@nat@height>\textheight\textheight\else\Gin@nat@height\fi}
\makeatother
% Scale images if necessary, so that they will not overflow the page
% margins by default, and it is still possible to overwrite the defaults
% using explicit options in \includegraphics[width, height, ...]{}
\setkeys{Gin}{width=\maxwidth,height=\maxheight,keepaspectratio}
% Set default figure placement to htbp
\makeatletter
\def\fps@figure{htbp}
\makeatother
\ifLuaTeX
  \usepackage{luacolor}
  \usepackage[soul]{lua-ul}
\else
  \usepackage{soul}
\fi
\setlength{\emergencystretch}{3em} % prevent overfull lines
\providecommand{\tightlist}{%
  \setlength{\itemsep}{0pt}\setlength{\parskip}{0pt}}
\setcounter{secnumdepth}{-\maxdimen} % remove section numbering
\ifLuaTeX
  \usepackage{selnolig}  % disable illegal ligatures
\fi
\IfFileExists{bookmark.sty}{\usepackage{bookmark}}{\usepackage{hyperref}}
\IfFileExists{xurl.sty}{\usepackage{xurl}}{} % add URL line breaks if available
\urlstyle{same}
\hypersetup{
  hidelinks,
  pdfcreator={LaTeX via pandoc}}

\author{}
\date{}

\begin{document}

\begin{enumerate}
\def\labelenumi{\arabic{enumi}.}
\item
  \textbf{\ul{OBTENIR UNE DATE DE DEPOT INTERNATIONAL}}
\end{enumerate}

\begin{longtable}[]{@{}
  >{\raggedright\arraybackslash}p{(\columnwidth - 4\tabcolsep) * \real{0.2345}}
  >{\raggedright\arraybackslash}p{(\columnwidth - 4\tabcolsep) * \real{0.5782}}
  >{\raggedright\arraybackslash}p{(\columnwidth - 4\tabcolsep) * \real{0.1874}}@{}}
\toprule\noalign{}
\begin{minipage}[b]{\linewidth}\raggedright
\end{minipage} & \begin{minipage}[b]{\linewidth}\raggedright
\textbf{Qui~?}
\end{minipage} & \begin{minipage}[b]{\linewidth}\raggedright
Art 11.1
\end{minipage} \\
\midrule\noalign{}
\endhead
\bottomrule\noalign{}
\endlastfoot
En principe & - un déposants qui a (i) un lien (domicile ou nationalité)
(ii) avec l'un quelconque des états contractants (iii) à la date du
dépôt & Art 11.1.i) \\
& - déposant peut être une personne physique ou une personne morale &
R18 \\
& - plusieurs co-déposants & \\
& \textbf{Effet~?} & \\
& - date de dépôt effectif dans chacun des états contractants (~?) & Art
11.3 \\
& - dépôt national régulier au sens CUP dans chacun des états
contractants désignés lors du dépôt & Art 11.4 \\
& \textbf{Où~?} & Art 10 \\
RO compétent & - un office national de l'état PCT où le demandeur est
domicilié & R 19.1.a)i) \\
& - un office national de l'état PCT om le demandeur est national & R
19.1.a)ii) \\
& - RO/IB & R 19.1.a)iii) \\
& - une organisation intergouvernemental ie. un office régional & R
19.1.b) \\
& - un office national agissant pour l'état dont un demandeur a la
nationalité ou est domicilié & \\
& \textbf{Langue~?} & Art 3 \\
Principe & - une langue que le RO accepte & R 12.1.a) \\
& - une seule langue & Art 3.4.i) \\
& - (i) une langue de publication cf. liste \ul{et} (ii) une langue
acceptée par une des ISA compétente pour ce RO & R 12.1.b) \\
& \textbf{Actes~?} & \\
A minima & - requête (i) indication «~PCT~»(ii) désignation d'un état
contractant et (iii) le nom du déposant & Art 4.1 \\
& - une description & \\
& - une revendication & \\
\end{longtable}

\begin{enumerate}
\def\labelenumi{\arabic{enumi}.}
\setcounter{enumi}{1}
\item
  \textbf{\ul{DESIGNER DES ETATS}}
\end{enumerate}

\begin{longtable}[]{@{}
  >{\raggedright\arraybackslash}p{(\columnwidth - 4\tabcolsep) * \real{0.2345}}
  >{\raggedright\arraybackslash}p{(\columnwidth - 4\tabcolsep) * \real{0.5782}}
  >{\raggedright\arraybackslash}p{(\columnwidth - 4\tabcolsep) * \real{0.1874}}@{}}
\toprule\noalign{}
\begin{minipage}[b]{\linewidth}\raggedright
\end{minipage} & \begin{minipage}[b]{\linewidth}\raggedright
\textbf{Désignation à exclure~?}
\end{minipage} & \begin{minipage}[b]{\linewidth}\raggedright
R. 4.9
\end{minipage} \\
\midrule\noalign{}
\endhead
\bottomrule\noalign{}
\endlastfoot
Priorité & - une demande de brevet DE, JP ou KR & \\
Retrait DE & - avant d'entrer en phase nationale & GD Annexe 8 \\
Retrait JP & - 16 mois / la date de dépôt de la demande de priorité
concernée & \\
Retrait KR & - 15 mois / la date de priorité & \\
& \textbf{Effets~?} & \\
Pas d'exclusion & - la demande nationale antérieure sera déclarée
retirée par l'office national & \\
Exclusion & - pas d'effet négatif sur la demande PCT & \\
& \textbf{Actes~?} & \\
\multirow{3}{*}{Requises pour obtenir une date de dépôt} & - dessins
nécessaire à la compréhension de l'invention & R. 7 \\
& - indications relatives à du matériel biologique & \\
& - listage des séquences de nucléotides & \\
\multirow{6}{*}{Requises mais pas pour obtenir une date de dépôt} & -
requête sous formulaire & \\
& - signature d'au moins un déposant quel qu'il soit &
\multirow{3}{*}{Art. 14 R. 4, 26, 51bis, 90} \\
& - abrégé \\
& - taxes \\
& - choix de l'ISA lorsqu'au moins deux ISA sont compétentes &
\multirow{2}{*}{Art. 10, 16, 32 R. 19, 35, 45bis, 59} \\
& - traduction \\
& \textbf{Anticiper~?} & \multirow{4}{*}{Art. 27 R. 4, 26ter, 48.2} \\
\multirow{5}{*}{Exigences nationales} & - identité de l'inventeur (pour
le droit US) \\
& - droit de demander et d'obtenir un brevet \\
& - droit de revendiquer la priorité \\
& - qualité d'inventeur & \\
& - divulgations non opposables & \\
& \textbf{Délai~?} & \\
pour remise & - 16 mois / la date de priorité & \\
\end{longtable}

\begin{enumerate}
\def\labelenumi{\arabic{enumi}.}
\setcounter{enumi}{2}
\item
  \textbf{\ul{REVENDIQUER LA PRIORITE}}
\end{enumerate}

\begin{longtable}[]{@{}
  >{\raggedright\arraybackslash}p{(\columnwidth - 4\tabcolsep) * \real{0.2345}}
  >{\raggedright\arraybackslash}p{(\columnwidth - 4\tabcolsep) * \real{0.5782}}
  >{\raggedright\arraybackslash}p{(\columnwidth - 4\tabcolsep) * \real{0.1874}}@{}}
\toprule\noalign{}
\begin{minipage}[b]{\linewidth}\raggedright
\end{minipage} & \begin{minipage}[b]{\linewidth}\raggedright
\textbf{Conditions~?}
\end{minipage} &
\multirow{3}{*}{\begin{minipage}[b]{\linewidth}\raggedright
PCT Art. 8; R. 4, 17

CUP Art. 4C
\end{minipage}} \\
\begin{minipage}[b]{\linewidth}\raggedright
\end{minipage} & \begin{minipage}[b]{\linewidth}\raggedright
- pays parties à la CUP
\end{minipage} \\
\begin{minipage}[b]{\linewidth}\raggedright
\end{minipage} & \begin{minipage}[b]{\linewidth}\raggedright
- membres de l'OMC (via accords ADPIC)
\end{minipage} \\
\midrule\noalign{}
\endhead
\bottomrule\noalign{}
\endlastfoot
& \textbf{Délais~?} & \multirow{4}{*}{PCT Art. 1, 2.xi), 8; R. 2, 4

CUP Art. 4C} \\
Dépôt sous priorité & - 12 mois / date de dépôt de la
1\textsuperscript{ère} demande \\
Doc. de priorité & - 16 mois / date de priorité \\
Rev. de priorité & - à tout moment pendant la phase internationale \\
& \textbf{Actes~?} & \\
\multirow{5}{*}{Pour revendiquer la priorité} & - date du dépôt
antérieur & R 4.10.a)i) \\
& - numéro du dépôt antérieur & R 4.10.a)ii) \\
& - pays du dépôt antérieur lorsque demande nationale & R 4.10.a)iii) \\
& - office du dépôt antérieur lorsque demande régionale & R
4.10.a)iv) \\
& - RO du dépôt antérieur lorsque demande PCT & R 4.10.a)v) \\
& \textbf{Forme~?} & \\
& - dans la requête & \\
& - sous forme d'une déclaration de priorité & \\
& \textbf{Document de priorité~?} & \\
& - copie certifiée conforme & R 17.1.a) \\
Remettre à & - au RO ou à IB & \\
\end{longtable}

\begin{enumerate}
\def\labelenumi{\arabic{enumi}.}
\setcounter{enumi}{3}
\item
  \textbf{\ul{LA RECHERCHE CHAPITRE I}}
\end{enumerate}

\begin{longtable}[]{@{}
  >{\raggedright\arraybackslash}p{(\columnwidth - 4\tabcolsep) * \real{0.2345}}
  >{\raggedright\arraybackslash}p{(\columnwidth - 4\tabcolsep) * \real{0.5782}}
  >{\raggedright\arraybackslash}p{(\columnwidth - 4\tabcolsep) * \real{0.1874}}@{}}
\toprule\noalign{}
\begin{minipage}[b]{\linewidth}\raggedright
\textbf{Recherche}
\end{minipage} & \begin{minipage}[b]{\linewidth}\raggedright
\textbf{Effet~?}
\end{minipage} &
\multirow{4}{*}{\begin{minipage}[b]{\linewidth}\raggedright
Art. 15, 16, 17; R. 12bis, 13, 13ter, 16, 33, 34, 39, 41
\end{minipage}} \\
\begin{minipage}[b]{\linewidth}\raggedright
\end{minipage} & \begin{minipage}[b]{\linewidth}\raggedright
- découvrir l'état de la technique pertinent
\end{minipage} \\
\begin{minipage}[b]{\linewidth}\raggedright
\end{minipage} & \begin{minipage}[b]{\linewidth}\raggedright
\textbf{Date pertinente~?}
\end{minipage} \\
\begin{minipage}[b]{\linewidth}\raggedright
\end{minipage} & \begin{minipage}[b]{\linewidth}\raggedright
\textbf{- date du dépôt PCT}
\end{minipage} \\
\midrule\noalign{}
\endhead
\bottomrule\noalign{}
\endlastfoot
& \textbf{Motifs~?} & \multirow{8}{*}{Art. 17, 18; R. 13ter, 39, 43
Accord entre chaque ISA et le BI} \\
Non-établissement & - objet «~exclu de la recherche~» \\
& - qualité de rédaction insuffisante \\
& - liste de séquence non conforme \\
& \textbf{Contenu~?} \\
& - citation des documents pertinents \\
& - classement CIB \\
& - domaines sur lesquels la recherche a porté \\
\textbf{Opinion écrite} & \textbf{Effet~?} & \multirow{2}{*}{art
33.1} \\
& - une opinion préliminaire et sans engagement \\
& \textbf{Contenu~?} & \\
& - une opinion sur la nouveauté, activité et application industrielle
& \\
& \textbf{Date pertinente~?} & \\
& \textbf{- date de priorité} & \\
& \textbf{Délai~?} & \multirow{3}{*}{Art. 18 R. 42, 44} \\
En théorie pour l'office & \begin{minipage}[t]{\linewidth}\raggedright
- délai maximum entre\\
(i) 3 mois / réception par ISA de la copie de recherche\\
(ii) 9 mois / date de priorité\strut
\end{minipage} \\
En pratique & \textbf{- 16 mois / date de priorité} \\
& \textbf{Réponse~?} & \multirow{6}{*}{Art. 19, 29, 31, 34 R. 46, 48,
53, 66} \\
& - modifier les revendications =\textgreater{} attention révèle aux
tiers ce qui intéresse le demandeur \\
& - déposer des observations informelles =\textgreater{} attention
seulement en cas d'erreur manifeste \\
& - demander une recherche supplémentaire \\
& - demander un examen préliminaire selon Chapitre II

- ne rien faire à ce stade \\
& - changer d'avis \\
\multirow{2}{*}{\textbf{Rapport pré. inter. s/ brevetabilité ch. I}} &
\textbf{Délai~?} & R. 44bis \\
& - 28 mois / date de priorité & \\
\end{longtable}

\begin{enumerate}
\def\labelenumi{\arabic{enumi}.}
\setcounter{enumi}{4}
\item
  \textbf{\ul{DEMANDER L'EXAMEN CHAPITRE II}}
\end{enumerate}

\begin{longtable}[]{@{}
  >{\raggedright\arraybackslash}p{(\columnwidth - 4\tabcolsep) * \real{0.2345}}
  >{\raggedright\arraybackslash}p{(\columnwidth - 4\tabcolsep) * \real{0.5782}}
  >{\raggedright\arraybackslash}p{(\columnwidth - 4\tabcolsep) * \real{0.1874}}@{}}
\toprule\noalign{}
\begin{minipage}[b]{\linewidth}\raggedright
\textbf{Dem. l'examen}
\end{minipage} & \begin{minipage}[b]{\linewidth}\raggedright
\textbf{Effet~?}
\end{minipage} &
\multirow{4}{*}{\begin{minipage}[b]{\linewidth}\raggedright
Art. 33, 34, 35

R. 66, 70
\end{minipage}} \\
\begin{minipage}[b]{\linewidth}\raggedright
\end{minipage} & \begin{minipage}[b]{\linewidth}\raggedright
- saisir l'option de poursuivre la phase d'examen
\end{minipage} \\
\begin{minipage}[b]{\linewidth}\raggedright
\end{minipage} & \begin{minipage}[b]{\linewidth}\raggedright
- intervenir dans la procédure
\end{minipage} \\
\begin{minipage}[b]{\linewidth}\raggedright
\end{minipage} & \begin{minipage}[b]{\linewidth}\raggedright
- engager des phases nationales dans de meilleurs conditions
\end{minipage} \\
\midrule\noalign{}
\endhead
\bottomrule\noalign{}
\endlastfoot
\multirow{2}{*}{} & \textbf{Délai~?} &
\multirow{2}{*}{\begin{minipage}[t]{\linewidth}\raggedright
R 54bis.1.a)\\
R 54bis.1.a)i)\\
54bis.1.a)ii)\strut
\end{minipage}} \\
& \begin{minipage}[t]{\linewidth}\raggedright
- délai maximal entre\\
(i) 3 mois / transmission du RRI et OE de l'ISA\\
\textbf{(ii) 22 mois / date de priorité}\strut
\end{minipage} \\
& \textbf{Par qui~?} & \multirow{3}{*}{Art. 31, 32 R. 54, 54bis, 59} \\
& - IPEA spécifiée par RO \\
& - le déposant choisit lorsque plusieurs IPEA spécifiées par RO \\
& \textbf{Conditions~?} & \multirow{3}{*}{Art. 31, 64.1 R. 54} \\
& - au moins un déposant a un lien avec un état contractant lié par le
chapitre II \\
& - dépôt de la demande PCT auprès l'office récepteur d'un état
contractant lié par le ch. II \\
& \textbf{Actes~?} & \multirow{6}{*}{Art. 31, 32, 64,1 R. 53, 54, 55,
57, 58, 59, 90} \\
& - présenter une demande d'IPE \\
& - élire les états éligibles \\
& \begin{minipage}[t]{\linewidth}\raggedright
- présenter une traduction dans une langue\\
(i) acceptée par l'IPEA\\
(ii) une langue de publication\strut
\end{minipage} \\
& - préciser la base sur laquelle l'examen devrait commencer~: demande
modifiée ou déposée \\
& - payer à l'IPEA la taxe de traitement au profit de l'IB et la taxe
d'examen pré. au profit de l'IPEA \\
\textbf{1\textsuperscript{ère} opinion écrite} & \textbf{Modalités~?} &
R. 61, 66, 69 \\
Principe & - reprend l'opinion écrite de l'ISA & \\
& - nouvelle opinion écrite lorsque IPEA = EP et ISA =/= EP & \\
\textbf{2\textsuperscript{e} opinion écrite} & \textbf{Réponse~?} &
\multirow{3}{*}{Art. 34 R. 66, 69} \\
& - répondre à la deuxième opinion écrite \\
& - ne pas répondre \\
\multirow{2}{*}{\textbf{Rpt. Exam. Pré. Inter. Ch. II}} &
\textbf{Délai~?} & \multirow{2}{*}{Art. 19, 35, 36 R. 64, 69, 70, 71,
72, 73, 74, 91, 94} \\
& \begin{minipage}[t]{\linewidth}\raggedright
- délai maximal entre\\
(i) 28 mois / date de priorité\\
(ii) 6 mois / début de l'examen\strut
\end{minipage} \\
\end{longtable}

\begin{enumerate}
\def\labelenumi{\arabic{enumi}.}
\setcounter{enumi}{5}
\item
  \textbf{\ul{LA PUBLICATION}}
\end{enumerate}

\begin{longtable}[]{@{}
  >{\raggedright\arraybackslash}p{(\columnwidth - 4\tabcolsep) * \real{0.2345}}
  >{\raggedright\arraybackslash}p{(\columnwidth - 4\tabcolsep) * \real{0.5782}}
  >{\raggedright\arraybackslash}p{(\columnwidth - 4\tabcolsep) * \real{0.1874}}@{}}
\toprule\noalign{}
\begin{minipage}[b]{\linewidth}\raggedright
\textbf{Publication}
\end{minipage} & \begin{minipage}[b]{\linewidth}\raggedright
\textbf{Délai~?}
\end{minipage} &
\multirow{4}{*}{\begin{minipage}[b]{\linewidth}\raggedright
Art. 19, 21 R. 26bis, 48, 86, 90bis
\end{minipage}} \\
\begin{minipage}[b]{\linewidth}\raggedright
\end{minipage} & \begin{minipage}[b]{\linewidth}\raggedright
\textbf{- \ul{18 mois} / date de priorité}
\end{minipage} \\
\begin{minipage}[b]{\linewidth}\raggedright
\end{minipage} & \begin{minipage}[b]{\linewidth}\raggedright
- publication hebdomadaire par IB
\end{minipage} \\
\begin{minipage}[b]{\linewidth}\raggedright
\end{minipage} & \begin{minipage}[b]{\linewidth}\raggedright
- préparation technique de la publication de 15 jours calendaires / date
de publication effective
\end{minipage} \\
\midrule\noalign{}
\endhead
\bottomrule\noalign{}
\endlastfoot
& \textbf{Langue~?} & \multirow{4}{*}{Art. 21 R. 48, 86} \\
& - dans l'une des dix langues de publication~: allemand, anglais,
arabe, chinois, coréen, espagnol, français, japonais, portugais,
russe \\
& - dans la langue de dépôt lorsque langue de dépôt = une langue de
publication \\
& - dans une langue de traduction \\
& \textbf{Effet~?} & \multirow{4}{*}{Art. 20, 21, 29, 30 R. 33, 34, 47,
48, 94} \\
& - entrée de la demande publiée dans l'état de la technique \\
& - protection provisoire~: attention à fournir une traduction pour
effet dans une législation nationale \\
& - accès des tiers au dossier du BI \\
& \textbf{Anticiper~?} & \multirow{3}{*}{Art. 21, 24, 64 R. 48,
90bis} \\
& - empêcher la publication de l'intégralité de la demande
internationale \\
& - empêcher la publication d'une partie de la demande internationale \\
& - retarder la publication =\textgreater{} attention à la revendication
de priorité & \\
& - avancer la publication & \\
& - différer la publication internationale lorsque seule la désignation
US reste dans la demande & \\
\textbf{Obs. des tiers} & \textbf{Motifs~?} & \multirow{2}{*}{Instr.
Adm. 801 à 805} \\
Limités à & - l'état de la technique \\
& \textbf{Délai~?} & \\
& - entre la date de publication et la fin du délai de 28 mois & \\
\end{longtable}

\begin{enumerate}
\def\labelenumi{\arabic{enumi}.}
\setcounter{enumi}{6}
\item
  \textbf{\ul{ENTRER EN PHASE NATIONALE}}
\end{enumerate}

\begin{longtable}[]{@{}
  >{\raggedright\arraybackslash}p{(\columnwidth - 4\tabcolsep) * \real{0.2345}}
  >{\raggedright\arraybackslash}p{(\columnwidth - 4\tabcolsep) * \real{0.5782}}
  >{\raggedright\arraybackslash}p{(\columnwidth - 4\tabcolsep) * \real{0.1874}}@{}}
\toprule\noalign{}
\begin{minipage}[b]{\linewidth}\raggedright
\textbf{Entrée en phase}
\end{minipage} & \begin{minipage}[b]{\linewidth}\raggedright
\textbf{Délais~?}
\end{minipage} &
\multirow{5}{*}{\begin{minipage}[b]{\linewidth}\raggedright
Art. 22, 39 R. 49, 76
\end{minipage}} \\
\begin{minipage}[b]{\linewidth}\raggedright
Principe
\end{minipage} & \begin{minipage}[b]{\linewidth}\raggedright
\textbf{- \ul{30 mois} / date de priorité}
\end{minipage} \\
\begin{minipage}[b]{\linewidth}\raggedright
Exceptions
\end{minipage} & \begin{minipage}[b]{\linewidth}\raggedright
- 20 mois / date de priorité pour LU et TZ
\end{minipage} \\
\begin{minipage}[b]{\linewidth}\raggedright
Rétablissement
\end{minipage} & \begin{minipage}[b]{\linewidth}\raggedright
- 42 mois / date de priorité
\end{minipage} \\
\begin{minipage}[b]{\linewidth}\raggedright
Anticipé
\end{minipage} & \begin{minipage}[b]{\linewidth}\raggedright
- dès 12 mois / date de priorité
\end{minipage} \\
\midrule\noalign{}
\endhead
\bottomrule\noalign{}
\endlastfoot
& \textbf{Actes~?} & \multirow{4}{*}{Art. 22, 39 R. 49, 49bis, 76} \\
& - demande expresse du déposant \\
& - traduire la demande \\
& - payer la taxe nationale \\
& \textbf{Effets~?} & \\
Publication & - publication nationale sous conditions & \\
Résultats & - recherche internationale (ISR) & \multirow{3}{*}{Art. 18,
19, 35 R. 92bis} \\
& - examen préliminaire international chapitre I ou chapitre II \\
Modifications & - revendications, description et dessins modifiés \\
\textbf{Exigences nat.} & \textbf{Actes ?} & Art. 27 \\
Exclus du PCT & - exigence nationale différente de celle prévue par le
PCT & \\
& - exigence nationale supplémentaire prévue par le PCT & \\
Admis par le PCT & - exigences nationales plus favorables pour le
déposant que celles du PCT & \multirow{6}{*}{Art. 27 R. 49, 51bis} \\
& - remise de preuves quant à des déclarations \\
& - signature du déposant \\
& - indications relatives au déposant \\
& - document de priorité \\
& - défense nationale \\
& - définition de l'état de la technique & \multirow{2}{*}{Art. 27 R.
51bis, 76} \\
& - traduction certifiée lorsque doute raisonnable \\
& \textbf{Effets~?} & \\
Garantissent & - un délai additionnel pour respecter les exigences
admises & \multirow{2}{*}{Art. 27 R. 17.1.c) et d), 49, 51bis} \\
& - un pouvoir de présenter des arguments en cas de non respect des
exigences admises \\
& \textbf{Conditions~?} & \\
Vérifier & \textbf{-} « Réserves et incompatibilités » sur la page PCT
du site Internet de l'OMPI & \\
\end{longtable}

\begin{enumerate}
\def\labelenumi{\arabic{enumi}.}
\setcounter{enumi}{7}
\item
  \textbf{\ul{BOITE A OUTILS}}
\end{enumerate}

\begin{longtable}[]{@{}
  >{\raggedright\arraybackslash}p{(\columnwidth - 4\tabcolsep) * \real{0.2436}}
  >{\raggedright\arraybackslash}p{(\columnwidth - 4\tabcolsep) * \real{0.5553}}
  >{\raggedright\arraybackslash}p{(\columnwidth - 4\tabcolsep) * \real{0.2011}}@{}}
\toprule\noalign{}
\begin{minipage}[b]{\linewidth}\raggedright
\textbf{Représentation}
\end{minipage} & \begin{minipage}[b]{\linewidth}\raggedright
\textbf{Qui~?}
\end{minipage} & \begin{minipage}[b]{\linewidth}\raggedright
Art. 49 R. 90
\end{minipage} \\
\midrule\noalign{}
\endhead
\bottomrule\noalign{}
\endlastfoot
Mandataire & - a le droit d'exercer auprès de l'office national & \\
Représentant commun & - un des déposants & \\
Modalité & - requête, pouvoir individuel, pouvoir général & \\
\multicolumn{3}{@{}>{\raggedright\arraybackslash}p{(\columnwidth - 4\tabcolsep) * \real{1.0000} + 4\tabcolsep}@{}}{%
\includegraphics[width=6.3in,height=3.70903in]{media/image1.png}} \\
\textbf{Délais} & \textbf{Calcul~?} & \\
& & Art. 47, 48

R. 79, 80, 82, 82bis, 82quater \\
& \textbf{Retards~?} & \\
& - cas de force majeure & \multirow{5}{*}{Art. 48 R. 80, 82bis,
82quater} \\
& - règle des 7 jours à réception d'un document fixant un délai \\
& - règle des 5 jours pour la réponse dans le délai \\
& - indisponibilité des moyens de communications électroniques
autorisés \\
& - du fait d'une épidémie ou de «~perturbations générales~» \\
\textbf{Taxes} & & \\
\multicolumn{3}{@{}>{\raggedright\arraybackslash}p{(\columnwidth - 4\tabcolsep) * \real{1.0000} + 4\tabcolsep}@{}}{%
\includegraphics[width=6.15711in,height=4.34436in]{media/image2.png}} \\
\textbf{Non-unité} & \textbf{Qui~?} & \multirow{3}{*}{Art. 17 R. 13,
40} \\
Pour ISA & - «~invention principale~» = la première revendiquée \\
& - invitation à payer une taxe pour chaque invention additionnelle sous
réserve \\
Pour IPEA & - «~invention principale~» = c'est le déposant qui la
choisit & \multirow{2}{*}{Art. 34 R. 13, 68} \\
& - invitation de l'IPEA à payer une taxe additionnelle pour chaque
invention additionnelle \\
Procédure de réserve & - payer une taxe de réserve & R. 40, 68 \\
& - valable pour la procédure auprès de l'ISA et de l'IPEA & \\
& - remboursement de taxes selon le résultat du réexamen & \\
\textbf{Accès au dossier} & \textbf{Qui~?} & Art. 30, 38 ; R. 17, 48,
94 \\
\multicolumn{3}{@{}>{\raggedright\arraybackslash}p{(\columnwidth - 4\tabcolsep) * \real{1.0000} + 4\tabcolsep}@{}}{%
\includegraphics[width=6.3in,height=3.49306in]{media/image3.png}} \\
\multirow{6}{*}{Exceptions à l'accès des tiers} & - document de priorité
(~?) & R. 17, 48, 94 \\
& - document de priorité (~?) & \\
& - document de priorité (~?) & \\
& - document pour l'usage interne du BI & \\
& - renseignements non publiés & \\
& - état de la procédure d'examen selon chapitre II & \\
\textbf{Rech. Inter. Sup.} & \textbf{Effet~?} & \\
& - réduire le risque de découverte tardive d'état de la technique & \\
& - accès à l'état de la technique dans diverses langues =/= langue de
travail de l'ISA & \\
& \textbf{Délai~?} & R. 45bis \\
& - 22 mois / date de priorité & \\
\textbf{Rech. Complt.} & \textbf{Modalités~?} & R. 66.1ter \\
& - commence automatiquement par l'IPEA & \\
& & \\
\end{longtable}

\begin{enumerate}
\def\labelenumi{\arabic{enumi}.}
\setcounter{enumi}{8}
\item
  \textbf{\ul{CORRIGER DES ERREURS}}
\end{enumerate}

\begin{longtable}[]{@{}
  >{\raggedright\arraybackslash}p{(\columnwidth - 4\tabcolsep) * \real{0.2434}}
  >{\raggedright\arraybackslash}p{(\columnwidth - 4\tabcolsep) * \real{0.5532}}
  >{\raggedright\arraybackslash}p{(\columnwidth - 4\tabcolsep) * \real{0.2034}}@{}}
\toprule\noalign{}
\begin{minipage}[b]{\linewidth}\raggedright
\textbf{Irrégularités}
\end{minipage} & \begin{minipage}[b]{\linewidth}\raggedright
\end{minipage} & \begin{minipage}[b]{\linewidth}\raggedright
\end{minipage} \\
\midrule\noalign{}
\endhead
\bottomrule\noalign{}
\endlastfoot
\multirow{3}{*}{Empêchant l'attribution d'une date de dépôt
international} & - aucun déposant habilité lors du dépôt & Art. 11,
25 \\
& - dépôt avec un élément manquant & \\
& - incompatibilité de la législation nationale & \\
\multirow{2}{*}{N'empêchant pas l'attribution d'une date de dépôt
internationale} & - catégories d'irrégularités & \multirow{2}{*}{Art. 14
; R. 4, 12, 16bis, 26, 26ter, 28, 29, 37, 38, 51bis} \\
& - procédure d'invitation de l'office récepteur \\
\textbf{Priorité} & & \multirow{3}{*}{Art. 2, 8 ; R. 4, 26bis.1,
26bis.2, 33, 48, 64 CUP Art. 4} \\
Délai & - 4 mois / date de dépôt PCT \\
Invitation & - lorsque RO ou IB constent l'irrégularité \\
Modalités & - respecte les exigences de la CUP & \\
Effet & - tous les délais non encore expriés sont recalculés & \\
& - effet sur l'état de la technique pertinent & \\
\textbf{Retraits} & & \\
\multicolumn{3}{@{}l@{}}{%
\includegraphics[width=6.3in,height=3.92014in]{media/image4.png}} \\
\end{longtable}

\begin{enumerate}
\def\labelenumi{\Alph{enumi}.}
\item
  \textbf{\ul{LA TERMINOLOGIE DU PCT}}
\end{enumerate}

\begin{longtable}[]{@{}
  >{\raggedright\arraybackslash}p{(\columnwidth - 4\tabcolsep) * \real{0.2345}}
  >{\raggedright\arraybackslash}p{(\columnwidth - 4\tabcolsep) * \real{0.5782}}
  >{\raggedright\arraybackslash}p{(\columnwidth - 4\tabcolsep) * \real{0.1874}}@{}}
\toprule\noalign{}
\begin{minipage}[b]{\linewidth}\raggedright
\end{minipage} & \begin{minipage}[b]{\linewidth}\raggedright
\textbf{Traité~?}
\end{minipage} &
\multirow{3}{*}{\begin{minipage}[b]{\linewidth}\raggedright
Préambule, art 1 et art 2
\end{minipage}} \\
\begin{minipage}[b]{\linewidth}\raggedright
ADPIC
\end{minipage} & \begin{minipage}[b]{\linewidth}\raggedright
Territoire dounier
\end{minipage} \\
\begin{minipage}[b]{\linewidth}\raggedright
PCT
\end{minipage} & \begin{minipage}[b]{\linewidth}\raggedright
\ul{Etat} contractant
\end{minipage} \\
\midrule\noalign{}
\endhead
\bottomrule\noalign{}
\endlastfoot
& \textbf{Début~?} & Art 63 \\
Entrée en vigueur & 24/01/1978 & \\
Date d'application & 01/06/1978 & \\
& \textbf{Configuration PCT~?} & \\
& - Gazette & \\
Relatif aux accords & - Accords internationaux OMPI / administrations
internationales & \\
Relatif au règlement~? & - Règlement d'exécution & \\
& - directives recherche et examen & \\
& - instructions administratives & \\
& - directives taxes & \\
Autres & - Instructions administratives & \\
& - directives offices récepteurs & \\
& \textbf{Termes spécifiques~?} & \\
Usuels & - phase internationale / phase nationale & Art2. ,9, 16, 32,
R2 \\
& - Chapitre I / chapitre II & \\
& - Demande d'examen & \\
\end{longtable}

\begin{enumerate}
\def\labelenumi{\Alph{enumi}.}
\setcounter{enumi}{1}
\item
  \textbf{\ul{LA PROCEDURE PCT EN RESUME}}
\end{enumerate}

\begin{longtable}[]{@{}
  >{\raggedright\arraybackslash}p{(\columnwidth - 4\tabcolsep) * \real{0.2345}}
  >{\raggedright\arraybackslash}p{(\columnwidth - 4\tabcolsep) * \real{0.5782}}
  >{\raggedright\arraybackslash}p{(\columnwidth - 4\tabcolsep) * \real{0.1874}}@{}}
\toprule\noalign{}
\begin{minipage}[b]{\linewidth}\raggedright
\textbf{Phase internationale}
\end{minipage} & \begin{minipage}[b]{\linewidth}\raggedright
\textbf{Chapitre I~?}
\end{minipage} & \begin{minipage}[b]{\linewidth}\raggedright
\end{minipage} \\
\midrule\noalign{}
\endhead
\bottomrule\noalign{}
\endlastfoot
0 mois & - priorité par le déposant à l'office national & \\
12 mois (9 mois sans priorité) & - dépôt PCT par le déposant au RO ou
RO-IB & \\
16 mois & - recherche internationale par ISA & \\
18 mois & - publication internationale & \\
22 mois & - recherche internationale supplémentaire par le déposant à
SISA & \\
& \textbf{Chapitre II~?} & \\
0 mois & - priorité par le déposant à l'office national & \\
12 mois (9 mois sans priorité) & - dépôt PCT par le déposant au RO ou
RO-IB & \\
16 mois & - début de l'examen par ISA & \\
18 mois & - publication internationale & \\
22 mois & - suite et fin de l'examen préliminaire par le déposant à IPEA
& \\
\textbf{Phase nationale} & \textbf{Cas de figure~?} & \\
30 mois & - abandon, délivrance, rejet & \\
31 mois & - abandon, délivrance, rejet & \\
30 mois & - aucune ouverture de la phase nationale & \\
\textbf{Moments critiques} & \textbf{Questions~?} & \\
Avant 12 mois & - déposer PCT~? & \\
Avant 18 mois & - empêcher/retarder la publication internationale & \\
Avant 22 mois & - demander une recherche internationale supplémentaire
& \\
Avant 22 mois & - demander la suite et fin de l'examen préliminaire
international Chapitre II & \\
Avant 30 mois & - entrer en phase nationale~? & \\
\end{longtable}

\end{document}
