% Options for packages loaded elsewhere
\PassOptionsToPackage{unicode}{hyperref}
\PassOptionsToPackage{hyphens}{url}
%
\documentclass[
]{article}
\usepackage{amsmath,amssymb}
\usepackage{iftex}
\ifPDFTeX
  \usepackage[T1]{fontenc}
  \usepackage[utf8]{inputenc}
  \usepackage{textcomp} % provide euro and other symbols
\else % if luatex or xetex
  \usepackage{unicode-math} % this also loads fontspec
  \defaultfontfeatures{Scale=MatchLowercase}
  \defaultfontfeatures[\rmfamily]{Ligatures=TeX,Scale=1}
\fi
\usepackage{lmodern}
\ifPDFTeX\else
  % xetex/luatex font selection
\fi
% Use upquote if available, for straight quotes in verbatim environments
\IfFileExists{upquote.sty}{\usepackage{upquote}}{}
\IfFileExists{microtype.sty}{% use microtype if available
  \usepackage[]{microtype}
  \UseMicrotypeSet[protrusion]{basicmath} % disable protrusion for tt fonts
}{}
\makeatletter
\@ifundefined{KOMAClassName}{% if non-KOMA class
  \IfFileExists{parskip.sty}{%
    \usepackage{parskip}
  }{% else
    \setlength{\parindent}{0pt}
    \setlength{\parskip}{6pt plus 2pt minus 1pt}}
}{% if KOMA class
  \KOMAoptions{parskip=half}}
\makeatother
\usepackage{xcolor}
\ifLuaTeX
  \usepackage{luacolor}
  \usepackage[soul]{lua-ul}
\else
  \usepackage{soul}
\fi
\setlength{\emergencystretch}{3em} % prevent overfull lines
\providecommand{\tightlist}{%
  \setlength{\itemsep}{0pt}\setlength{\parskip}{0pt}}
\setcounter{secnumdepth}{-\maxdimen} % remove section numbering
\ifLuaTeX
  \usepackage{selnolig}  % disable illegal ligatures
\fi
\IfFileExists{bookmark.sty}{\usepackage{bookmark}}{\usepackage{hyperref}}
\IfFileExists{xurl.sty}{\usepackage{xurl}}{} % add URL line breaks if available
\urlstyle{same}
\hypersetup{
  hidelinks,
  pdfcreator={LaTeX via pandoc}}

\author{}
\date{}

\begin{document}

\textbf{\hl{Article 1240 du Code Civil}} «~Tout \hl{fait (100)}
quelconque de l\textquotesingle homme, qui cause à autrui un dommage,
oblige celui par \hl{la faute (200)} duquel il est arrivé à le
réparer.~»

Distinguer le champ d'application matériel, \emph{ratione materiae}~:\\
\hl{(110) une action en concurrence déloyale} = appartient au champ
économique, comprenant le champ commercial càd entre sociétés\\
\hl{(120) une action en responsabilité} = hors du champ économique

Distinguer le champ d'application personnel, \emph{ratione personae~:\\
}\hl{(210) approche classique}~: action en concurrence déloyale limitée
par principe de spécialité (211), càd même domaine, et en concurrence
directe (212), càd même clientèle~;\hl{\hfill\break
(220) objectification de l'action~:} Cour de cassation élargit l'action
en concurrence déloyale au delà du rapport de concurrence entre les
entreprises {[}\hl{base~: Cass. com. 12 févr. 2008, n°06-17501}{]}.

«~\textbf{\hl{Une faute (200)}} en concurrence déloyale est un
comportement contraire aux usages professionnels (210) {[}\emph{in
abstracto} par une fiction juridique d'un professionnel honnête (211),
prudent (212) et scrupuleux en affaires (213), compte tenu des usages de
la profession{]}, rompant par la même l'égalité des chances qui doit en
principe existe entre les concurrents (220)~».

Distinguer~:\\
la concurrence déloyale~;\\
la concurrence anti-contractuelle~: violation d'une obligation~;\\
la concurrence illicite.

\textbf{\hl{Acte de désorganisation}} est une attaque de la capacité
commerciale d'une entreprise rivale

\textbf{\hl{Acte de confusion}}

\textbf{\hl{Acte de dénigrement}}

\textbf{\hl{Acte de parasitisme} «~}est l'ensemble des comportements par
lesquelles un agent économique s'immisce dans le sillage d'un autre afin
de tirer profit, sans rien dépenser, de ses efforts et de son
savoir-faire~» {[}\hl{base~: Cass. com., 26 janvier 1999, Bull. civ. IV,
n°25}{]}

\end{document}
