% Options for packages loaded elsewhere
\PassOptionsToPackage{unicode}{hyperref}
\PassOptionsToPackage{hyphens}{url}
%
\documentclass[
]{article}
\usepackage{amsmath,amssymb}
\usepackage{iftex}
\ifPDFTeX
  \usepackage[T1]{fontenc}
  \usepackage[utf8]{inputenc}
  \usepackage{textcomp} % provide euro and other symbols
\else % if luatex or xetex
  \usepackage{unicode-math} % this also loads fontspec
  \defaultfontfeatures{Scale=MatchLowercase}
  \defaultfontfeatures[\rmfamily]{Ligatures=TeX,Scale=1}
\fi
\usepackage{lmodern}
\ifPDFTeX\else
  % xetex/luatex font selection
\fi
% Use upquote if available, for straight quotes in verbatim environments
\IfFileExists{upquote.sty}{\usepackage{upquote}}{}
\IfFileExists{microtype.sty}{% use microtype if available
  \usepackage[]{microtype}
  \UseMicrotypeSet[protrusion]{basicmath} % disable protrusion for tt fonts
}{}
\makeatletter
\@ifundefined{KOMAClassName}{% if non-KOMA class
  \IfFileExists{parskip.sty}{%
    \usepackage{parskip}
  }{% else
    \setlength{\parindent}{0pt}
    \setlength{\parskip}{6pt plus 2pt minus 1pt}}
}{% if KOMA class
  \KOMAoptions{parskip=half}}
\makeatother
\usepackage{xcolor}
\usepackage{longtable,booktabs,array}
\usepackage{calc} % for calculating minipage widths
% Correct order of tables after \paragraph or \subparagraph
\usepackage{etoolbox}
\makeatletter
\patchcmd\longtable{\par}{\if@noskipsec\mbox{}\fi\par}{}{}
\makeatother
% Allow footnotes in longtable head/foot
\IfFileExists{footnotehyper.sty}{\usepackage{footnotehyper}}{\usepackage{footnote}}
\makesavenoteenv{longtable}
\ifLuaTeX
  \usepackage{luacolor}
  \usepackage[soul]{lua-ul}
\else
  \usepackage{soul}
\fi
\setlength{\emergencystretch}{3em} % prevent overfull lines
\providecommand{\tightlist}{%
  \setlength{\itemsep}{0pt}\setlength{\parskip}{0pt}}
\setcounter{secnumdepth}{-\maxdimen} % remove section numbering
\ifLuaTeX
  \usepackage{selnolig}  % disable illegal ligatures
\fi
\IfFileExists{bookmark.sty}{\usepackage{bookmark}}{\usepackage{hyperref}}
\IfFileExists{xurl.sty}{\usepackage{xurl}}{} % add URL line breaks if available
\urlstyle{same}
\hypersetup{
  hidelinks,
  pdfcreator={LaTeX via pandoc}}

\author{}
\date{}

\begin{document}

Conseils~:

\begin{itemize}
\item
  Lire partie G brevetabilité des directives OEB et directives INPI
\item
  Fiches de cours Jeannequin sur google
\item
  Lister tous les articles du CPI et les apprendre par cœur
\item
  Lire les G de la Grande Chambre de Recours
\item
\end{itemize}

Epreuve technique~:

\begin{itemize}
\item
  Brevetabilité sur 8 points
\item
  Contrefaçon sur 5-6 points
\item
  Question juridique 6 points
\item
  Répondre à \ul{toutes} les questions
\end{itemize}

Pages~:

\begin{itemize}
\item
  une première page~: pour la première question juridique
\item
  une deuxième page~: pour la deuxième question juridique
\item
  une première page~: pour la brevetabilité A138(1)e)
\item
  \_\_\_ deuxième page~: pour la brevetabilité A138(1)d)
\item
  \_\_\_ troisième page~: pour la nouveauté R1/D1
\item
  \_\_\_ quatrième page~: pour la nouveauté R1/D2
\item
  \_\_\_ cinquième page~: pour l'activité inventive
\end{itemize}

Sources~:

\begin{itemize}
\item
  CPI
\item
  CBE
\item
  PCT
\item
  AJUB
\end{itemize}

Directives G

\begin{longtable}[]{@{}
  >{\raggedright\arraybackslash}p{(\columnwidth - 0\tabcolsep) * \real{1.0000}}@{}}
\toprule\noalign{}
\begin{minipage}[b]{\linewidth}\raggedright
\textbf{Algorithme d'analyse de l'épreuve sur la brevetabilité}~:

\begin{itemize}
\item
  demande de brevet français

  \begin{itemize}
  \item
    nullité, juridiction compétence = tribunal judiciaire, source =
    \textbf{L613-25 CPI}
  \item
    opposition, \_\_\_ = INPI, source = \textbf{L613-23 CPI}
  \end{itemize}
\item
  demande de brevet européen

  \begin{itemize}
  \item
    nullité, \_\_\_ = tribunal judicaire français (pas espagnol,
    italien), source = \textbf{L614-12} pointe vers \textbf{A138CBE}
  \item
    nullité, \_\_\_ = JUB, source = \textbf{A65 AJUB} pointe vers
    \textbf{A138CBE}
  \item
    opposition, \_\_\_\_ = OEB, source = \textbf{A100CBE}
  \end{itemize}
\end{itemize}

Analyse des motifs~:

\begin{itemize}
\item
  commencer par le motif e) puis remonter à a)
\item
  en général, \ul{écrire} que le motif A138(1)e) n'est pas applicable
\item
  en général, passer rapidement sur le motif b), un paragraphe au plus
\item
  écrire une phrase pour expliquer que l'invention est brevetable, pas
  d'exception de la brevetabilité
\end{itemize}
\end{minipage} \\
\midrule\noalign{}
\endhead
\bottomrule\noalign{}
\endlastfoot
\end{longtable}

\textbf{MOTIFS}

\textbf{E) le titulaire du brevet européen n'avait pas le droit de
l'obtenir en vertu de l'article 60, paragraphe 1}

Passer rapidement dessus

\textbf{C) l'objet du brevet européen s'étend au-delà du contenu de la
demande telle qu'elle a été déposée // A123(2)}

Ajout de matière nouvelle dans les revendications ou la description

\textbf{D) la protection conférée par le brevet européen a été étendue
// A123(3)}

\emph{=\textgreater{} principe de protection des tiers}

Augmenter la portée des revendications est interdit

\ul{Piège 123(2) 123(3)}

R1 telle que déposée~: Composition caractérisée en ce que la composition
comprend un halogénure

R1 modifiée~: Composition caractérisée en ce que la composition comprend
un halogénure \ul{sauf le chlore} (non supportée) =\textgreater{}
A123(2)

R1 délivrée~: Composition caractérisée en ce que la composition comprend
un halogénure \st{sauf le chlore} =\textgreater{} A123(3)

\textbf{C) suffisance de description}

=\textgreater{} \emph{peut on réaliser l'invention}

R1~: \_\_\_ caractérisé en ce que l'élément X est \ul{rigide}

Problème de clarté~: Il est rigide par rapport à quoi~?

Mais, pas de problème pour expliquer comment réaliser l'invention.

Paragraphe 1 de la description~: l'axe est parallèle à la surface

Paragraphe 2 de la description~: l'axe est perpendiculaire à la surface

\begin{itemize}
\item
  problème de suffisance de description
\end{itemize}

R1~: \_\_\_\_ caractérisée en ce que le produit X comprend 3 éléments
dont A et B

Il manque le troisième élément

\begin{itemize}
\item
  problème de suffisance de description
\end{itemize}

\begin{enumerate}
\def\labelenumi{\Alph{enumi})}
\item
  \textbf{nouveauté}
\end{enumerate}

\textbf{Définir les objets}

Dans une même revendication

R1~: A+B+C =\textgreater{} 1 objet

R1~: A ou B ou C =\textgreater{} 3 objets

R1~: A et/ou B et/ou C =\textgreater{} 7 objets à analyser

Dans un ensemble de revendications

\begin{longtable}[]{@{}
  >{\raggedright\arraybackslash}p{(\columnwidth - 0\tabcolsep) * \real{1.0000}}@{}}
\toprule\noalign{}
\begin{minipage}[b]{\linewidth}\raggedright
R1~=\textgreater{} 1 objet R1
\end{minipage} \\
\midrule\noalign{}
\endhead
\bottomrule\noalign{}
\endlastfoot
R2 selon R1 =\textgreater{} 1 objet R1+R2 \\
R3 selon R2 =\textgreater{} 1 objet R1+R2+R3 \\
R4 selon R1 =\textgreater{} 1 objet R1+R4 \\
R5 selon R1 ou R2 =\textgreater{} 2 objet~: R1+R5 et R1+R2+R5 \\
\end{longtable}

\textbf{Identifier \ul{pour chacun des objets} la date effective = la
date à considérer comme référence}

\begin{itemize}
\item
  \ul{dépôt de dépôt} de la demande
\item
  dépôt de dépôt de la demande de \ul{priorité}
\item
  les caractéristiques 123(2) sont considérées à la date de dépôt de la
  demande
\end{itemize}

\begin{longtable}[]{@{}
  >{\raggedright\arraybackslash}p{(\columnwidth - 2\tabcolsep) * \real{0.5294}}
  >{\raggedright\arraybackslash}p{(\columnwidth - 2\tabcolsep) * \real{0.4706}}@{}}
\toprule\noalign{}
\begin{minipage}[b]{\linewidth}\raggedright
R1~=\textgreater{} 1 objet R1
\end{minipage} & \begin{minipage}[b]{\linewidth}\raggedright
DATE
\end{minipage} \\
\midrule\noalign{}
\endhead
\bottomrule\noalign{}
\endlastfoot
R2 selon R1 =\textgreater{} 1 objet R1+R2 & DATE \\
R3 selon R2 =\textgreater{} 1 objet R1+R2+R3 & DATE \\
R4 selon R1 =\textgreater{} 1 objet R1+R4 & DATE \\
R5 selon R1 ou R2 =\textgreater{} 2 objet~: R1+R5 et R1+R2+R5 & DATE \\
\end{longtable}

\textbf{Faire une ligne de temps}

\emph{Au brouillon}

\textbf{Prio =\textgreater{} objets dépôt =\textgreater{} objets}

\textbf{\_\_\_\_\_\_\_\_\_\_\_54(2)\_\_\_\_\_\_\_\_\_\_\_54(3)
dépôt\_\_\_\textbar\_\_\_\_\_\_\_54(3)
pub\_\_\_\_\_\_\textbar\_\_\_\_\_\_\_\_\_\_\_\_\_\_\_\_\_\textgreater{}}

\textbf{Date effective Date effective}

Etats de la technique~:

\begin{itemize}
\item
  divulgations 54(2)
\item
  demande de brevet 54(3)
\end{itemize}

Caractériser les divulgations 54(2)

\begin{itemize}
\item
  Date
\item
  Objet
\item
  Circonstances
\end{itemize}

Caractériser les demandes de brevet 54(3) // L611-3~:

\begin{itemize}
\item
  Date de dépôt
\item
  Date de publication
\item
  Territorialité

  \begin{itemize}
  \item
    Pour un brevet français sont pris en compte~: d'autres demandes de
    brevet français, demandes EP désignant FR, demandes internationales
    désignant EP désignant FR
  \item
    Pour un brevet européen sont pris en compte~: d'autres demandes EP,
    demandes internationales désignant EP
  \end{itemize}
\end{itemize}

\textbf{Analyse de la nouveauté}

\begin{itemize}
\item
  R1 = objet 1
\item
  R1~: A, B, C, (caractérisé en ce que) D, E
\end{itemize}

\emph{Au brouillon}

\begin{longtable}[]{@{}
  >{\raggedright\arraybackslash}p{(\columnwidth - 6\tabcolsep) * \real{0.2499}}
  >{\raggedright\arraybackslash}p{(\columnwidth - 6\tabcolsep) * \real{0.2499}}
  >{\raggedright\arraybackslash}p{(\columnwidth - 6\tabcolsep) * \real{0.2501}}
  >{\raggedright\arraybackslash}p{(\columnwidth - 6\tabcolsep) * \real{0.2501}}@{}}
\toprule\noalign{}
\begin{minipage}[b]{\linewidth}\raggedright
Objet 1
\end{minipage} & \begin{minipage}[b]{\linewidth}\raggedright
D1 opposable
\end{minipage} & \begin{minipage}[b]{\linewidth}\raggedright
D2 opposable
\end{minipage} & \begin{minipage}[b]{\linewidth}\raggedright
D3 opposable
\end{minipage} \\
\midrule\noalign{}
\endhead
\bottomrule\noalign{}
\endlastfoot
A & p5 ligne 23 & & \\
B & x & & \\
C & x & & \\
D & P6 ligne 35 & & \\
E & x & & \\
\end{longtable}

Rédiger sur une double page~:

\begin{itemize}
\item
  D1 divulgue A (p5 ligne 23) et D (p6 ligne 35)
\item
  D1 ne divulgue pas B, C, E
\item
  Conclusion~: objet 1 définit par R1 nouveau par rapport à D1
\end{itemize}

\textbf{Analyse de l'activité inventive}

\begin{itemize}
\item
  \hl{Lire cisaille T305/87}
\item
  Lire directives activité inventive
\item
  Avoir un raisonnement clair, logique, convaincant
\item
  Est-ce qu'un homme du métier \ul{à la date de dépôt} était en mesure
  de combiner D1 et D2~?
\item
  Sélectionner un point de départ~: le document (i) chercher à résoudre
  le même problème technique que l'invention et (ii) le même domaine
  technique, si échoue (iii) il faut choisir le document qui présente le
  plus grand nombre de caractéristiques communes avec l'invention
\item
  Regarder les effets technique. Regarder les effets de synergie entre
  les effets techniques. Ne pas argumenter outre mesure sur la synergie.
\item
  Identifier le problème technique objectif~: \textbf{comment en partant
  de D1, l'homme du métier est en mesure d'obtenir l'effet B, C, E.}
\item
  Définir l'homme du métier~: spécialiste du domaine technique. Définir
  les connaissances générales de l'homme du métier.
\item
  Chercher \ul{le même effet} dans D2
\item
  Identifier la caractéristique liée à l'effet dans D2
\item
  Peut-on transposer cette caractéristique à D1
\end{itemize}

\begin{longtable}[]{@{}
  >{\raggedright\arraybackslash}p{(\columnwidth - 8\tabcolsep) * \real{0.2000}}
  >{\raggedright\arraybackslash}p{(\columnwidth - 8\tabcolsep) * \real{0.2000}}
  >{\raggedright\arraybackslash}p{(\columnwidth - 8\tabcolsep) * \real{0.2000}}
  >{\raggedright\arraybackslash}p{(\columnwidth - 8\tabcolsep) * \real{0.2001}}
  >{\raggedright\arraybackslash}p{(\columnwidth - 8\tabcolsep) * \real{0.2001}}@{}}
\toprule\noalign{}
\begin{minipage}[b]{\linewidth}\raggedright
invention
\end{minipage} & \begin{minipage}[b]{\linewidth}\raggedright
D1
\end{minipage} & \begin{minipage}[b]{\linewidth}\raggedright
D2
\end{minipage} & \begin{minipage}[b]{\linewidth}\raggedright
D3
\end{minipage} & \begin{minipage}[b]{\linewidth}\raggedright
\end{minipage} \\
\midrule\noalign{}
\endhead
\bottomrule\noalign{}
\endlastfoot
A & Reproduit & & & \\
B & Pas reproduit~: Effet B p5 ligne 23 & Reproduit B~:

Effet B' différent de B =\textgreater{} activité inventive & Reproduit
B' différent de B~: effet B =\textgreater{} activité inventive & \\
C & reproduit & & & \\
D & reproduit & & & \\
E & reproduit & & & \\
\end{longtable}

\textbf{1 effet technique ou synergie d'effets technique = 1 PTO}

Position adoptée par le titulaire

\begin{longtable}[]{@{}
  >{\raggedright\arraybackslash}p{(\columnwidth - 8\tabcolsep) * \real{0.2000}}
  >{\raggedright\arraybackslash}p{(\columnwidth - 8\tabcolsep) * \real{0.2000}}
  >{\raggedright\arraybackslash}p{(\columnwidth - 8\tabcolsep) * \real{0.2000}}
  >{\raggedright\arraybackslash}p{(\columnwidth - 8\tabcolsep) * \real{0.2001}}
  >{\raggedright\arraybackslash}p{(\columnwidth - 8\tabcolsep) * \real{0.2001}}@{}}
\toprule\noalign{}
\begin{minipage}[b]{\linewidth}\raggedright
invention
\end{minipage} & \begin{minipage}[b]{\linewidth}\raggedright
D1
\end{minipage} & \begin{minipage}[b]{\linewidth}\raggedright
D2
\end{minipage} & \begin{minipage}[b]{\linewidth}\raggedright
D3
\end{minipage} & \begin{minipage}[b]{\linewidth}\raggedright
\end{minipage} \\
\midrule\noalign{}
\endhead
\bottomrule\noalign{}
\endlastfoot
A & Reproduit & & & \\
B & Pas reproduit~: Effet B p5 ligne 23 & & & \\
C & reproduit & & & \\
D & Pas reproduit~: effet D p5 & & & \\
E & reproduit & & & \\
\end{longtable}

\begin{itemize}
\item
  PTO1 sur l'effet B
\item
  PTO2 sur l'effet D
\item
  Un document par PTO
\end{itemize}

\textbf{SOURCES JURIDIQUES}

\textbf{Article L613-25 CPI}

Le brevet est déclaré nul par décision de justice :

a) Si son objet n\textquotesingle est pas \hl{brevetable} aux termes des
\href{https://www.legifrance.gouv.fr/affichCodeArticle.do?cidTexte=LEGITEXT000006069414\&idArticle=LEGIARTI000006279404\&dateTexte=\&categorieLien=cid}{articles
L. 611-10, L. 611-11 et L. 611-13 à L. 611-19 ;}

b) S\textquotesingle il n\textquotesingle expose pas
l\textquotesingle invention de façon suffisamment \hl{claire et
complète} pour qu\textquotesingle un homme du métier puisse
l\textquotesingle exécuter ;

c) Si \hl{son objet s\textquotesingle étend au-delà du contenu de la
demande telle qu\textquotesingle elle a été déposée} ou, lorsque le
brevet a été délivré sur la base d\textquotesingle une demande
divisionnaire, si son objet s\textquotesingle étend au-delà du contenu
de la demande initiale telle qu\textquotesingle elle a été déposée ;

d) Si, après limitation ou opposition, \hl{l\textquotesingle étendue de
la protection conférée par le brevet a été accrue.}

Si les motifs de nullité n\textquotesingle affectent le brevet
qu\textquotesingle en partie, la nullité est prononcée sous la forme
d\textquotesingle une limitation correspondante des revendications.

Dans le cadre d\textquotesingle une action en nullité du brevet, son
titulaire est habilité à limiter le brevet en modifiant les
revendications ; le brevet ainsi limité constitue
l\textquotesingle objet de l\textquotesingle action en nullité engagée.

La partie qui, lors d\textquotesingle une même instance, procède à
plusieurs limitations de son brevet, de manière dilatoire ou abusive,
peut être condamnée à une amende civile d\textquotesingle un montant
maximum de 3 000 euros, sans préjudice de dommages et intérêts qui
seraient réclamés.

\textbf{Article L613-23-1 CPI}

L\textquotesingle opposition ne peut être fondée que sur un ou plusieurs
des motifs suivants :\\
\strut \\
1° L\textquotesingle objet du brevet n\textquotesingle est pas
brevetable aux termes des articles L. 611-10, L. 611-11 et L. 611-13 à
L. 611-19 ;\\
\strut \\
2° Le brevet n\textquotesingle expose pas l\textquotesingle invention de
façon suffisamment claire et complète pour qu\textquotesingle un homme
du métier puisse l\textquotesingle exécuter ;\\
\strut \\
3° L\textquotesingle objet du brevet s\textquotesingle étend au-delà du
contenu de la demande telle qu\textquotesingle elle a été déposée ou,
lorsque le brevet a été délivré sur la base d\textquotesingle une
demande divisionnaire, l\textquotesingle objet s\textquotesingle étend
au-delà du contenu de la demande initiale telle qu\textquotesingle elle
a été déposée.\\
\strut \\
L\textquotesingle opposition peut porter sur tout ou partie du brevet
délivré.

\textbf{Article L614-12 CPI}

La nullité du brevet européen est prononcée en ce qui concerne la France
par décision de justice pour l\textquotesingle un quelconque des motifs
visés à \hl{l\textquotesingle article 138, paragraphe 1}, de la
Convention de Munich.

Si les motifs de nullité n\textquotesingle affectent le brevet
qu\textquotesingle en partie, la nullité est prononcée sous la forme
d\textquotesingle une limitation correspondante des revendications.

Dans le cadre d\textquotesingle une action en nullité du brevet
européen, son titulaire est habilité à limiter le brevet en modifiant
les revendications conformément à l\textquotesingle article 105 bis de
la convention de Munich ; le brevet ainsi limité constitue
l\textquotesingle objet de l\textquotesingle action en nullité
engagée.\\
\strut \\
La partie qui, lors d\textquotesingle une même instance, procède à
plusieurs limitations de son brevet de manière dilatoire ou abusive peut
être condamnée à une amende civile d\textquotesingle un montant maximum
de 3 000 euros, sans préjudice de dommages et intérêts qui seraient
réclamés.

\textbf{Article 138(1) CBE}

Sous réserve de
\href{https://www.epo.org/fr/legal/epc/2020/a139.html}{l\textquotesingle article~139},
le brevet européen ne peut être déclaré nul, avec effet pour un État
contractant, que si~:

a) l\textquotesingle objet du brevet européen n\textquotesingle est
\hl{pas brevetable en vertu des
\href{https://www.epo.org/fr/legal/epc/2020/a52.html}{articles~52} à
\href{https://www.epo.org/fr/legal/epc/2020/a57.html}{57}~;}

b) le brevet européen n\textquotesingle expose pas
l\textquotesingle invention de façon \hl{suffisamment claire et
complète} pour qu\textquotesingle un homme du métier puisse
l\textquotesingle exécuter~;~

c)\href{https://www.epo.org/fr/legal/epc/2020/a138.html\#conv.f171-note}{\textsuperscript{171}}
\hl{l\textquotesingle objet du brevet européen s\textquotesingle étend
au-delà du contenu de la demande telle qu\textquotesingle elle a été
déposée} ou, lorsque le brevet a été délivré sur la base
d\textquotesingle une demande divisionnaire ou d\textquotesingle une
nouvelle demande déposée en vertu de
\href{https://www.epo.org/fr/legal/epc/2020/a61.html}{l\textquotesingle article~61},
si l\textquotesingle objet du brevet s\textquotesingle étend au-delà du
contenu de la demande antérieure telle qu\textquotesingle elle a été
déposée~;

d) \hl{la protection conférée par le brevet européen a été étendue}~;
ou~

\hl{e) le titulaire du brevet européen n\textquotesingle avait pas le
droit de l\textquotesingle obtenir en vertu de
\href{https://www.epo.org/fr/legal/epc/2020/a60.html}{l\textquotesingle article~60,
paragraphe~1}.}

\textbf{Article 65 AJUB}

\textbf{Article~65\\
Décision sur la validité d\textquotesingle un brevet}

1.La Juridiction statue sur la validité d\textquotesingle un brevet sur
la base d\textquotesingle une action en nullité ou d\textquotesingle une
demande reconventionnelle en nullité.~

2.La Juridiction ne peut annuler un brevet, en tout ou en partie,
\hl{que pour les motifs visés à
\href{https://www.epo.org/fr/legal/epc/2020/a138.html}{l\textquotesingle article~138,
paragraphe 1}, et à
\href{https://www.epo.org/fr/legal/epc/2020/a139.html}{l\textquotesingle article~139,
paragraphe 2, de la CBE}.}

3.Sans préjudice de
\href{https://www.epo.org/fr/legal/epc/2020/a138.html}{l\textquotesingle article~138,
paragraphe 3, de la CBE}, si les motifs de nullité ne visent le brevet
que partiellement, le brevet est limité par une modification
correspondante des revendications et est annulé en partie.

4.Dans la mesure où un brevet a été annulé, il est réputé avoir été,
d\textquotesingle emblée, dépourvu des effets précisés aux
\href{https://www.epo.org/fr/legal/epc/2020/a64.html}{articles 64} et
\href{https://www.epo.org/fr/legal/epc/2020/a67.html}{67 de la CBE}.

5.Lorsque la Juridiction, dans une décision définitive, annule un brevet
en tout ou en partie, elle transmet une copie de la décision à
l\textquotesingle Office européen des brevets et, s\textquotesingle il
s\textquotesingle agit d\textquotesingle un brevet européen, à
l\textquotesingle office national des brevets de tout Etat membre
contractant concerné.~

\textbf{Article 100 CBE\\
Motifs d'opposition}

L\textquotesingle opposition ne peut être fondée que sur les motifs
suivants~:

a) l\textquotesingle objet du brevet européen n\textquotesingle est
\hl{pas brevetable en vertu des
\href{https://www.epo.org/fr/legal/epc/2020/a52.html}{articles~52} à
\href{https://www.epo.org/fr/legal/epc/2020/a57.html}{57}~;}

b) le brevet européen n\textquotesingle expose pas
l\textquotesingle invention de façon \hl{suffisamment claire et
complète} pour qu\textquotesingle un homme du métier puisse
l\textquotesingle exécuter~;~

c) \hl{l\textquotesingle objet du brevet européen
s\textquotesingle étend au-delà du contenu de la demande telle
qu\textquotesingle elle a été déposée} ou, si le brevet a été délivré
sur la base d\textquotesingle une demande divisionnaire ou
d\textquotesingle une nouvelle demande déposée en vertu de
\href{https://www.epo.org/fr/legal/epc/2020/a61.html}{l\textquotesingle article~61},
au-delà du contenu de la demande antérieure telle
qu\textquotesingle elle a été déposée.

\end{document}
