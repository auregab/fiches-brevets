% Options for packages loaded elsewhere
\PassOptionsToPackage{unicode}{hyperref}
\PassOptionsToPackage{hyphens}{url}
%
\documentclass[
]{article}
\usepackage{amsmath,amssymb}
\usepackage{iftex}
\ifPDFTeX
  \usepackage[T1]{fontenc}
  \usepackage[utf8]{inputenc}
  \usepackage{textcomp} % provide euro and other symbols
\else % if luatex or xetex
  \usepackage{unicode-math} % this also loads fontspec
  \defaultfontfeatures{Scale=MatchLowercase}
  \defaultfontfeatures[\rmfamily]{Ligatures=TeX,Scale=1}
\fi
\usepackage{lmodern}
\ifPDFTeX\else
  % xetex/luatex font selection
\fi
% Use upquote if available, for straight quotes in verbatim environments
\IfFileExists{upquote.sty}{\usepackage{upquote}}{}
\IfFileExists{microtype.sty}{% use microtype if available
  \usepackage[]{microtype}
  \UseMicrotypeSet[protrusion]{basicmath} % disable protrusion for tt fonts
}{}
\makeatletter
\@ifundefined{KOMAClassName}{% if non-KOMA class
  \IfFileExists{parskip.sty}{%
    \usepackage{parskip}
  }{% else
    \setlength{\parindent}{0pt}
    \setlength{\parskip}{6pt plus 2pt minus 1pt}}
}{% if KOMA class
  \KOMAoptions{parskip=half}}
\makeatother
\usepackage{xcolor}
\usepackage{longtable,booktabs,array}
\usepackage{calc} % for calculating minipage widths
% Correct order of tables after \paragraph or \subparagraph
\usepackage{etoolbox}
\makeatletter
\patchcmd\longtable{\par}{\if@noskipsec\mbox{}\fi\par}{}{}
\makeatother
% Allow footnotes in longtable head/foot
\IfFileExists{footnotehyper.sty}{\usepackage{footnotehyper}}{\usepackage{footnote}}
\makesavenoteenv{longtable}
\usepackage{graphicx}
\makeatletter
\def\maxwidth{\ifdim\Gin@nat@width>\linewidth\linewidth\else\Gin@nat@width\fi}
\def\maxheight{\ifdim\Gin@nat@height>\textheight\textheight\else\Gin@nat@height\fi}
\makeatother
% Scale images if necessary, so that they will not overflow the page
% margins by default, and it is still possible to overwrite the defaults
% using explicit options in \includegraphics[width, height, ...]{}
\setkeys{Gin}{width=\maxwidth,height=\maxheight,keepaspectratio}
% Set default figure placement to htbp
\makeatletter
\def\fps@figure{htbp}
\makeatother
\ifLuaTeX
  \usepackage{luacolor}
  \usepackage[soul]{lua-ul}
\else
  \usepackage{soul}
\fi
\setlength{\emergencystretch}{3em} % prevent overfull lines
\providecommand{\tightlist}{%
  \setlength{\itemsep}{0pt}\setlength{\parskip}{0pt}}
\setcounter{secnumdepth}{-\maxdimen} % remove section numbering
\ifLuaTeX
  \usepackage{selnolig}  % disable illegal ligatures
\fi
\IfFileExists{bookmark.sty}{\usepackage{bookmark}}{\usepackage{hyperref}}
\IfFileExists{xurl.sty}{\usepackage{xurl}}{} % add URL line breaks if available
\urlstyle{same}
\hypersetup{
  hidelinks,
  pdfcreator={LaTeX via pandoc}}

\author{}
\date{}

\begin{document}

\includegraphics[width=4.24201in,height=1.53073in]{media/image1.png}

\textbf{\hl{Article 1832 du Code Civil}} «~\hl{\textbf{La société}
(100)} est \hl{instituée (110)} par deux ou plusieurs personnes qui
conviennent par \hl{un contrat (121, 122, 123)}
\hl{d\textquotesingle affecter (111)} à \hl{une entreprise (200)}
\hl{commune des biens ou leur industrie (112)} en vue de \hl{partager le
bénéfice (113)} ou de profiter de l\textquotesingle économie qui pourra
en résulter.\\
Elle peut être instituée, dans les cas prévus par la loi, par
l\textquotesingle acte de volonté d\textquotesingle une seule
personne.\\
Les associés s'engagent à contribuer aux pertes.\textbf{~»}

distinguer~:\\
\hl{(100) société} = notion juridique~;\\
\hl{(200) entreprise} = notion économique~;\\
\hl{(300) fonds de commerce} = une universalité exploitée par un
commerçant ou une société (par exemple, bail sur locaux, clientèle)~;\\
\hl{(400) personne morale} = construction juridique permettant à
l'entité d'être sujet de droit à part entière, de disposer d'un
patrimoine, \hl{la personne morale (410, 420)} nait par immatriculation
au RNE \hl{{[}base~: l'article 1842 du Code Civil{]}.}

une société est à la fois~:\\
\hl{(110) une institution} =\textgreater{} existence de règles
impératives restreignant la liberté contractuelle~;\\
\hl{(121, 122, 123) un contrat} =\textgreater{} principe de liberté
contractuelle mais peu de règles du droit commun s'appliquent.

conditions générales de validité du contrat de société~:\\
\hl{(121) consentement} = créer la société par signature des statuts et
à en être associé par apports~;\\
\hl{(122) capacité} = au moins avoir une capacité juridique, voire une
capacité commerciale (par exemple pour une SNC)~;\\
\hl{(123) contenu} = objet social =/= cause sociale =/= activité
sociale, objet social statutaire =/= objet social réel.\\
\strut \\
conditions spéciales de validité du contrat de société\\
\hl{(111) affectio societatis}, le cas échéant en crise grave
=\textgreater{} dissolution de la société~;\\
\hl{(112) apports} = numéraire, en nature et/ou en industrie, forme en
somme le capital social =/= capital économique \hl{{[}base~: article
1843-3 du Code Civil{]}~};\\
\hl{(113) participation aux résultats}, positifs ou négatifs, de manière
proportionnelle aux apports dans la limite d'une clause léonine
\hl{{[}base~: article 1844-1 du Code Civil{]}}.\\
\strut \\
\strut \\
\textbf{\hl{Article 1835 du Code Civil}} «~\textbf{Les statuts} doivent
être établis \textbf{par écrit (i)}. Ils déterminent, outre les apports
de chaque associé, la forme, l\textquotesingle objet,
\textbf{l\textquotesingle appellation {[}dénomination sociale{]}},
\textbf{le siège social}, le capital social, la durée de la société et
\textbf{les modalités de son fonctionnement (ii)}. Les statuts peuvent
préciser \textbf{une raison d\textquotesingle être {[}option loi pacte
(iii){]}} {[}\ldots{]}~».

\textbf{\hl{Article 1844-10 du Code Civil} «~La nullité} de la société
ne peut résulter que de la violation des dispositions \textbf{\hl{des
articles 1832 {[}conditions spéciales de validité (i){]}}}, 1832-1,
alinéa 1\textsuperscript{er}, et \textbf{1833 {[}illicéité de l'objet
social (ii){]}}, ou de \textbf{\hl{l\textquotesingle une des causes de
nullité des contrats en général {[}conditions générales de validité
(iii){]}}}~».

\textbf{\hl{Article 1844-15 du Code Civil}} «~Lorsque la nullité de la
société est prononcée, elle met fin, \textbf{sans rétroactivité}, à
l\textquotesingle exécution du contrat~».

\textbf{\hl{Article 1843 du Code Civil}} «~Les personnes qui ont agi au
nom d\textquotesingle une \textbf{société en formation avant
l\textquotesingle immatriculation} sont tenues des obligations nées des
actes ainsi accomplis, avec solidarité si la société est commerciale,
sans solidarité dans les autres cas. \textbf{La société régulièrement
immatriculée peut reprendre} \textbf{{[}soit par annexion au statuts des
contrats, soit par reprise balai par assemblé générale
extraordinaire{]}} les engagements souscrits, qui sont alors réputés
avoir été dès l\textquotesingle origine contractés par celle-ci.~»

\textbf{\hl{Article 1837 du Code Civil}} «~Toute société dont le
\textbf{siège} est situé sur le territoire français est soumise
\textbf{aux dispositions de la loi française {[}nationalité de la
société{]}}.\\
Les tiers peuvent se prévaloir du \textbf{siège statutaire}, mais
celui-ci ne leur est pas opposable par la société si \textbf{le siège
réel} est situé en un autre lieu \textbf{{[}distinction siège social =/=
siège réel{]}}.~»

\textbf{\hl{Article 121-2 du Code pénal}} «~Les \hl{personnes morales
(410)}, à l\textquotesingle exclusion de l\textquotesingle Etat,
\textbf{sont responsables pénalement}, selon les distinctions des
\href{https://www.legifrance.gouv.fr/affichCodeArticle.do?cidTexte=LEGITEXT000006070719\&idArticle=LEGIARTI000006417209\&dateTexte=\&categorieLien=cid}{articles
121-4 à 121-7}, des infractions commises, pour leur compte, par leurs
organes ou représentants. {[}\ldots{]}~»

\textbf{\hl{Article 131-38 du Code pénal}} «~Le taux maximum de
l\textquotesingle amende applicable aux \hl{personnes morales (411)} est
égal au \textbf{quintuple} de celui prévu pour les personnes physiques
par la loi qui réprime l\textquotesingle infraction.~»

\textbf{\hfill\break
}

\emph{\textbf{Tableau de classification des sociétés}}

\begin{longtable}[]{@{}
  >{\raggedright\arraybackslash}p{(\columnwidth - 8\tabcolsep) * \real{0.1637}}
  >{\raggedright\arraybackslash}p{(\columnwidth - 8\tabcolsep) * \real{0.2210}}
  >{\raggedright\arraybackslash}p{(\columnwidth - 8\tabcolsep) * \real{0.2184}}
  >{\raggedright\arraybackslash}p{(\columnwidth - 8\tabcolsep) * \real{0.2025}}
  >{\raggedright\arraybackslash}p{(\columnwidth - 8\tabcolsep) * \real{0.1945}}@{}}
\toprule\noalign{}
\begin{minipage}[b]{\linewidth}\raggedright
\textbf{Catégorie}
\end{minipage} & \begin{minipage}[b]{\linewidth}\raggedright
\textbf{Sous-catégorie}
\end{minipage} & \begin{minipage}[b]{\linewidth}\raggedright
\textbf{Critère / Principe}
\end{minipage} & \begin{minipage}[b]{\linewidth}\raggedright
\textbf{Exemples / Observations}
\end{minipage} & \begin{minipage}[b]{\linewidth}\raggedright
\textbf{Base légale}
\end{minipage} \\
\midrule\noalign{}
\endhead
\bottomrule\noalign{}
\endlastfoot
\textbf{Nature de la société} & Société commerciale (par
l\textquotesingle objet) & Objet social est commercial & Achat/revente,
activités industrielles & Livre 2 du Code de commerce \\
& Société commerciale (par la forme) & Commerciale même si objet civil &
SA, SAS, SARL, SNC, SCS & \hl{Article L210-1 C. com.} \\
& Société civile (par l\textquotesingle objet) & Objet civil :
immobilier, prof. libéral & SCI, SCP, SCM & Code Civil \\
& Société civile (obligatoire) & Activité nécessairement civile &
Professions libérales, agriculteurs & \hl{Article 1845 et s. C. civ.} \\
\textbf{Nature du lien} & Société de personnes & Intuitu personae fort,
responsabilité illimitée & SNC, SCS, sociétés civiles & Jurisprudence \\
& Société de capitaux & Capital \textgreater{} personne, responsabilité
limitée & SA, SAS & --- \\
& Société mixte & Intuitu personae + responsabilité limitée & SARL &
--- \\
\textbf{Risque des associés} & Société à risque limité & Responsables à
hauteur des apports & SA, SAS, SARL, EURL, SASU & Principes généraux \\
& Société à risque illimité & Engagés sur leur patrimoine personnel &
SNC, sociétés civiles & --- \\
\textbf{Cotation} & Société cotée & Actions proposées au public & SA
cotée, parfois SAS & Code monétaire et financier \\
& Société non cotée & Pas d\textquotesingle appel public à
l\textquotesingle épargne & Majorité des sociétés & --- \\
\textbf{Nombre d'associés} & Société pluripersonnelle & Plusieurs
associés & SA, SAS, SARL & \hl{Article 1832 al. 1 C. civ.} \\
& Société unipersonnelle & Un seul associé & EURL, SASU & \hl{Art. 1832
al. 2 C. civ. ; Loi 85-697} \\
\end{longtable}

\textbf{\hfill\break
}

\begin{longtable}[]{@{}
  >{\raggedright\arraybackslash}p{(\columnwidth - 2\tabcolsep) * \real{0.5080}}
  >{\raggedright\arraybackslash}p{(\columnwidth - 2\tabcolsep) * \real{0.4920}}@{}}
\toprule\noalign{}
\begin{minipage}[b]{\linewidth}\raggedright
ACTIF
\end{minipage} & \begin{minipage}[b]{\linewidth}\raggedright
PASSIF
\end{minipage} \\
\midrule\noalign{}
\endhead
\bottomrule\noalign{}
\endlastfoot
Définition : Ensemble des biens, droits et ressources appartenant à la
société. & Définition : Ensemble des dettes et obligations de la société
envers des tiers. \\
Actifs immobilisés : corporels (locaux, machines), incorporels (brevets,
logiciels), financiers (titres, prêts). & Passif à long terme : dettes
dont l'échéance dépasse un an (emprunts, dettes fournisseurs long
terme). \\
Actifs circulants : stocks, créances clients, trésorerie, placements
financiers court terme. & Passif à court terme : dettes exigibles à
court terme (charges sociales, dettes fiscales, fournisseurs). \\
Autres actifs : droits futurs (promesses, contrats), crédits d'impôt,
provisions pour charges/risques. & \\
Actifs hors bilan : contrats de location, sûretés ou garanties
(hypothèques, cautions). & \\
\end{longtable}

\end{document}
