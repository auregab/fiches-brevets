% Options for packages loaded elsewhere
\PassOptionsToPackage{unicode}{hyperref}
\PassOptionsToPackage{hyphens}{url}
%
\documentclass[
]{article}
\usepackage{amsmath,amssymb}
\usepackage{iftex}
\ifPDFTeX
  \usepackage[T1]{fontenc}
  \usepackage[utf8]{inputenc}
  \usepackage{textcomp} % provide euro and other symbols
\else % if luatex or xetex
  \usepackage{unicode-math} % this also loads fontspec
  \defaultfontfeatures{Scale=MatchLowercase}
  \defaultfontfeatures[\rmfamily]{Ligatures=TeX,Scale=1}
\fi
\usepackage{lmodern}
\ifPDFTeX\else
  % xetex/luatex font selection
\fi
% Use upquote if available, for straight quotes in verbatim environments
\IfFileExists{upquote.sty}{\usepackage{upquote}}{}
\IfFileExists{microtype.sty}{% use microtype if available
  \usepackage[]{microtype}
  \UseMicrotypeSet[protrusion]{basicmath} % disable protrusion for tt fonts
}{}
\makeatletter
\@ifundefined{KOMAClassName}{% if non-KOMA class
  \IfFileExists{parskip.sty}{%
    \usepackage{parskip}
  }{% else
    \setlength{\parindent}{0pt}
    \setlength{\parskip}{6pt plus 2pt minus 1pt}}
}{% if KOMA class
  \KOMAoptions{parskip=half}}
\makeatother
\usepackage{xcolor}
\usepackage{longtable,booktabs,array}
\usepackage{multirow}
\usepackage{calc} % for calculating minipage widths
% Correct order of tables after \paragraph or \subparagraph
\usepackage{etoolbox}
\makeatletter
\patchcmd\longtable{\par}{\if@noskipsec\mbox{}\fi\par}{}{}
\makeatother
% Allow footnotes in longtable head/foot
\IfFileExists{footnotehyper.sty}{\usepackage{footnotehyper}}{\usepackage{footnote}}
\makesavenoteenv{longtable}
\ifLuaTeX
  \usepackage{luacolor}
  \usepackage[soul]{lua-ul}
\else
  \usepackage{soul}
\fi
\setlength{\emergencystretch}{3em} % prevent overfull lines
\providecommand{\tightlist}{%
  \setlength{\itemsep}{0pt}\setlength{\parskip}{0pt}}
\setcounter{secnumdepth}{-\maxdimen} % remove section numbering
\ifLuaTeX
  \usepackage{selnolig}  % disable illegal ligatures
\fi
\IfFileExists{bookmark.sty}{\usepackage{bookmark}}{\usepackage{hyperref}}
\IfFileExists{xurl.sty}{\usepackage{xurl}}{} % add URL line breaks if available
\urlstyle{same}
\hypersetup{
  hidelinks,
  pdfcreator={LaTeX via pandoc}}

\author{}
\date{}

\begin{document}

\begin{enumerate}
\def\labelenumi{\arabic{enumi}.}
\item
  \textbf{\ul{VERIFIER UNE PRIORITE}}
\end{enumerate}

\begin{longtable}[]{@{}
  >{\raggedright\arraybackslash}p{(\columnwidth - 4\tabcolsep) * \real{0.2345}}
  >{\raggedright\arraybackslash}p{(\columnwidth - 4\tabcolsep) * \real{0.5782}}
  >{\raggedright\arraybackslash}p{(\columnwidth - 4\tabcolsep) * \real{0.1874}}@{}}
\toprule\noalign{}
\begin{minipage}[b]{\linewidth}\raggedright
\textbf{(I) Celui qui / son ayant cause~}
\end{minipage} & \begin{minipage}[b]{\linewidth}\raggedright
\textbf{Preuve~?}
\end{minipage} & \begin{minipage}[b]{\linewidth}\raggedright
\textbf{A87(1) CBE}
\end{minipage} \\
\midrule\noalign{}
\endhead
\bottomrule\noalign{}
\endlastfoot
Prio. valable à l'OEB & - présomption réfragable du droit de priorité &
G1/22, G2/22 \\
\multirow{2}{*}{Prio. Valable à l'INPI} & - une «~copie de la demande
antérieure~» & \multirow{2}{*}{Cass arrêt n° 537 FS-B} \\
& - une «~autorisation de revendiquer la priorité donnée par écrit par
le propriétaire de la demande antérieure~» \\
\textbf{(II) Rég.déposé dans ou pour} & \textbf{Etat partie à l'OMC~?} &
\textbf{A87(1) CBE} \\
Prio. pas valable à l'OEB & - avant 13/12/2007 & G2/02, G3/02 \\
Prio. Valable à l'OEB & - après 13/12/2007 (inclus) & CBE 2000 \\
& \textbf{Etat accédant à CUP ou OMC~?} & \\
\multirow{2}{*}{Prio valable à l'OEB} & - accord en vigueur à la date de
la priorité, \ul{ET} & \\
& - législation applicable à la date de dépôt & \\
\textbf{(III) La même inv.} & \textbf{Fondement~?} & \textbf{A87(1)
CBE} \\
Prio.

valable & (i) allouer des dates de priorité à l'intérieur d'une
revendication de la demande sous priorité & G1/15 \\
& - «~suffisamment décrit~» dans la demande antérieure & \\
& (ii) DDSA de l'ensemble des pièces de la demande antérieure & G2/98 \\
Prio pas valable & - combiner des documents sans lien entre eux \emph{à
la manière d'une mosaïque} & Directives F-VI, 1.5 \\
& - un état de la technique cité dans la description de la demande
antérieure & Directives F-VI, 2.2 \\
& - un disclaimer décrit dans la description de la demande antérieure
& \\
\textbf{(IV) Délai de 12 mois} & \textbf{Calcul~?} & \textbf{A87(1)
CBE} \\
& (i) appliquer la règle 131(4) pour le calcul du délai & R131(4) \\
& (ii) appliquer la règle 134(1) à (4) pour les prorogations de délai &
R134(1)-(4) \\
& (iii) appliquer la règle 133 pour la fiction de respect du délai pour
les pièces reçues tardivement, la règle 134(5) pour la fiction de
respect pour les envois postaux & R133 \\
\textbf{(V) première dem.} & \textbf{Cf. 2.} & \\
\end{longtable}

\begin{enumerate}
\def\labelenumi{\arabic{enumi}.}
\setcounter{enumi}{1}
\item
  \textbf{\ul{VERIFIER UNE PREMIERE DEMANDE ET EXCEPTION}}
\end{enumerate}

\begin{longtable}[]{@{}
  >{\raggedright\arraybackslash}p{(\columnwidth - 4\tabcolsep) * \real{0.2345}}
  >{\raggedright\arraybackslash}p{(\columnwidth - 4\tabcolsep) * \real{0.5782}}
  >{\raggedright\arraybackslash}p{(\columnwidth - 4\tabcolsep) * \real{0.1874}}@{}}
\toprule\noalign{}
\begin{minipage}[b]{\linewidth}\raggedright
\textbf{(V) Première dem.}
\end{minipage} & \begin{minipage}[b]{\linewidth}\raggedright
\textbf{Principe~?}
\end{minipage} & \begin{minipage}[b]{\linewidth}\raggedright
\textbf{A87(1) CBE}
\end{minipage} \\
\midrule\noalign{}
\endhead
\bottomrule\noalign{}
\endlastfoot
== 1ere dem. & - vérifier qu'il n'existe pas une demande antérieure à la
date de priorité qui vérifie l'ensemble des critères de A87(1) & \\
& \textbf{Même déposant~?} & \\
== 1ere dem. & - une demande déposée par \ul{B} après une demande
déposée par \ul{A et B} & T788/05 \\
& - une demande déposée par un \ul{demandeur A} après une demande
déposée par un \ul{demandeur B} & T5/05 \\
& \textbf{Dans ou pour un même Etat~?} & \\
== 1ere dem. & - une demande déposée à \ul{un état partie à la CUP ou
OMC} après une demande déposée à \ul{un état pas partie à la CUP ou OMC}
& \\
& \textbf{Même invention~?} & \\
== 1ere dem. & - une demande généralisant des exemples déposée après une
demande comprenant des exemples (mais pas une 1ere demande pour les
exemples spécifiques) & T1222/11, G1/15 \\
& - une demande recouvrant une plage de valeurs plus large que celle
d'une demande déposée avant, seulement pour la partie nouvelle de la
plage de valeur & T282/12 \\
=/= 1ere dem. & - une demande non provisoire déposée après une demande
provisoire & \\
& - une demande de continuation déposée après une demande US & \\
\textbf{Dem. ultérieure} & \textbf{(i) Même invention~?} &
\textbf{A87(4) CBE} \\
\textless==\textgreater{} 1ere dem. & - cf. A87(1) & \\
& \textbf{(ii) Dans ou pour même Etat~?} & \textbf{A87(4) CBE} \\
\textless==\textgreater{} 1ere dem. & - au moins un Etat contractant
commun désigné & A79(1) CBE, règle 4.9a PCT \\
& \textbf{(iii) Dem. ant. retirée, abandonnée ou refusée~?} &
\textbf{A87(4) CBE} \\
\textless==\textgreater{} 1ere dem. & - apprécié à la date de dépôt
(incluse) de la demande ultérieure & \\
& - la fiction de retrait suffit (réputé retiré) & \\
& - la fiction de retrait prend effet à l'expiration du délai non
respecté & G4/98 motifs 3.3 et 7.2 \\
& \textbf{(iv) Dem. ant. non soumise à l'inspection publique~?} &
\textbf{A87(4) CBE} \\
\textless==\textgreater{} 1ere dem. & - apprécié à la date de dépôt de
la demande ultérieure & \\
\textless=/=\textgreater{} 1ere dem. & - suite à une notification de la
demande de brevet non encore plus publiée & L615-4 CPI \\
& - consultation d'un tier du dossier EP lorsque le demandeur s'en est
prévalu à son encontre & A128(2) \\
& \textbf{(v) dem. ant. n'ayant pas laissé subsister de droits~?} &
\textbf{A87(4) CBE} \\
\textless==\textgreater{} 1ere dem. & - apprécié à la date de dépôt de
la dem. ultérieure & \\
& - subsiste une voie de recours pour la dem. ant. & \\
\textless==\textgreater{} 1ere dem. & - subsiste le bénéfice de la date
de dépôt de la dem. ant. en FR & L612-3 CPI, R 612-25 \\
& -\/- subsiste la \emph{continuation} ou la \emph{continuation-in-part}
pour la dem. ant. en US & 35 U.S.C§120 \\
& - subsiste le bénéfice de la date de dépôt de la dem. ant. en EP
(divisionnaire) & A76 CBE \\
& - rétroactivement lorsqu'une voie de recours est exercée pour la dem.
antérieure & \\
& \textbf{(vi) dem. ant. n'ayant servi de base pour la rev. du droit de
priorité~?} & \textbf{A87(4) CBE} \\
\textless==\textgreater{} 1ere dem. & - apprécié à la date de dépôt de
la dem. ultérieure & \\
& \textbf{Conditions cumulatives~?} & \textbf{A87(4) CBE} \\
\textless=/=\textgreater{} 1ere dem. & - une demande supplémentaire
après une demande ultérieure déposée après une demande antérieure, la
demande ultérieure étant retirée & Mathely p.590 \\
& - une demande ultérieure déposée après une demande intermédiaire
déposée après une demande antérieure & Bodenhausen p.48 \\
& \textbf{Interdiction~?} & \\
S'applique & - rev. priorité de la dem. ant. lorsque dem. ant. XX = état
partie à la CUP ou l'OMC et dem. sous prio. YY = EP & \textbf{A87(4)
CBE} \\
& - rev. priorité de la dem. ant. lorsque dem. ant. XX = état partie à
la CUP et dem. sous prio. YY = état partie à la CUP & A4C4 CUP \\
& - rev. priorité de la dem. ant. lorsque dem. ant. XX = état partie à
l'OMC et dem. sous prio. YY = état partie à l'OMC, XX ou YY
=\textbackslash= CUP & ADPIC renvoi à CUP \\
Ne s'applique pas & - à objet B lorsque dem. ant. porte sur A et B, dem.
ult porte sur A & \\
& - à une FR3 déposée après FR1, FR2 et EP, EP sous prio FR2, FR1
retiré, FR3 rev. bénéfice de la date de dépôt de FR1 & L612-3 CPI \\
\end{longtable}

\end{document}
