\section{Conditions de responsabilité}

\begin{longtblr}{colspec = {X[3,l] X[8,l] X[3,l]}}

% L613-3 L613-4 A25 A26 AJUB
\hline
\textbf{Contrefaçon directe de produit} &
&
L615-1\\

principe &
-- exonérer le revendeur du coin de la rue &
\\

&
-- prétexte pour éviter l'accusation de dénigrement
lorsque mise en connaissance de cause &
\\

&
\textbf{Qui?} &
\\

\ok{bénéficie de la condition} &
-- une autre personne que le fabricant du produit contrefaisant &
\\

\nok{ne bénéficie pas de la condition} &
-- le fabricant du produit contrefaisant &
\\


&
\textbf{Condition de responsabilité?} &
\\

\ok{la connaissance de cause} &
-- \action{l'offre} d'un produit contrefaisant ou produit obtenu directement
par un procédé contrefaisant &
L615-1 al. 3\\

&
-- \action{la mise dans le commerce} d'un produit contrefaisant ou produit obtenu directement
par un procédé contrefaisant &
\\

&
-- \action{l'utilisation} d'un produit contrefaisant ou produit obtenu directement
par un procédé contrefaisant &
\\

&
-- \action{la détention en vue de l'offre, de la mise dans le commerce,
l'utilisation} d'un produit contrefaisant ou produit obtenu directement
par un procédé contrefaisant &
\\

\nok{pas de connaissance de cause} &
-- \action{la fabrication} d'un produit de contrefaisant &
\\

&
-- \action{l'importation} d'un produit de contrefaisant ou produit obtenu directement
par un procédé contrefaisant &
\\

&
-- \action{l'exportation} d'un produit de contrefaisant ou produit obtenu directement
par un procédé contrefaisant &
\\

&
-- \action{le transbordement} d'un produit de contrefaisant ou produit obtenu directement
par un procédé contrefaisant &
\\

&
-- \action{la détention en vue de l'importation, l'exportation,
le transbordement} ou produit obtenu directement
par un procédé contrefaisant &
\\

&
\textbf{Mise en connaissance?} &
\\

&
-- lettre $\in$ revendication du brevet et produit &
\\

&
-- ne pas envoyer la lettre à des tiers pour les informer
d'une possible contrefaçon $\Rightarrow$ dénigrement &
Pourvoi n°R24-11.150 \\

&
-- la connaissance de cause est présumée lorsque le domaine
technique est restreint &
(?) \\

&
-- les faits ont été commis en connaissance de cause &
L615-1 al.3 \\

\hline \textbf{Contrefaçon directe d'un procédé} &
\textbf{Condition de responsabilité?} &
L613-3(b)\\

\ok{la connaissance de cause} &
-- l'offre de l'utilisation d'un procédé objet du brevet
sur le territoire français &
\\

\nok{pas de connaissance de cause} &
-- l'utilisation d'un procédé objet du brevet &
\\

&
\textbf{Mise en connaissance?} &
\\

&
-- le tier sait ou les circonstances rendent évident &
\\

\hline \textbf{Contrefaçon indirecte} &
\textbf{Conditions de responsabilité?} &
L613-4\\

\ok{connaissance de cause} &
-- les moyens sont aptes et destinés à la mise en oeuvre
de l'invention &
\\

&
\textbf{Mise en connaissance?} &
\\

&
-- le tier sait ou les circonstances rendent évident &
\\

\end{longtblr}