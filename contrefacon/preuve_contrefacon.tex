\section{La preuve de la contrefaçon}

\begin{longtblr}{colspec = {X[3,l] X[8,l] X[3,l]}}

\hline
\textbf{Généralités} &
\textbf{Preuve?} &
\\

principe &
-- par tous moyens &
L615-5\\

\ok{grande force probante} &
-- constat d'huissier: constat internet, constat d'achat &
A145 CPC\\

&
-- constat d'huissier par un stagiaire pour un constant d'achat &
Cass (?)\\

\nok{faible force probante} &
-- témoignage en France & 
\\

\hline
\textbf{Saisie contrefaçon} &
\textbf{Procédure sur requête?} &
\\

principe &
-- procédure non contradictoire &
\\

&
-- caractère exécutoire: applicable immédiatement &
A489 CPC\\

\ok{recevable} &
-- \action{requête $\in$ copie conforme du brevet, état des annuités, 
inscriptions, projet d'ordonnance} &
\\

\nok{pas recevable} &
-- principe de proportionnalité: le juge refuse une saisie
lorsque la demande a un grand risque de ne pas être délivrée &
V. Jurisprudence\\

& 
\textbf{Objet?} & 
\\

\ok{peut être saisie} &
-- saisie réelle: documents, échantillons, produits;
description: prendre des photos, détails écrits.
\\

\nok{pas être saisie} &
-- pas dans le but d'interrompre la prodution,
par exemple saisir tout le stock, toutes les machines &
\\

&
\textbf{Confidentialité?} &
\\

&
-- l'huissier communique au juge le scellé &
R615-4\\

&
-- toutes les informations pour prouver la contrefaçon
sont communiquées au breveté &
\\

\ok{accès ouvert} &
-- décision expurgée des éléments confidentiels lorsque éléments
lien avec contrefaçon &
\\

\nok{accès restreint} &
-- décision complète: accès restreint au \textit{cercle de confidentialité}
lorsque lien avec contrefaçon et confidentiel &
\\

&
\textbf{Nullité du PV?} &
L615-5\\

\nok{pour vice de fond} &
-- "le conseil en brevet indique que \mydots" &
\\

&
-- "le conseil en brevet pose une question au saisi" &
\\

&
-- saisie d'un objet qui n'est pas dans l'ordonnance &
\\

&
-- pas de signification de l'ordonnance
\action{$\Rightarrow$ pas nécessaire de faire preuve d'un grief} &
\\

\nok{pour vice de forme} &
-- signification de l'ordonnance
dans un délai trop court par exemple quelques minutes
\action{$\Rightarrow$ nécessaire de faire preuve d'un grief} &
\\

&
\textbf{Recours?} &
\\

&
-- \action{référé-rétractation: réintroduction du contradictoire},
dire que le juge a prononcé une ordonnance en étant mal-informé &
A497 CPC \\

&
\textbf{Délai?} &
\\

&
-- \delai{maximum entre 20 jours ouvrables ou 31 jours civils
pour assignation / date de la saisie contrefaçon} &
R615-3\\

\hline
\textbf{Interdiction provisoire} &
&
L615-3\\

\hline
\textbf{Droit de l'information} &
&
L615-5-2\\

\end{longtblr}
