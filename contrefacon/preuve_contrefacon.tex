\section{Preuve de la contrefaçon}

\begin{longtblr}{colspec = {X[3,l] X[7,l] X[5,l]}}

\textbf{preuve=} & & \\
\hline
\textbf{généralités} &
\textbf{comment?} &
\\

principe &
-- par tous moyens &
L615-5 al. 1\\

\ok{+ force probante} &
-- constat d'huissier: constat internet, constat d'achat &
A145 CPC\\

&
-- constat d'achat effectué par un stagiaire de la partie requérante &
Pourvoi n°22-20.739\\

\nok{- force probante} &
-- témoignage en France & 
\\

\hline
\textbf{saisie cf.} &
\textbf{comment?} &
\\

\textbf{requête et \mydots} \newline principe
 & -- procédure sur requête (non contradictoire) et
 caractère exécutoire 
 & A489 CPC\\

\ok{recevable}
& -- \action{requête $\in$ état des annuités, 
inscriptions, projet d'ordonnance + copie du brevet}
& R615-4 al. 2 \\
% \nok{pas recevable} &
% -- principe de proportionnalité: le juge refuse une saisie
% lorsque la demande de brevet a un grand risque de ne pas être délivrée &
% V. Jurisprudence\\

\SetHline{2-3}{}
\textbf{objet et \mydots} & 
\textbf{quoi?} &
\\

saisie réelle & -- documents, échantillons, produit
 & L615-5 al. 2\\

desc. détaillée & -- photos, détails écrits
 & L615-5 al. 2\\

constat d'huissier & -- en vue d'établir l'origine, la consistance et l'étendue de la contrefaçon
 & R615-4 al. 4\\
% \nok{pas être saisie} &
% -- pas dans le but d'interrompre la prodution,
% par exemple saisir tout le stock, toutes les machines &
% \\

\SetHline{2-3}{}
\SetCell[r=2]{valign=h}\textbf{confidentialité et \mydots} &
\textbf{sur quoi?} &
\\

&
-- l'huissier communique au juge le scellé &
R615-4\\

&
-- toutes les informations pour prouver la contrefaçon
sont communiquées au breveté &
\\

\ok{accès ouvert} &
-- décision expurgée des éléments en lien avec le secret des affaires &
\\

\nok{accès restreint} &
-- décision complète avec accès restreint au \textit{cercle de confidentialité} &
\\

% \nok{pour vice de fond} &
% -- "le conseil en brevet indique que \mydots" &
% \\
% &
% -- "le conseil en brevet pose une question au saisi" &
% \\
% &
% -- saisie d'un objet qui n'est pas dans l'ordonnance &
% \\
\SetHline{2-3}{}
\SetCell[r=2]{valign=h} \textbf{\mydots délai de poursuite}
 & \textbf{quand?}
 & \\

& -- \delai{maximum entre 20 jours ouvrables ou 31 jours civils
pour assignation / date de l'ordonnance}
& R615-3\\

% \SetHline{2-3}{}
% \textbf{//JUB} & \textbf{comment?} & \\
\hline
\SetCell[r=2]{valign=h}\textbf{droit de l'information} \newline principe
 & \textbf{pourquoi?}
 & \\

 & -- pour évaluer le préjudice entre le premier jugement sur la contrefaçon et le deuxième jugement
sur le préjudice
 & \\

& \textbf{comment?} & \\

 & -- au fond ou en référé
 & L615-5-2 al. 1 \\

 & -- produire des documents, généralement des factures
 lorsque pas empêchement légitime
 & L615-5-2 al. 2 \\

\end{longtblr}
