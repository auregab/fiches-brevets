\section{Les conditions de l'action en contrefaçon}

\begin{longtblr}{colspec = {X[3,l] X[8,l] X[3,l]}}

\hline
&
\textbf{Qui?} &
L615-2\\

\ok{peut agir} &
-- déposant &
\\

&
-- co-propriétaire &
L613-29\\

&
-- cessionnaire: transmission à titre particulier, transmission à titre universel
(p. ex: fusion, absorption, héritage) &
\\

&
-- cessionnaire: régularisation rétroactive jusqu'au jour de la cession et
 inscription RNB en cours de procédure &
Cass. 22/04/2024\\

&
-- licencié exclusif: action en cessation de l'atteinte et action en réparation &
L615-2 al. 2\\

&
-- licencié non exclusif: action en réparation de son préjudice ou
intervention dans l'instance engagée par le breveté pour réparation de son préjudice propre &
L615-2 al. 6\\

&
-- licencié non inscrit au RNB pour se joindre à l'action du breveté
pour réparation de son préjudice propre &
L613-9\\

&
- pour les marques/modèles: licencié non inscrit au RNB engage une action &
Cour de Justice 22/06/2016\\

\nok{peut pas agir} &
-- copropriétaire lorsque pas de notification aux autres copropriétaires &
L613-29\\

&
-- cessionnaire irrecevable (inopposable aux tiers) lorsque \action{cession non inscrite au RNB} &
L613-9\\

&
-- cédant sur réparation en préjudice lorsque \action{cession des droits de poursuite antérieurs} &
\\

&
-- licencié exclusif lorsque \action{pas de notification au breveté} et
lorsque \action{interdiction d'engager une action en contrefaçon} &
\\

&
-- licencié non-exclusif lorsque \action{pas de notification au breveté} et
lorsque \action{pas d'autorisation expresse d'engager une action en contrefaçon} &
\\

\hline
&
\textbf{Titre?}
\\

\ok{valable} &
-- demande FR: le droit exclusif d'exploitation nait au dépôt de la demande &
L613-1\\

&
-- demande FR publiée ou notifiée: les faits = actes de contrefaçon à partir de la publication de la demande
ou à partir de la notification au contrefacteur présumé &
L615-4\\

&
-- demande EP publiée: protection provisoire à partir de la publication &
L614-9\\

&
-- brevet EP sous opposition le juge surseoit à statuer juqu'à la fin de l'opposition
pour \textit{une bonne administration de la justice} &
Code de Procédure Civile\\

&
-- brevet FR expiré pour action en réparation sur actes 5 ans avant assignation
et jusqu'à expiration &
\\

\nok{pas valable} &
-- demande FR modifiée: jusqu'à la délivrance pas opposable $\Rightarrow$ sauf renotification au contrefacteur
présumé &
L615-4\\

&
-- brevet FR expiré pour action en cessation &
\\

\hline
&
\textbf{Quand?} &
\\

&
-- \delai{5 ans à/c jour où le titulaire d'un droit a connu ou aurait dû connaître
le dernier fait lui permettant de l'exercer} &
L615-8\\

&
-- prescription distributive: chaque acte fait naitre un délai de prescription propre &
\\

&
-- interruption du délai par assignation en justice &
\\

\hline
&
\textbf{Exception} &
\\

&
-- possession personnelle antérieure &
L613-7\\

\end{longtblr}