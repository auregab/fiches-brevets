\section{Actes de contrefaçon directe}

\begin{longtblr}{colspec = {X[3,l] X[7,l] X[5,l]}, rowsep = 1pt}

% L613-3 L613-4 A25 A26 AJUB
\textbf{cf. directe=} & & \\

\hline
\textbf{produit} & \textbf{acte?} & \\ 

\nok{fabrct.} et/ou \mydots \newline produit &
-- tout acte qui permet de parvenir à l'objet de la revendication de produit
 & L613-3(a) \\

\nok{offre} et/ou \mydots \newline produit ou produit issu d'un procédé
 & -- tout opération matérielle tendant à préparer la clientèle potentielle
 à la commercialisation prochaine du produit
 & L613-3(a) ou L613-3(c) \newline Pourvoi n°15-20554 \\

\nok{mise ds le com.} et/ou \mydots \newline produit ou produit issu d'un procédé
 & -- mise à disposition ayant pour effet de transférer la jouissance du produit,
 par la vente ou par la location
 & L613-3(a) ou L613-3(c) \newline Pourvoi n°91-21228 \\

\nok{utilisation} et/ou \mydots \newline produit ou produit issu d'un procédé
& -- \action{utilisation} d'un produit objet d'un brevet ou d'un produit directement issu d'un procédé objet d'un brevet
& L613-3(a) ou L613-3(c) \\

\nok{importation} et/ou \mydots \newline produit ou produit issu d'un procédé
& -- faire entrer en France une marchandise provenant d'un autre territoire;
 importateur = opérateur français ou fabriquant étranger ayant \textit{un lien particulier}
 avec l'opérateur français
 & L613-3(a) ou L613-3(c) \newline RG n°09/14081 \\
 % \ok{pas contrefaisant}
%  & -- un fabriquant étranger met seulement la marchandise dans son usine
%  sans s'occuper du transport ni de l'importation
%  & RG n° 20/10445 \\

%  \nok{contrefaisant}
%  & -- une société exportratrice étrangère et un revendeur français,
%  toutes deux sociétés soeurs du même groupe
%  & RG n°21/06748 \\

\nok{exportation} et/ou \mydots \newline produit ou produit issu d'un procédé
 & -- \action{exportation} d'un produit objet d'un brevet ou d'un produit directement issu d'un procédé objet d'un brevet
 & L613-3(a) ou L613-3(c) \\

\nok{transbordement} et/ou \mydots \newline produit ou produit issu d'un procédé
 & -- transfert de marchandises qui sont enlevées du moyen de transport pour l'importation
 vers un autre moyen de transport pour l'exportation
 & L613-3(a) ou L613-3(c) \newline Convention de Kyoto \\

\nok{détention} et/ou \mydots \newline produit ou produit issu d'un procédé
 & -- \action{détention} d'un produit objet d'un brevet ou d'un produit directement issu d'un procédé objet d'un brevet aux fins précitées
 & L613-3(a) ou L613-3(c) \\
% \ok{pas contrefaisant}
%  & -- un transporteur détenant des produits contrefaisant aux seules fins du transport
%  & \\
% \ok{pas contrefaisant}
%  & -- détention aux fins de la fabrication car le produit n'existe pas avant sa fabrication
%  & \\
%  & -- détention aux fins de l'importation du fait de la territorialité du brevet
%  & \\
\hline
\textbf{procédé} & \textbf{acte?} & \\
\nok{utilisation} et/ou \mydots \newline d'un procédé
& -- \action{utilisation} d'un procédé objet d'un brevet
& L613-3(b) \\

\nok{offre d'utilisation} et/ou \mydots \newline d'un procédé
& -- deux conditions territorialité: (i) offre d'utilisation en FR
et (ii) utilisation envisagée en FR
& L613-3(b) \\

% \ok{pas contrefaisant}
%  & -- une première personne reproduit une partie des étapes du procédé et
%  le reste des étapes est mis en oeuvre par une deuxième personne
%  & PIBD 2006, 822, IIIB-43 \\

% \ok{pas contrefaisant}
%  & -- utilisation en FR d'une partie des étapes du procédé
%  & \\

%  \ok{pas contrefaisant}
%  & -- offre faite en France d'utilisation d'un procédé breveté à l'étranger
%  & \\

\end{longtblr}