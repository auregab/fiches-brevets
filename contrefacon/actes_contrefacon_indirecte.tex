\section{Actes de contrefaçon indirecte}

\begin{longtblr}{colspec = {X[3,l] X[7,l] X[5,l]}, rowsep = 1pt}

\textbf{fourniture de moyens=} & & \\
\hline
\textbf{lieu et \mydots} & \textbf{où?} & \\

\SetCell[r=2]{valign=h}première condition de territorialité
 & -- livraison sur le territoire français ou \mydots & L613-4(1) \\

 & -- offre de livraison sur le territoire français & L613-4(1) \\

\SetCell[r=2]{valign=h}deuxième condition de territorialité
 & -- moyens mis en oeuvre sur le territoire français & L613-4(1) \\

 & -- moyens de mise en oeuvre de l'invention livrés sur
 le territoire français mais mis en oeuvre hors du territoire français
 & Pourvoi n°05-15124 \\

\SetHline{2-3}{}
\textbf{personne et \mydots} & \textbf{qui?} & \\

\nok{peut pas fournir les moyens} & -- une personne autre que celles habilitées à
 exploiter l'invention brevetée
 & L613-4(1) \\

\SetHline{2-3}{}
\SetCell[r=2]{valign=h}\textbf{moyens de mise en oeuvre et \mydots} & \textbf{quoi?} & \\

principe & -- les moyens de mise en oeuvre de l'invention
 se rapportant à un \textit{élément essentiel de l'invention brevetée}
 & L613-4(1) \\

 première interprétation "technique"
 & -- moyens sans lesquels l'invention ne pourrait être mise en oeuvre
 & Rennes, 03/12/2008 \\

 deuxième interprétation "juridique"
 & -- caractéristique conférant nouveauté ou activité inventive
 & PIBD 2010, 909 IIIB-14, PIBD 2009, 887, IIIB-723 \\

 & -- un moyen partie d'une combinaison, voire même un consommable
 & Pourvoi n°15-29378 \\

\SetHline{2-3}{}
\SetCell[r=3]{valign=h}\textbf{\mydots moyens aptes et destinés}\newline
\nok{à associer à la condition de responsabilité}
 & \textbf{quoi?} & \\

 & -- aptitude à mettre en oeuvre l'invention
 & \\
 
difficile à vérifier
 & -- destination de la mise en oeuvre de l'invention
 & \\

 \end{longtblr}

