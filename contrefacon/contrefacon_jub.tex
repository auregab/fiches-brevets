\section{Contrefaçon à la JUB}

\begin{longtblr}{colspec = {X[3,l] X[7,l] X[5,l]}}

\hline

\textbf{preuve cf.} & & \\

moyens de preuve
 & -- moyens de preuve
 & A53 AJUB \\

saisie cf.
& -- ordonnance de conservation des preuves et de descente sur les lieux 
& A60 \\

droit d'information
 & -- pouvoir d'ordonner la communication d'information
 & A67 AJUB \\

\SetCell[r=2]{}\newline \textbf{CF indirecte}
\newline \ok{le tier sait ou aurait du savoir} & & \\

  &  [...] sur le territoire des
Etats membres contractants dans
lesquels le brevet produit ses effets, [...], des
moyens de mise en œuvre, sur ce territoire, 
de cette invention se rapportant à
un élément essentiel de celle-ci, [...]
 & A26 AJUB \\

 \textbf{CF directe} \newline \nok{pas de condition}
 & -- tous les actes de cf. directe
 & A25 AJUB \\

\SetHline{2-3}{}

\textbf{compétence} & -- lieu domicile du défenseur 
 & A33(1)b) 1ère partie AJUB \\

 & -- lieu du dommage 
 & A33(1)a) AJUB \\

 & -- lieu du défenseur d'ancrage
 à condition (i) lien commercial
 (ii) même contrefaçon alléguée
 & A33(1)b) 2e partie AJUB \\

\SetHline{2-3}{}

\textbf{exceptions}
 & -- limitations des effets d'un brevet
 & A27 AJUB \\

possession antérieure personnelle
 & -- droit fondé sur une utilisation antérieure de l'invention
 & A28 AJUB \\ 

\textbf{dommages} & & \\

\nok{sciemment ou motifs raisonnables}
 & -- dommages
 & A62(2)+A62(3) AJUB\\

\ok{pas sciemment ou pas motif raisonnable}
 & -- réparation financière = (i) restitution
 des bénéfices ou (ii) versement d'indemnités
 & A62(2)+A62(3) AJUB\\

\end{longtblr}