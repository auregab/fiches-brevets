\section{Interdiction provisoire}

\begin{longtblr}{colspec = {X[3,l] X[7,l] X[5,l]}, rowsep = 1pt}

\hline

\textbf{personne et \mydots} & \textbf{qui?} & \\

 & -- toute personne ayant qualité pour agir en contrefaçon
 & L615-3 al. 1 1ère phrase \\

\SetHline{2-3}{}
\textbf{procédure et \mydots} & \textbf{comment?} & \\

en référé & -- procédure contradictoire sous conditions: (i) urgence et
 (ii) la mesure ne se heurte à aucune constestation sérieuse
 & L615-3 al. 1 1ère phrase + A494 CPC\\

 & -- délai d'appel: \delai{15 jours / ordonnance de référé}
 & A490 CPC \\

sur requête & -- procédure non contradictoire sous condition urgence
 & L615-3 al. 1 2ème phrase \\

 & -- délai d'appel: \delai{15 jours / ordonnance sur requête}
 & A496 CPC \\

\SetHline{2-3}{}
\textbf{preuve et \mydots} & \textbf{quoi?} & \\
avant cf. & -- preuve raisonnablement accessible au demandeur et vraisemblable
qu'une atteinte à ses droits est imminente
 & L615-3 al. 1 3ère phrase \\
 
après cf. & -- preuve raisonnablement accessible au demandeur et vraisemblable
qu'il est porté atteinte à ses droits
 & L615-3 al. 1 3ère phrase \\

 \SetHline{2-3}{}
\textbf{mesure et \mydots} & \textbf{quoi?} & \\

& -- interdire la poursuite des actes argués de contrefaçon
 & L615-3 al. 2 1ère phrase \\

 & -- la subordonner à la consitution de garanties par le défendeur
 & L615-3 al. 2 1ère phrase \\

 & -- ordonner la saisie ou la remise entre les mains d'un tiers
 des produits soupçonnés de contrefaçon
 & L615-3 al. 2 1ère phrase \\

 & -- saisie conservatoire des biens mobiliers et immobiliers
 du prétendu contrefacteur, y compris le blocage de ses comptes bancaires
 lorsque difficulté de recouvrement des D\&I
 & L615-3 al. 2 2e phrase \\

 & -- ordonner la communication des documents comptables pour déterminer les biens suceptibles de faire l'objet de la saisie 
 & L615-3 al. 2 2e phrase \\

 & -- accorder au demandeur une provision quand le préjudice est peu constestable
 & L615-3 al. 3 \\

\hline

 \SetCell[r=2] \textbf{conséquences} \newline délai
 & -- délai maximum entre
 \delai{20 jours ouvrables / ordonnance d'interdiction provisoire} et
 \delai{31 jours / ordonnance d'interdiction provisoire}
 & R615-1 \\

délai non respecté & -- annulation des mesures ordonnées
 & \\

préjudice & -- cf présumé peut demander réparation pour le préjudice subi
 du fait de l'exécution lorsque décision provisoire = interdiction provisoire
 et lorsque décision au fond = pas de cf
 & \\

\end{longtblr}