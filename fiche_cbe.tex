% Options for packages loaded elsewhere
\PassOptionsToPackage{unicode}{hyperref}
\PassOptionsToPackage{hyphens}{url}
%
\documentclass[
]{article}
\usepackage{amsmath,amssymb}
\usepackage{iftex}
\ifPDFTeX
  \usepackage[T1]{fontenc}
  \usepackage[utf8]{inputenc}
  \usepackage{textcomp} % provide euro and other symbols
\else % if luatex or xetex
  \usepackage{unicode-math} % this also loads fontspec
  \defaultfontfeatures{Scale=MatchLowercase}
  \defaultfontfeatures[\rmfamily]{Ligatures=TeX,Scale=1}
\fi
\usepackage{lmodern}
\ifPDFTeX\else
  % xetex/luatex font selection
\fi
% Use upquote if available, for straight quotes in verbatim environments
\IfFileExists{upquote.sty}{\usepackage{upquote}}{}
\IfFileExists{microtype.sty}{% use microtype if available
  \usepackage[]{microtype}
  \UseMicrotypeSet[protrusion]{basicmath} % disable protrusion for tt fonts
}{}
\makeatletter
\@ifundefined{KOMAClassName}{% if non-KOMA class
  \IfFileExists{parskip.sty}{%
    \usepackage{parskip}
  }{% else
    \setlength{\parindent}{0pt}
    \setlength{\parskip}{6pt plus 2pt minus 1pt}}
}{% if KOMA class
  \KOMAoptions{parskip=half}}
\makeatother
\usepackage{xcolor}
\usepackage{longtable,booktabs,array}
\usepackage{multirow}
\usepackage{calc} % for calculating minipage widths
% Correct order of tables after \paragraph or \subparagraph
\usepackage{etoolbox}
\makeatletter
\patchcmd\longtable{\par}{\if@noskipsec\mbox{}\fi\par}{}{}
\makeatother
% Allow footnotes in longtable head/foot
\IfFileExists{footnotehyper.sty}{\usepackage{footnotehyper}}{\usepackage{footnote}}
\makesavenoteenv{longtable}
\ifLuaTeX
  \usepackage{luacolor}
  \usepackage[soul]{lua-ul}
\else
  \usepackage{soul}
\fi
\setlength{\emergencystretch}{3em} % prevent overfull lines
\providecommand{\tightlist}{%
  \setlength{\itemsep}{0pt}\setlength{\parskip}{0pt}}
\setcounter{secnumdepth}{-\maxdimen} % remove section numbering
\ifLuaTeX
  \usepackage{selnolig}  % disable illegal ligatures
\fi
\IfFileExists{bookmark.sty}{\usepackage{bookmark}}{\usepackage{hyperref}}
\IfFileExists{xurl.sty}{\usepackage{xurl}}{} % add URL line breaks if available
\urlstyle{same}
\hypersetup{
  hidelinks,
  pdfcreator={LaTeX via pandoc}}

\author{}
\date{}

\begin{document}

\begin{enumerate}
\def\labelenumi{\arabic{enumi}.}
\item
  \textbf{\ul{OBTENIR UNE DATE DE DEPOT}}
\end{enumerate}

\begin{longtable}[]{@{}
  >{\raggedright\arraybackslash}p{(\columnwidth - 4\tabcolsep) * \real{0.2345}}
  >{\raggedright\arraybackslash}p{(\columnwidth - 4\tabcolsep) * \real{0.5782}}
  >{\raggedright\arraybackslash}p{(\columnwidth - 4\tabcolsep) * \real{0.1874}}@{}}
\toprule\noalign{}
\begin{minipage}[b]{\linewidth}\raggedright
\end{minipage} & \begin{minipage}[b]{\linewidth}\raggedright
\textbf{Où~?}
\end{minipage} & \begin{minipage}[b]{\linewidth}\raggedright
A75
\end{minipage} \\
\midrule\noalign{}
\endhead
\bottomrule\noalign{}
\endlastfoot
& (i) auprès OEB & A75(1)a), R35(1) \\
& (ii) office national compétent & A75(1)b) \\
& \textbf{Langue}~? & A14 \\
Date de dépôt & - (i) description dans une première langue (ii) dessins
dans une deuxième langue sous condition traduction dans une langue
officielle & T382/94 \\
& - seul suffit de traduire dans une langue officielle la description,
pas les dessins & J7/80 \\
Pas de date de dépôt & - (i) première partie de la description dans une
première langue (ii) deuxième partie de la description dans une deuxième
langue & J18/96 \\
& \textbf{Parties manquantes~?} & R56 \\
Date de dépôt initiale & - demande sans les parties manquantes &
R56(4) \\
& - demande renvoie à une demande antérieure à condition (i)
revendication de priorité (ii) fourniture d'une copie (iii) intégralité
& R56(3) \\
& - demande complétée mais les parties manquantes sont retirées
(changement d'avis) & R56(6) \\
(~?) & - rajouter une figure & J15/12 \\
Date de dépôt corrigée & - demande complétée avec les parties manquantes
& R56(2) \\
Non admissible & - remplacer un texte par un autre & J25/10 \\
& - pas remplacer des dessins de mauvaise qualité & J12/14 \\
\end{longtable}

\begin{enumerate}
\def\labelenumi{\arabic{enumi}.}
\setcounter{enumi}{1}
\item
  \textbf{\ul{CORRIGER UNE ERREUR}}
\end{enumerate}

\begin{longtable}[]{@{}
  >{\raggedright\arraybackslash}p{(\columnwidth - 4\tabcolsep) * \real{0.2345}}
  >{\raggedright\arraybackslash}p{(\columnwidth - 4\tabcolsep) * \real{0.5782}}
  >{\raggedright\arraybackslash}p{(\columnwidth - 4\tabcolsep) * \real{0.1874}}@{}}
\toprule\noalign{}
\begin{minipage}[b]{\linewidth}\raggedright
\textbf{Priorité}
\end{minipage} & \begin{minipage}[b]{\linewidth}\raggedright
\textbf{Délai~?}
\end{minipage} & \begin{minipage}[b]{\linewidth}\raggedright
\end{minipage} \\
\midrule\noalign{}
\endhead
\bottomrule\noalign{}
\endlastfoot
& \begin{minipage}[t]{\linewidth}\raggedright
- délai maximum entre\\
(i) 4 mois / dépôt de la demande EP~;\\
(ii) 16 mois / date de priorité la plus ancienne corrigée\strut
\end{minipage} & R52(3) \\
\textbf{Pièce =/= dem.} & \textbf{Motif~?} & R139 1\textsuperscript{ère}
phrase \\
Principe & \begin{minipage}[t]{\linewidth}\raggedright
- balance entre\\
(i) intérêt du demandeur~;\\
(ii) sécurité juridique des tiers\strut
\end{minipage} & J6/91 \\
valable & - erreur sur le demandeur & J7/80 \\
& - retrait de désignation & (~?) \\
& - erreur sur compte courant & T186/22 \\
Pas valable & - paiements & J21/84 \\
& - recours suite à retrait & T824/00 \\
& - désignations manquantes & J8/01 \\
& - mauvaise communication entre mandataire et client & T433/21 \\
\textbf{Dem. EP.} & \textbf{Rectifiable~?} & R139 2\textsuperscript{e}
phrase \\
Principe & - DDSA = directement déductible sans ambiguïté & G3/91,
G11/91 \\
valable & - une plage de valeur revendiquée différente de tous les
exemples décrits & T200/89 \\
Pas valable & - une plage de valeur revendiquée correspondant en partie
à certains exemples décrits & T1702/12 \\
& - une correction de caractéristiques présentées comme essentielles &
T2058/18 \\
& - une erreur de calcul & T13/83 \\
& - une erreur de référence à un document inaccessible par un tier &
T1486/12 \\
& \textbf{Irrégularité~?} & \\
corrigeable & - demandeur pas d'un état contractant =\textgreater{}
faire un recours avec un mandataire EP (mais c'est cher) & J18/08 \\
Pas corrigeable & - implique une connaissance technique & J11/20 \\
\end{longtable}

\begin{enumerate}
\def\labelenumi{\arabic{enumi}.}
\setcounter{enumi}{2}
\item
  \textbf{\ul{NOMENCLATURE DE DEMANDES}}
\end{enumerate}

\begin{longtable}[]{@{}
  >{\raggedright\arraybackslash}p{(\columnwidth - 4\tabcolsep) * \real{0.2338}}
  >{\raggedright\arraybackslash}p{(\columnwidth - 4\tabcolsep) * \real{0.5776}}
  >{\raggedright\arraybackslash}p{(\columnwidth - 4\tabcolsep) * \real{0.1886}}@{}}
\toprule\noalign{}
\begin{minipage}[b]{\linewidth}\raggedright
\textbf{Divisionnaire}
\end{minipage} & \begin{minipage}[b]{\linewidth}\raggedright
\textbf{Délai~?}
\end{minipage} & \begin{minipage}[b]{\linewidth}\raggedright
\end{minipage} \\
\midrule\noalign{}
\endhead
\bottomrule\noalign{}
\endlastfoot
recevable & - 2 mois / décision de rejet (délai de recours) & G1/09 \\
& - entre le recours sur délivrance et la délivrance = effet suspensif &
J1/24 \\
& - la veille de la publication de la mention de délivrance & J24/10 \\
& \textbf{Langue~?} & \\
recevable & - la langue de procédure & J13/14 \\
& \textbf{Contenu~?} & \\
recevable & - une demande (i) pas conforme à A123(2) (ii) l'irrégularité
est corrigeable & G1/05, G1/06 \\
& - pas le même objet revendiqué qu'un brevet EP délivré pour le même
demandeur & G4/19 \\
& \textbf{Déposant~?} & \\
Irrecevable & - un des déposants & J2/01 \\
\textbf{Demande volée} & \textbf{Titulaire (présumé usurpateur)~?} &
Protocole s/ la reconnaissance \\
& - Domicilié dans un état contractant =\textgreater{} tribunal de
l'état contractant du titulaire & A2 PR \\
& - Pas dans un état contractant et demandeur dans un état contractant
=\textgreater{} tribunal de l'état contractant du demandeur & A3 PR \\
& - Contrat dans un état contractant pour une invention de salarié
=\textgreater{} tribunal de l'état contractant & A4 PR \\
& - Convention entre parties d'un tribunal =\textgreater{} le tribunal
de l'état contractant convenu & A5 PR \\
& - Demandeur et titulaire pas dans un état contractant =\textgreater{}
tribunal DE & A6 PR \\
& \textbf{Effet~?} & \\
& - décision du tribunal compétent = décision reconnue par les autres
tribunaux & A9, A11 \\
& - redéposer une invention volée en bénéficiant de la date de dépôt
initiale & G3/92 \\
\textbf{Dem. Euro-PCT} & \textbf{Langue~?} & \\
& - langue de procédure EP = langue de procédure PCT lorsque langue de
procédure PCT est une langue officielle OEB & G1/08 \\
& \textbf{Effet~?} & \\
& - euro-PCT A54(3) lorsque (i) payé taxe de dépôt (ii) traduite dans
une langue officielle & R165 \\
\end{longtable}

\begin{enumerate}
\def\labelenumi{\arabic{enumi}.}
\setcounter{enumi}{3}
\item
  \textbf{\ul{LA RECHERCHE}}
\end{enumerate}

\begin{longtable}[]{@{}
  >{\raggedright\arraybackslash}p{(\columnwidth - 4\tabcolsep) * \real{0.2345}}
  >{\raggedright\arraybackslash}p{(\columnwidth - 4\tabcolsep) * \real{0.5782}}
  >{\raggedright\arraybackslash}p{(\columnwidth - 4\tabcolsep) * \real{0.1874}}@{}}
\toprule\noalign{}
\begin{minipage}[b]{\linewidth}\raggedright
\textbf{Type de recherche}
\end{minipage} & \begin{minipage}[b]{\linewidth}\raggedright
\textbf{Acte~?}
\end{minipage} & \begin{minipage}[b]{\linewidth}\raggedright
\end{minipage} \\
\midrule\noalign{}
\endhead
\bottomrule\noalign{}
\endlastfoot
Non-unitaire & - choisir une des revendications indépendantes &
R62bis \\
Pas claire & - déclaration motivée pour défauts de clartés & R63 \\
Sous priorité & - fournir la recherche antérieure & R141 \\
Entrée en phase & - aucun (~?), recherche complémentaire & A153 \\
\textbf{Unité} & \textbf{Actes~?} & \\
Objet non recherché & - déposer une demande divisionnaire & G2/92 \\
/ examen & - a priori et a posteriori (~?) & G1/89 \\
& - pas d'invitation multiple & W1/97 \\
& - examinateur lié par l'objet couvert dans la recherche d'avant PCT
=\textgreater{} EP & T44/19 \\
\textbf{Publication} & \textbf{Type~?} & \\
A1 & - demande EP publiée avec RRE & \\
A2 & - demande EP publiée sans RRE & \\
A3 & - RRE publié & \\
B1 & - brevet EP délivré & \\
B2 & - brevet EP maintenu & \\
\end{longtable}

\begin{enumerate}
\def\labelenumi{\arabic{enumi}.}
\setcounter{enumi}{4}
\item
  \textbf{\ul{L'EXAMEN}}
\end{enumerate}

\begin{longtable}[]{@{}
  >{\raggedright\arraybackslash}p{(\columnwidth - 4\tabcolsep) * \real{0.2345}}
  >{\raggedright\arraybackslash}p{(\columnwidth - 4\tabcolsep) * \real{0.5782}}
  >{\raggedright\arraybackslash}p{(\columnwidth - 4\tabcolsep) * \real{0.1874}}@{}}
\toprule\noalign{}
\begin{minipage}[b]{\linewidth}\raggedright
\textbf{Traitement accéléré}
\end{minipage} & \begin{minipage}[b]{\linewidth}\raggedright
\textbf{Type~?}
\end{minipage} & \begin{minipage}[b]{\linewidth}\raggedright
\end{minipage} \\
\midrule\noalign{}
\endhead
\bottomrule\noalign{}
\endlastfoot
OEB & - demande PACE pour EP et PCT-EP & \\
International & - demande PPH pour IP5 & \\
& \textbf{Réduction~?} & \\
valide & - 100\% taxe d'examen lorsque retrait avant le début de
l'examen & A11 RRT \\
& - 50\% taxe d'examen lorsque retrait à la 1ere A94(3) ou A71(3) & \\
& - 100\% taxe d'examen pour rapport de recherche complémentaire lorsque
retrait avant le début de l'examen complémentaire & J6/83 \\
\textbf{Droit d'être entendu} & \textbf{Motif~?} & A113 \\
2\textsuperscript{e} chance & - le demandeur a formulé un raisonnement
erroné & T185/82 \\
& - nouvelles antériorités & T1982/07 \\
pas de 2\textsuperscript{e} chance & - une réponse pas claire à une
notification & T84/82 \\
& - obligation de neutralité~; le demandeur demande à l'OEB les
modifications à faire & T448/16 \\
& - la surprise quant au raisonnement tenu par la chambre de recours &
R5/16 \\
\textbf{Décision sur examen} & \textbf{Délai~?} & A97 \\
Pour modifier & - jusqu'à la décision de délivrance & G7/93 \\
& - date de décision = date de remise au service du courrier interne &
G12/91 \\
& \textbf{Preuve~?} & \\
principe & - charge de la preuve à la division d'examen & T655/13 \\
& - expliciter les modifications non soutenues & T951/92 \\
& - refuser toutes les requêtes auxiliaires avant la dernière &
T1105/96 \\
exception & - tous les motifs ne sont pas nécessairement dans la LO &
T208/84 \\
\end{longtable}

\begin{enumerate}
\def\labelenumi{\arabic{enumi}.}
\setcounter{enumi}{5}
\item
\item
  \textbf{\ul{MODIFICATIONS}}
\end{enumerate}

\begin{longtable}[]{@{}
  >{\raggedright\arraybackslash}p{(\columnwidth - 4\tabcolsep) * \real{0.2518}}
  >{\raggedright\arraybackslash}p{(\columnwidth - 4\tabcolsep) * \real{0.5608}}
  >{\raggedright\arraybackslash}p{(\columnwidth - 4\tabcolsep) * \real{0.1874}}@{}}
\toprule\noalign{}
\begin{minipage}[b]{\linewidth}\raggedright
\textbf{Procédure}
\end{minipage} & \begin{minipage}[b]{\linewidth}\raggedright
\textbf{Délai~?}
\end{minipage} & \begin{minipage}[b]{\linewidth}\raggedright
\end{minipage} \\
\midrule\noalign{}
\endhead
\bottomrule\noalign{}
\endlastfoot
En principe & - après le rapport de recherche avant la première
notification de modification & R137(2) \\
\textbf{Extension} & \textbf{Modification~?} & \\
Principe & - test de nouveauté & T194/84 \\
& - norme de référence = directement et sans ambiguïté & G3/89,
G11/91 \\
inadmissible & - revendications en contradiction avec la description &
T1961/11 \\
& - modification de revendications non supportée par la description &
G1/93 \\
& - changement de catégorie de revendication vers une catégorie
supérieure ie. revendication d'utilisation de produit en revendication
de produit & G2/88 \\
& - modification de la description et des dessins étendant la protection
conférée au sens A69 protocole & T1149/97 \\
& - suppression d'une caractéristique essentielle & T260/85 \\
& - généraliser une structure & T416/86 \\
& - une modification R139 s'étendant au-delà de la demande telle que
déposée & T401/88 \\
& - une caractéristique dans la partie art antérieure mais pas dans la
description détaillée & T689/90 \\
& - une caractéristique d'un document de renvoi comprenant des termes ne
correspondant pas DDSA à ceux de la demande & T1084/22 \\
& - une caractéristique fonctionnelle à partir d'une caractéristique
spécifique & T416/86 \\
admissible & - changement de catégorie de revendication vers une
catégorie inférieure ie. revendication de produit changée en
revendication d'utilisation du produit & G2/88 \\
& - une caractéristique n'ayant pas de lien étroit à d'autres
caractéristique d'un exemple & T962/98 \\
& - la combinaison de caractéristique est divulguée de manière directe
et non ambiguë & T298/22 \\
\end{longtable}

\begin{enumerate}
\def\labelenumi{\arabic{enumi}.}
\setcounter{enumi}{7}
\item
  \textbf{\ul{DISCLAIMER}}
\end{enumerate}

\begin{longtable}[]{@{}
  >{\raggedright\arraybackslash}p{(\columnwidth - 4\tabcolsep) * \real{0.2345}}
  >{\raggedright\arraybackslash}p{(\columnwidth - 4\tabcolsep) * \real{0.5782}}
  >{\raggedright\arraybackslash}p{(\columnwidth - 4\tabcolsep) * \real{0.1874}}@{}}
\toprule\noalign{}
\begin{minipage}[b]{\linewidth}\raggedright
\end{minipage} & \begin{minipage}[b]{\linewidth}\raggedright
\textbf{Non-supporté~?}
\end{minipage} & \begin{minipage}[b]{\linewidth}\raggedright
\end{minipage} \\
\midrule\noalign{}
\endhead
\bottomrule\noalign{}
\endlastfoot
Admissible & - rétablir nouveauté par rapport à A54(3) &
\multirow{2}{*}{G1/03 sommaire} \\
& - rétablir nouveauté par rapport à une divulgation fortuite A54(2) \\
& - exclure une exception à la brevetabilité & \\
& - pas retrancher plus que ce qui est nécessaire & \\
& - pas pertinent pour l'activité inventive & \\
& - claire et concise & \\
Pas admissible & - pas plus que ce qui est nécessaire =\textgreater{}
A123(2) & T747/00 motif 2.1.5 \\
& - rétablir la nouveauté (i) par rapport à une antériorité A54(3) et
(ii) par rapport à une antériorité non fortuite A54(2) & T788/05 \\
& - disclaimer pour rétablir la priorité pour écarter une antériorité
non fortuite & T1213/05 \\
& \textbf{Supporté~?} & \\
admissible & - exclure un mode de réalisation particulier à condition
que l'objet restant est décrit DDSA & G2/10 \\
Pas admissible & - supprimer le disclaimer & T2327/18 \\
& \textbf{Résumé~?} & \\
& \begin{minipage}[t]{\linewidth}\raggedright
(i) disclaimer non supporté =\textgreater{} G1/03~;\\
(ii) disclaimer supporté =\textgreater{} G2/10\strut
\end{minipage} & G1/16 \\
\end{longtable}

\begin{enumerate}
\def\labelenumi{\arabic{enumi}.}
\setcounter{enumi}{8}
\item
  \textbf{\ul{LA PRIORITE (cf. fiche priorité)}}
\item
  \textbf{\ul{CONDITIONS DE BREVETABILITE (cf. fiche conditions de
  brevetabilité)}}
\end{enumerate}

\begin{longtable}[]{@{}
  >{\raggedright\arraybackslash}p{(\columnwidth - 4\tabcolsep) * \real{0.2345}}
  >{\raggedright\arraybackslash}p{(\columnwidth - 4\tabcolsep) * \real{0.5782}}
  >{\raggedright\arraybackslash}p{(\columnwidth - 4\tabcolsep) * \real{0.1874}}@{}}
\toprule\noalign{}
\begin{minipage}[b]{\linewidth}\raggedright
\textbf{A52}
\end{minipage} & \begin{minipage}[b]{\linewidth}\raggedright
\textbf{Invention~?}
\end{minipage} & \begin{minipage}[b]{\linewidth}\raggedright
\textbf{Source}
\end{minipage} \\
\midrule\noalign{}
\endhead
\bottomrule\noalign{}
\endlastfoot
brevetable & - une méthode comprenant une étape de diagnostique &
G1/04 \\
& - un logiciel qui résout un problème technique & G1/19 \\
Pas brevetable & - une invention trop polluante & T356/93 \\
& - une chimère & T1553/22 \\
& - une variété de tomate & G2/12 \\
& - un produit obtenu par un procédé essentiellement biologique &
G3/19 \\
\textbf{A54} & \textbf{Accessibilité au public~?} & \\
admise & - un échantillon dans une clinique & T505/15 \\
& - un produit dans une utilisation du produit & G2/88 motif 5 \\
& \textbf{Etat de la technique~?} & \\
Admise & - une composition chimique accessible même si non analysable et
non reproductible & G1/23 \\
\textbf{A56} & \textbf{Preuve~?} & \\
admise & - un effet technique plausible à la date du dépôt & G2/21 \\
& - un document postérieur à la date du dépôt pour détailler un préjugé
technique² & T1110/03 \\
\end{longtable}

\begin{enumerate}
\def\labelenumi{\arabic{enumi}.}
\setcounter{enumi}{10}
\item
  \textbf{\ul{L'OPPOSITION}}
\end{enumerate}

\begin{longtable}[]{@{}
  >{\raggedright\arraybackslash}p{(\columnwidth - 4\tabcolsep) * \real{0.2504}}
  >{\raggedright\arraybackslash}p{(\columnwidth - 4\tabcolsep) * \real{0.5719}}
  >{\raggedright\arraybackslash}p{(\columnwidth - 4\tabcolsep) * \real{0.1777}}@{}}
\toprule\noalign{}
\begin{minipage}[b]{\linewidth}\raggedright
\textbf{A99}
\end{minipage} & \begin{minipage}[b]{\linewidth}\raggedright
\textbf{Qui~?}
\end{minipage} & \begin{minipage}[b]{\linewidth}\raggedright
Source
\end{minipage} \\
\midrule\noalign{}
\endhead
\bottomrule\noalign{}
\endlastfoot
\multirow{3}{*}{Peut former opposition} &
\begin{minipage}[t]{\linewidth}\raggedright
-\ul{principe~:} «~Toute personne~» selon l'article 58 CBE~:

\begin{enumerate}
\def\labelenumi{\alph{enumi})}
\item
  Toute personne physique ou
\item
  Toute personne morale ou
\item
  Tout organisme assimilé à une personne morale en vertu du droit dont
  il relève
\end{enumerate}
\end{minipage} & G3/99, motif 9 \\
& - Conjointement & G3/99, motif 3 \\
& - Un homme de paille & G3/97, sommaire 1a \\
\multirow{2}{*}{Ne peut pas former opposition} & - Le titulaire du
brevet européen & G9/93 \\
& - Un homme de paille utilisé pour cacher le titulaire du brevet
européen & G3/97, sommaire 1c \\
\textbf{R76(2)} & \textbf{Mesure~?} & \\
Pouvant être mise en cause & - \ul{principe~:} Les revendications mises
en cause dans l'acte d'opposition & G9/91 motif 10 \\
& - Une revendication «~product-by-process~» d'une revendication de
procédé mise en cause & T525/96 motif 4.6 \\
& - Une revendication dépendante d'une revendication mise en cause &
G9/91 motif 11 \\
& \textbf{Faits et preuves~?} & \\
Pouvant être examiné & - Un document cité à la procédure d'examen &
T198/88, T536/88 \\
& - \ul{principe~:} Un usage antérieur comprenant la date, l'objet et
les circonstances & T328/87 \\
Ne pouvant pas être examiné & - Une antériorité pertinente présentée
trop tardivement & T156/84 \\
& - Un usage antérieur au stade du recours & T534/83 \\
\textbf{R80} & \textbf{Modification~?} & \\
Admissible & - Revendications, description, dessins pour répondre à un
motif A100 même s'il n'est pas invoqué & R80 \\
& - Tant que l'examen de l'opposition n'est pas presque terminé &
T406/86 sommaire 2 \\
& - Ne respecte pas l'unité d'invention & G1/91 sommaire \\
Non admissible & - Pas opportune, pas nécessaire & T406/86 sommaire 1 \\
& - Revendication modifiée pas claire & G3/14 exergue \\
& \textbf{Niveau de preuve~?} & \\
En faveur du breveté & Effet technique sur l'étendu des revendications &
T270/90 \\
Balance de probabilité & Usage antérieur par un tier & T1464/05 \\
& Usage antérieur par le breveté & \\
Au-delà de tout doute raisonnable & Usage antérieur de l'opposant & \\
\textbf{R77} & \textbf{Décision sur irrecevabilité~?} & \\
Admissible & - décision par un agent de formalité pour le compte de la
division d'opposition & G1/02 \\
& - accepté par un agent de formalité et rejet par la division
d'opposition & T222/85 \\
& - par la chambre de recours & T458/22 \\
& - à tous les stades de la procédure & T522/94 \\
\textbf{(~?)} & \textbf{Transfert~?} & \\
Admissible & - avec la cession d'un département sans personnalité
juridique & G4/88 \\
Non admissible & - avec une filiale d'une holding avec une personnalité
juridique & G2/04 \\
\end{longtable}

\begin{longtable}[]{@{}
  >{\raggedright\arraybackslash}p{(\columnwidth - 4\tabcolsep) * \real{0.2345}}
  >{\raggedright\arraybackslash}p{(\columnwidth - 4\tabcolsep) * \real{0.5759}}
  >{\raggedright\arraybackslash}p{(\columnwidth - 4\tabcolsep) * \real{0.1896}}@{}}
\toprule\noalign{}
\begin{minipage}[b]{\linewidth}\raggedright
\textbf{A100}
\end{minipage} & \begin{minipage}[b]{\linewidth}\raggedright
\textbf{Motif~?}
\end{minipage} & \begin{minipage}[b]{\linewidth}\raggedright
\end{minipage} \\
\midrule\noalign{}
\endhead
\bottomrule\noalign{}
\endlastfoot
\multirow{2}{*}{pouvant être soulevé} & - Un motif d'opposition non
soulevé qui s'oppose de prime abord au maintien du brevet & G10/91
sommaire 1 \\
& - Un motif présenté tardivement & D-V,2.2 \\
\end{longtable}

\begin{longtable}[]{@{}
  >{\raggedright\arraybackslash}p{(\columnwidth - 4\tabcolsep) * \real{0.2345}}
  >{\raggedright\arraybackslash}p{(\columnwidth - 4\tabcolsep) * \real{0.5782}}
  >{\raggedright\arraybackslash}p{(\columnwidth - 4\tabcolsep) * \real{0.1874}}@{}}
\toprule\noalign{}
\begin{minipage}[b]{\linewidth}\raggedright
\textbf{A101}
\end{minipage} & \begin{minipage}[b]{\linewidth}\raggedright
\textbf{Décision sur opposition~?}
\end{minipage} & \begin{minipage}[b]{\linewidth}\raggedright
\end{minipage} \\
\midrule\noalign{}
\endhead
\bottomrule\noalign{}
\endlastfoot
admissible & - d'une requête subsidiaire & T234/86 motif 2 \\
& - un recours sur la décision faute de réponse dans les délais & G1/88
sommaire \\
Non admissible & - \emph{tacet consentire videtur} & T861/16 exergue \\
\end{longtable}

\begin{longtable}[]{@{}
  >{\raggedright\arraybackslash}p{(\columnwidth - 4\tabcolsep) * \real{0.2345}}
  >{\raggedright\arraybackslash}p{(\columnwidth - 4\tabcolsep) * \real{0.5782}}
  >{\raggedright\arraybackslash}p{(\columnwidth - 4\tabcolsep) * \real{0.1874}}@{}}
\toprule\noalign{}
\begin{minipage}[b]{\linewidth}\raggedright
\textbf{A104}
\end{minipage} & \begin{minipage}[b]{\linewidth}\raggedright
\textbf{Frais~?}
\end{minipage} & \begin{minipage}[b]{\linewidth}\raggedright
\end{minipage} \\
\midrule\noalign{}
\endhead
\bottomrule\noalign{}
\endlastfoot
A l'encontre du responsable du retard & De nouvelles pièces présentées
tardivement & T867/92 \\
& Document tardifs & T117/86 \\
& No show & T930/92 \\
& (~?) & T617/20 \\
\end{longtable}

\begin{enumerate}
\def\labelenumi{\arabic{enumi}.}
\setcounter{enumi}{11}
\item
  \textbf{\ul{LE RECOURS}}
\end{enumerate}

\begin{longtable}[]{@{}
  >{\raggedright\arraybackslash}p{(\columnwidth - 4\tabcolsep) * \real{0.2345}}
  >{\raggedright\arraybackslash}p{(\columnwidth - 4\tabcolsep) * \real{0.5780}}
  >{\raggedright\arraybackslash}p{(\columnwidth - 4\tabcolsep) * \real{0.1875}}@{}}
\toprule\noalign{}
\begin{minipage}[b]{\linewidth}\raggedright
\textbf{A105}
\end{minipage} & \begin{minipage}[b]{\linewidth}\raggedright
\textbf{Qui~?}
\end{minipage} & \begin{minipage}[b]{\linewidth}\raggedright
\end{minipage} \\
\midrule\noalign{}
\endhead
\bottomrule\noalign{}
\endlastfoot
\multirow{2}{*}{Peut intervenir en opposition et en recours} & Tout
tiers après qu'une action en contrefaçon fondée sur un brevet EP a été
introduite à son encontre & Article 105(1)a) \\
& \begin{minipage}[t]{\linewidth}\raggedright
Tout tiers après

\begin{enumerate}
\def\labelenumi{(\roman{enumi})}
\item
  Avoir été requis par le titulaire du brevet de cesser la contrefaçon
\item
  Engagé une action en déclaration de non-contrefaçon
\end{enumerate}
\end{minipage} & Article 105(1)b) \\
\multirow{2}{*}{Ne peut pas intervenir en opposition et en recours} & Un
tiers après avoir été informé d'\ul{une réserve} sur une requête de
cessation de la contrefaçon & T392/97, motif 2.3 \\
& Un tiers après qu'une action en contrefaçon soit fondée sur \ul{un
brevet d'un état d'extension} & T1196/08, exergue 2 \\
Ne peut pas intervenir en recours & Un opposant qui n'a pas formé
recours & T1038/00 \\
& \textbf{Délai~?} & \\
En opposition et en recours & - 3 mois / action en déclaration de
non-contrefaçon & règle 89(1) \\
& - 3 mois / action en contrefaçon & règle 89(1) \\
En opposition & - Tant qu'une opposition recevable est encore en
instance & \\
En recours & - tant que le recours est encore en instance & G1/94
sommaire \\
& - pendant le délai de recours de 2 mois & G4/91 \\
& \textbf{Conséquence juridique~?} & \\
\multirow{3}{*}{en opposition} & - intervenant acquiert la qualité
d'opposant & article 105(2) \\
& - intervenant peut poursuivre la procédure d'opposition même si les
autres opposants retirent leur oppositions & G3/04 motif 10 \\
& - intervenant peut former recours & G3/04 motif 4 \\
en recours & - intervenant acquiert la qualité d'opposant mais pas de
requérant & G3/04, motif 4 \\
\end{longtable}

\begin{longtable}[]{@{}
  >{\raggedright\arraybackslash}p{(\columnwidth - 4\tabcolsep) * \real{0.2518}}
  >{\raggedright\arraybackslash}p{(\columnwidth - 4\tabcolsep) * \real{0.6213}}
  >{\raggedright\arraybackslash}p{(\columnwidth - 4\tabcolsep) * \real{0.1270}}@{}}
\toprule\noalign{}
\begin{minipage}[b]{\linewidth}\raggedright
\textbf{A105bis}
\end{minipage} & \begin{minipage}[b]{\linewidth}\raggedright
\end{minipage} & \begin{minipage}[b]{\linewidth}\raggedright
\end{minipage} \\
\midrule\noalign{}
\endhead
\bottomrule\noalign{}
\endlastfoot
& & \\
\end{longtable}

\begin{longtable}[]{@{}
  >{\raggedright\arraybackslash}p{(\columnwidth - 4\tabcolsep) * \real{0.2506}}
  >{\raggedright\arraybackslash}p{(\columnwidth - 4\tabcolsep) * \real{0.6201}}
  >{\raggedright\arraybackslash}p{(\columnwidth - 4\tabcolsep) * \real{0.1293}}@{}}
\toprule\noalign{}
\begin{minipage}[b]{\linewidth}\raggedright
\textbf{A106}
\end{minipage} & \begin{minipage}[b]{\linewidth}\raggedright
\textbf{Qui~?}
\end{minipage} & \begin{minipage}[b]{\linewidth}\raggedright
\end{minipage} \\
\midrule\noalign{}
\endhead
\bottomrule\noalign{}
\endlastfoot
peut former recours & - deux parties dont une nouvelle partie après
cessions d'une partie du brevet EP & T158/19 \\
& \textbf{Transfert~?} & \\
Admissible & - au nom de la société absorbée & T1421/05 \\
\textbf{A107} & \textbf{Poursuite~?} & \\
Non admissible & - une partie observatrice au sens A107 après le retrait
du recours par le requérant & G2/91 sommaire 2 \\
\textbf{A108} & \textbf{Délai~?} & \\
& - acte de recours 2 mois / signification de la décision & A108 \\
& - mémoire recours 4 mois / signification de la décision & A108 \\
& \textbf{Actes~?} & \\
pas recevable & - payer la taxe de recours mais pas de recours écrit &
T371/92 \\
& - déposer le recours mais pas payer la taxe de recours & T2406/16 \\
\textbf{A109} & \textbf{Fondement~?} & \\
Pas fondé & - raisonnement pas cohérent & T922/05 \\
& - raisonnement pas assez détaillé & T220/83 \\
& =\textgreater{} pas de report de la décision de recours & T592/15 \\
Fondé & - objections surmontées & T139/87 \\
& - refus injustifié & T64/07 \\
\textbf{R100} & \textbf{Examen~?} & \\
Pas recevable & - un simple renvoi à une question posée / au dossier en
1ere instance & R8/16 \\
\textbf{A12 RPCR} & \textbf{Moyens invoqués~?} & \\
Recevable & - \ul{principe}~: reprendre ensemble des moyens invoqués en
1ere instance & A12(3) RPCR \\
Pas recevable & - des moyens invoqués en 1ere instance mais pas au stade
du recours & T644/20 exergue 1 \\
& - une requête qui aurait dû être présentée en 1ere instance &
T144/09 \\
& \textbf{Nouveaux motifs d'opposition~?} & \\
recevables & - \ul{principe~}: nouveaux motifs seulement avec l'accord
du titulaire & G10/91 \\
& - un nouveau motif soulevé par l'intervenant & G1/94 \\
& - un motif de nouveauté sur un ETTP lorsqu'un motif d'activité
inventive a été soulevé sur l'ETTP & G7/95 sommaire \\
& - un motif d'activité inventive sur un ETTP lorsqu'un motif de
nouveauté a été soulevé sur l'ETTP & T131/01 sommaire \\
& - un motif soulevé par un autre opposant & (~?) \\
& \textbf{Non reformatio in pejus~?} & \\
S'applique & - supprimer une caractéristique & G1/99 sommaire \\
& - introduire une extension dans la limite de 123(3) & \\
Ne s'applique pas & - introduire une limitation supplémentaire lorsque
la modification est 123(3) & \\
& - recours sur examen & G10/93 sommaire \\
\textbf{A111} & \textbf{Renvoi~?} & \\
Pris en compte & - requête auxiliaire tardive au recours & T2194/22 \\
& - un texte complet sans R71(6) & T2558/18 \\
& \textbf{Autorité de la chose jugée~?} & \\
S'applique & - une demande divisionnaire d'une demande parente jugée en
recours & T51/08 \\
& - citer l'ETTP jugé en recours dans la description & T450/97 sommaire
2 \\
& - décision du 1\textsuperscript{er} recours lie la décision du
2\textsuperscript{e} recours & T720/93 (~?) \\
\end{longtable}

\begin{enumerate}
\def\labelenumi{\arabic{enumi}.}
\setcounter{enumi}{12}
\item
  \textbf{\ul{LA REQUETE EN REVISION}}
\end{enumerate}

\begin{longtable}[]{@{}
  >{\raggedright\arraybackslash}p{(\columnwidth - 4\tabcolsep) * \real{0.2353}}
  >{\raggedright\arraybackslash}p{(\columnwidth - 4\tabcolsep) * \real{0.6050}}
  >{\raggedright\arraybackslash}p{(\columnwidth - 4\tabcolsep) * \real{0.1596}}@{}}
\toprule\noalign{}
\begin{minipage}[b]{\linewidth}\raggedright
\textbf{A112}
\end{minipage} & \begin{minipage}[b]{\linewidth}\raggedright
\textbf{Fondement~?}
\end{minipage} & \begin{minipage}[b]{\linewidth}\raggedright
\end{minipage} \\
\midrule\noalign{}
\endhead
\bottomrule\noalign{}
\endlastfoot
Admissible & - question de droit soumise par les chambres de recours &
A22(1)a) \\
& - question de droit par le Président de l'OEB & A22(1)b) \\
& - requête en révision & A22(1)c) \\
\textbf{A112bis} & \textbf{Fondement~?} & \\
Admissible & - conflit d'intérêt & A112bis(2)a) \\
& - pas un membre d'une chambre de recours & A112bis(2)b) \\
& - violation du droit d'être entendu & A112bis(2)c) \\
& - vice fondamental de procédure & A112bis(2)d) \\
& - corruption d'un membre & A112bis(2)e) \\
& - pas de mention d'un argument déterminant et évident pour le lecteur
moyen & R7/22 \\
& - absence de la notification d'une décision d'une chambre de recours &
R7/09 \\
Non admissible & - «~droit d'être entendu~» n'est pas l'occasion de
rediscuter des décisions de la chambre de recours & R1/08 motif 2.1 \\
& - pas de mention d'argument dans la décision de la chambre de recours
ie. présomption de considération & R10/20 \\
& - une réponse d'une chambre de recours laissée équivoque & R6/22 \\
\end{longtable}

\begin{enumerate}
\def\labelenumi{\arabic{enumi}.}
\setcounter{enumi}{13}
\item
  \textbf{\ul{LA RESTITUTIO IN INTEGRUM}}
\end{enumerate}

\begin{longtable}[]{@{}
  >{\raggedright\arraybackslash}p{(\columnwidth - 4\tabcolsep) * \real{0.2506}}
  >{\raggedright\arraybackslash}p{(\columnwidth - 4\tabcolsep) * \real{0.6201}}
  >{\raggedright\arraybackslash}p{(\columnwidth - 4\tabcolsep) * \real{0.1293}}@{}}
\toprule\noalign{}
\begin{minipage}[b]{\linewidth}\raggedright
\textbf{A122}
\end{minipage} & \begin{minipage}[b]{\linewidth}\raggedright
\textbf{Qui~?}
\end{minipage} & \begin{minipage}[b]{\linewidth}\raggedright
\end{minipage} \\
\midrule\noalign{}
\endhead
\bottomrule\noalign{}
\endlastfoot
Peut faire une restitutio & - le demandeur, le titulaire ou le
mandataire & \\
& - l'héritier & T15/01 sommaire 2 \\
& - l'opposant pour déposer le mémoire de recours & G1/86 \\
& \textbf{Délai~?} & \\
& - 2 mois / cessation de l'empêchement & \\
& - 12 mois / expiration du délai non observé & \\
& \textbf{Preuve de vigilance~?} & \\
Non admissible & - présomption de connaissance & J1/20 \\
& - une restauration auprès d'un autre office que l'OEB exigeant un
critère moins contraignant que la diligence requise & J13/16 \\
& \textbf{Condition d'application~?} & \\
remplies & - \ul{principe~:} une erreur isolée dans un système
normalement satisfaisant & J2/86 sommaire \\
& - pas nécessaire de respecter le délai de restitutio pour fournir les
preuves & T324/90 sommaire \\
Non remplies & - ne pas donner tous les motifs immédiatement &
T1874/23 \\
& - incapacité physique (par exemple chute de neige~) & J7/20 \\
& - une difficulté financière & J22/88 \\
& - un congé d'un mandataire & J12/10 \\
& - une erreur de calcul d'un mandataire & J41/92 \\
& - une erreur de calcul d'un assistant, intérimaire & T430/06 \\
& - un courriel dans la boîte spam & T1289/10 \\
& - un bug dans un programme d'informatique & T473/07 \\
& - une erreur d'un prestataire de paiement d'annuité & T1882/23 \\
& - empêchement par l'ignorance de la possibilité de restitutio &
T178/23 \\
\end{longtable}

\end{document}
